\section*{Reply to Examiner No. 1}

\begin{replyheader}
\end{replyheader}  

First, I would like to express my appreciation towards Examiner No. 1 for his \textbf{over-all positive comments}.
I also appreciate the fact that Examiner No 1 includes
  not only his comments in the examiner's report
  but also helpful suggestions inside the original thesis.
  
The examiner asked no direct question in his comments, so the following are my replies to his comments as listed in the returned thesis:

\rowcolors{2}{gray!25}{white}    
\noindent
\begin{longtable}[c]{p{0.08\textwidth}|p{0.08\textwidth}|p{0.36\textwidth}|p{0.36\textwidth}}
\textbf{Page} & \textbf{Line} & \textbf{Gramatical Errors} & \textbf{Corrections} \\
 \hline
\endhead
4 & 5 & missing references & citation added \\
4 & 8 & missing references & citation added \\
5 & 13 & techniques ... has & techniques ... have \\
6 & 5 & many fold & manyfold \\
14 & 5 & Assuming the number of these elementary back scatterers is sufficiently large, & Assuming these elementary back scatterers are large in number and independent in nature, \\
%**IVM: the above should be 'number' (you had written 'numbers')
14 & Eq 2.1 & $pdf(A_x)=\frac{1}{\sqrt{\pi}\sigma} e^{\frac{A_x^2}{\sigma^2}}$ & $pdf(A_x)=\frac{1}{\sqrt{\pi}\sigma} e^{-\frac{A_x^2}{\sigma^2}}$ \\
%**IVM: did you mention this in the thesis? If so, it might be good to mention.
\end{longtable}    
\rowcolors{1}{white}{}

\replyToComment
    {Page 14: With reference to Equations 2.1, 2.2, 2.3 what are the range of $A_x$, $A$ and $I$}
    {The range of $A_x$ is from $(-\infty, \infty)$, and the range of both $A$ and $I$ is from $(0,\infty)$} 

\replyToComment
  {Page 14: Does the integration of pdf for $A_x,A,I$ sum to 1?}
  {With the defined range above, YES they are all sum to 1. This is easy to verify as they belong to the classes of Normal, Chi-Squared (2 degree-of-freedom) and Exponential Distribution respectively.}

\rowcolors{2}{gray!25}{white}    
\noindent
\begin{longtable}[c]{p{0.08\textwidth}|p{0.08\textwidth}|p{0.36\textwidth}|p{0.36\textwidth}}
\textbf{Page} & \textbf{Line} & \textbf{Gramatical Errors} & \textbf{Corrections \& Comments} \\
 \hline
 \endhead
15 & 6 & missing explaination for $\vec{E},B$ & explaination added \\
15 & 12 & x,y plane & x \& y planes \\
15 & -2 & $2 \frac{\xi_x \xi_y}{E_x^0 E_y^0} \cos (\Delta \phi)$ & $2 \frac{\xi_x(t) \xi_y(t)}{E_x^0 E_y^0} \cos (\Delta \phi)$ \\
16 & 2 & The parameters ... are: & The parameters ... are: the intensity orientation angle ($\theta$), the ellipticity angle ($\Psi$) and the orientation angle ($\chi$) \\
17 & 4 & machenics & mechanics \\
17 & -1 & missing explaination for $\Re,\Im$ & explaination added \\
20 & -1 & font size for Eqns 2.10 and 2.11 is different from the rest of the thesis & This is due to space restriction, where the matrix representation in the normal font size would require a larger width that that is available in one line \\
33 & -10 & Eqn. [2.15) & Eqn. 2.15 \\
34 & 10 & how did you get this Eqn? & References have now been added to show the source of the equation \\
35 & 10 & which distribution is more appropriate & The most appropriate distribution is that of the ``true signal'', which of course, is unknown, so we are trying to model it \\
43 & -3 & $pdf(A_1)=2A_1e^{A_1^2}$ & $pdf(A_1)=2A_1e^{-A_1^2}$ \\
76 & -4 & chapter 2 & Chapter 2 \\
129 & 1 & cahpter 5 & Chapter 5 \\
129 & 6 & theoretical results & theoretical contributions \\
129 & 21 & this thesis unite & this thesis unites \\
130 & 1 & articles that has been & articles that have been \\
132 & 14 & an important benefits & several important benefits \\
\end{longtable}    
\rowcolors{1}{white}{}

