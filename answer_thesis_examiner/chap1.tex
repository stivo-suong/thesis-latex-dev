\section*{Reply to Examiner No. 1}

\begin{replyheader}
\end{replyheader}  

First, I would like to express my appreciation towards Examiner No. 1 for his \textbf{over-all positive comments}.
I also appreciate the fact that Examiner No 1 includes
  not only his comments in the examiner's report
  but also helpful suggestions inside the original thesis.
There was no direct question in the report, and followings are my replies to his comments in the returned thesis.

\rowcolors{2}{gray!25}{white}    
\noindent
\begin{longtable}[c]{p{0.1\textwidth}|p{0.1\textwidth}|p{0.3\textwidth}|p{0.3\textwidth}}
\textbf{Page} & \textbf{Line} & \textbf{Gramatical Errors} & \textbf{Corrections} \\
 \hline
 \endhead
4 & 5 & missing references & citation added \\
4 & 8 & missing references & citation added \\
5 & 13 & techniques ... has & techniques ... have \\
6 & 5 & many fold & manyfold \\
14 & Eq 2.1 & $pdf(A_x)=\frac{1}{\sqrt{\pi}\sigma} e^{\frac{A_x^2}{\sigma^2}}$ & $pdf(A_x)=\frac{1}{\sqrt{\pi}\sigma} e^{-\frac{A_x^2}{\sigma^2}}$ \\
\end{longtable}    

\rowcolors{1}{white}{}
\replyToComment
    {Page 14: With reference to Equations 2.1, 2.2, 2.3 what are the range of $A_x$, $A$ and $I$}
    {The range of $A_x$ is from $(-\infty, \infty)$, and the range of both $A$ and $I$ is from $(0,\infty)$} 

\replyToComment
  {Page 14: Does the integration of pdf for $A_x,A,I$ sum to 1?}
  {With the defined range above, YES they are all sum to 1. This is easy to verify as they belong to the classes of Normal, Chi-Squared (2 degree-of-freedom) and Exponential Distribution respectively.}

\rowcolors{2}{gray!25}{white}    
\noindent
\begin{longtable}[c]{p{0.1\textwidth}|p{0.1\textwidth}|p{0.3\textwidth}|p{0.3\textwidth}}
\textbf{Page} & \textbf{Line} & \textbf{Gramatical Errors} & \textbf{Corrections} \\
 \hline
 \endhead
15 & 6 & missing explaination for $E,B$ & explaination added \\
\end{longtable}    

In page 14 of the returned thesis, Examiner No. 1 wonders ``Does the intagration of the PDF given for $A_x, A, I$ sum up to 1?''.
First I would like to thank Examiner No. 2 for raising such a question.
By going through his question, I realized that a negative sign were missed in the equation for $pdf(A_x)$.
With that corrected, let me repeat the Equations here

\begin{align}
pdf(A_x) &= \frac{1}{\sqrt{\pi} \sigma} e^{- \frac{A_x^2}{\sigma^2}} \\
pdf(A)   &= \frac{2A}{\sigma^2} e^{\frac{-A^2}{\sigma^2}} \\
pdf(A_1) &= 2A_1 e^{-A_1^2} \\
pdf(I)   &= \frac{1}{\sigma^2} e^{-\frac{I}{\sigma^2}}
\end{align}

The integration of these becomes
\begin{align}
cdf(A_x) &= \int_{-\infty}^{\infty} \frac{1}{\sqrt{\pi} \sigma} e^{- \frac{A_x^2}{\sigma^2}}  \; \mathrm{d}A_x = \frac{1}{2} erf \left( \frac{A_x}{\sigma} \right)  \Big|_{-\infty}^{\infty} &= 1 \\
cdf(A) &= \int_0^{\infty} \frac{2A}{\sigma^2} e^{\frac{-A^2}{\sigma^2}}  \; \mathrm{d}A = -e^{\frac{-A^2}{\sigma^2}} \Big|_0^{\infty} &= 1\\
cdf(A_1) &= \int_0^{\infty} 2A_1 e^{-A_1^2} \; \mathrm{d}A_1 = -e^{-A_1^2} \Big|_0^{\infty} &= 1\\
cdf(I) &= \int_0^{\infty} \frac{1}{\sigma^2} e^{-\frac{I}{\sigma^2}} \; \mathrm{d}A_1 = -e^{-\frac{I}{\sigma^2}}  \Big|_0^{\infty} &= 1
\end{align}


For the first equation, consider the standard PDF equation for normal distribution centering around $\mu$ and having variance $\sigma$ given below

\begin{align}
\textbf{Normal Distribution: } & pdf(x) = \frac{1}{\sigma \sqrt{2\pi}} e^{- \frac{(x-\mu)^2}{2 \sigma^2}} \\
\textbf{Chi-Squared Distribution (dof=2): } & pdf(x) = \frac{e^{-x/2}}{2} \\ %pdf(x) = \frac{x^{\frac{k}{2}-1}e^{-x/2}}{\Gamma(k/2)2^{k/2}} \\
\textbf{Exponential Distribution: } & pdf(x) = \lambda e^{- \lambda x}
\end{align}

\begin{align}
\textbf{Normal Distribution: } & cdf(x) = \frac{1}{2} \left[ 1 + erf \left( \frac{x-\mu}{\sqrt{2\sigma^2}} \right) \right] \\
\textbf{Chi-Squared Distribution (dof=2): } & cdf(x) = 1 - e^{-x/2}\\
\textbf{Exponential Distribution: } & cdf(x) = 1 - e^{- \lambda x}
\end{align}

Clearly, $A_x$ follows the normal distribution with expectation $avg(A_x)=0$ and variance $var(A_x)=\frac{\sigma}{\sqrt{2}}$.

Next consider the standard PDF equation for chi-square distribution with $k$ degree of freedom given below:

\begin{align}
pdf(x) = \frac{x^{\frac{k}{2}-1}e^{-x/2}}{\Gamma(k/2)2^{k/2}}
\end{align}

Set $k=2$, and thus $\Gamma(k/2)=\Gamma(1)=1$, then:

\begin{align}
pdf(x) = \frac{e^{-x/2}}{2} 
\end{align}

Set $\frac{x}{2} = \frac{A^2}{\sigma^2}$ or $x = \frac{2A^2}{\sigma^2}$, thus $\frac{dx}{dA} = \frac{4A}{\sigma^2}$.
Variable change theorem give us:

\begin{align}
pdf(A) = \frac{e^{-A^2/\sigma^2}}{2} \frac{4A}{\sigma^2} = \frac{2A}{\sigma^2} e^{-\frac{A^2}{\sigma^2}}
\end{align}


For the $pdf(I)$, if we set $\lambda = \frac{1}{\sigma^2}$ then the equation turns out to be the standard PDF equation for the Exponential Distribution 

\begin{align}
pdf(x) = \lambda e^{- \lambda x}
\end{align}
