\section*{Reply to Examiner No. 3}

\begin{replyheader}
\end{replyheader}  

Since Examiner No. 3 presented his comments and suggestions in the order of the thesis' chapters, 
  I would follow his suite.

\subsubsection*{1. Overall Organization and presentation}

\replyToComment
    {Page ii of the Contents section is missing}
    {Page numbers are added for Contents section}

\replyToComment
    {Usually, a Glossary is put at the end of the document}
    {Updated (now the Glossary section stays at the end of the thesis)}

\replyToComment
    {Page xiii: List of source codes. These are actually pseudocodes and not source codes.}
    {Updated (now called ``List of Pseudocodes'')}

%Examiner No. 3 noted: ``The overall organization of the thesis is appropriate''.
%Still minor editions have been included to address his  other concerns:

%I appreciate the fact that the examiner noted: ``The overall organization of the thesis is appropriate''.
%Besides the issue of appropriate headings for chapter 3 and 4, which will be discussed in their respective section, minor editions have been included to address his other concerns.

%\begin{itemize}
%  \item \textbf{Page ii of the Contents section is missing:} This were probably an administrative mistake and oversight from us. Please accept our apologies!
%  \item \textbf{Usually, a Glossary is put at the end of the document: } Updated (now the Glossary section stays at the end of the thesis)
%  \item \textbf{Some page number missing: } Updated. 
%  \item \textbf{List of source codes: } Updated (now ``List of Pseudocodes'' as suggested.)
%\end{itemize}

\subsubsection*{2. Abstract}

\replyToComment
    {Since these models are the key contributions of the thesis, it would be good to give a separate name to each of these models}
    {The models' names are added into the abstract}

\replyToComment
    {The abstract lists the benefits of these models but not the shortcomings.}
    {The shortcomings have also been included into the abstract.}

%The abstract has been updated so that:
%\begin{enumerate}[topsep=0pt]
%  \item Each of the proposed models have their own name and
%  \item Not only the benefits of the proposed models are listed, but so are their shortcomings.
%\end{enumerate}

\subsubsection*{3. Chapter 1}

\replyToComment
    {Table captions appears at the top of a table, rather than below it}
    {The suggestion has been updated in this chapter, as well as through out the whole thesis}

\replyToComment
    {No references throughout this chapter.}
    {References have been included in this chapter to increase its persuasion power.}

\replyToComment
    {you are trying to propose an additive and homoskedastic model for the data. What makes you think that thsi model is appropriate?}
    {Before an answer is laid out, I wish to make a few points.
First, I thank the Examiner to raise this question, which no doubt is an important one. 
Second, while it is important to judge the end result, 
  I am not quite sure that should be done at the very first chapter.
In fact, a large portion of the final chapter is devoted towards such a discussion.

To directly address the Examiner's concern, let me iterate a few points that has been noted in the thesis.
First the generic model proposed in this thesis is applicable to multi-dimensional POLSAR data.
Second, when the multi-dimensional data is collapsed into one dimensional, 
  the specific case of the proposed model matches perfectly with the model proposed for SAR by Arsenault \cite{Arsenault_JOptSocAm_1976}. 
Thus, I believe, the proposed generic model is at least as appropriate as the widely-accepted model cited above.
}

\replyToComment
    {In section 1.4, it may be worthwhile to mention that the author's contribution had been published in various venues, together with a list of author's publications.}
    {The list of publication is now included in this section as well.}

%The examiner suggests that ``Table captions appears at the top of a table, rather than below it''. 
%This suggestion has been updated in this chapter, as well as through out the whole thesis.
%Also references are included to increase the persuasion power of the claims.
%The list of publication is also now included in this chapter, in addition to the existing one found in the final chapter.
%
%At this juncture, I would also wish to address one of Examiner No. 3's main concern.
%He wonders: ``What makes you think that this model is appropriate?''.
%Before the answer is laid out , I wish to make a few points.
%First, I thank the Examiner to raise this question, which no doubt is an important one. 
%Second, while it is important to judge the end result, 
%  I am not quite sure that should be done at the very first chapter.
%In fact, a large portion of the final chapter is devoted towards such a discussion.
%
%To directly address the Examiner's concern, let me iterate a few points that has been noted in the thesis.
%First the generic model proposed in this thesis is applicable to multi-dimensional POLSAR data.
%Second, when the multi-dimensional data is collapsed into one dimensional, 
%  the specific case of the proposed model matches perfectly with the model proposed for SAR by Arsenault \cite{Arsenault_JOptSocAm_1976}. 
%Thus, I believe, the proposed generic model is at least as appropriate as the widely-accepted model cited above.

%It is termed as: ``it doesn't mean that by fitting a homoskedastic model to heteroskedastic data, these algorithm will give correct results''.
%The approach described in this thesis differs slightly from ``fitting a homoskedastic model to heteroskedastic data''.
%Here, I proposed to used log-transformation which has long been noted as converting the data from mutiplicative to additive domain.
%The model for the data in this additive domain has been derived and accepted in \cite{Arsenault_JOptSocAm_1976}.
%My work, in simple terms, suggests that in addition to additive properties, the transformed domain also exhibits homoskedastic properties as contrast to the heteroskedastic properties in the original domain.
%These properties are investigated in this thesis to show that it does offers several benefits (for example: in designing and evaluating speckle filters).
%It should be noted that the model for SAR used in this thesis is exactly the same as the additive model proposed by Arsenault \cite{Arsenault_JOptSocAm_1976} and is mathematically equivalent to the model multiplicative model for SAR data in its original domain.

\subsubsection*{4. Chapter 2}

\replyToComment
    {Perhaps the theory section can be moved to a separate chapter, so that more space can be dedicated to the discussion of related work}
    {In a bid of not confusing other examiners, however, I have decided not to move the theory section into a separate chapter.
Still, this chapter is heavily updated so that: 
\vspace{-3mm}
\begin{enumerate}
  \item Section 2.1 is shortened considerably and 
  \item Section 2.2 is extended significantly. 
\end{enumerate}
These editions, I believe, lead to a better balance between describing the nature of the data and reviewing the related work.
}

%For Chapter 2, Examiner No. 3's suggests: ``Perhaps the theory section can be moved to a separate chapter, so that more space can be dedicated to the discussion of related work''.
%In a bid of not confusing other examiners, however, I have decided not to move the theory section into a separate chapter.
%Still, this chapter is heavily updated so that: 
%\vspace{-3mm}
%\begin{enumerate}
%  \item Section 2.1 is shortened considerably and 
%  \item Section 2.2 is extended significantly. 
%\end{enumerate}
%
%This I believe leads to a better balance between describing the nature of the data and reviewing the related work.
%This also help growing the space dedicated to the discussion of ``Other approaches that have recently been pursued in the research community (SAR speckle filters)''.

\subsubsection*{5. Chapter 3}

\replyToComment
    {how can the statement on page 44 line 1 ``... intensity are equal to a scaled version of these unit variables, specifically $A=\sigma A_1$ and $I=\sigma^2 I_1$'' be true?}
    {the statement in page 44 line 1 should be intepreted strictly in the context of what preceeds it. 
That is mathematically  speaking $\sigma$ is just a mathematical constant, which makes the statement true. 
Any resemblence of it with the SAR signal is only established in the next paragraph onwards.
In fact, line 7 page 44 clearly indicated where $\sigma$ can be considered as constant (i.e. spatial homogeneous area).
}

\replyToComment
    {why homoskedasticity is valid, and what are the assumption in this analysis?}
    {Spartial homogeneity is the assumption used in the analysis thus far. 
Its limitation is acknowledged in the very next sentence, as the Examiner No. 3 pointed out precisely: ``Over heterogeneous area, ... $\sigma$ varies significantly ...''.

This assumption actually is not very restrictive.
It should be noted that at the level of each radar physical resolution cell, 
  the value measured in SAR is not deterministic.
For all practical purposes, the single measured value is to be considered as the result of a stochastic process, with one ``true'' signal $\sigma$.
In that sense, the assumption is actually applicable to both homogeneous and heterogeneous areas.
}

\replyToComment
    {If $\sigma$ is not a constant, then $var(L_A)$ and $var(L_I)$ cannot be homoskedastic}
    {In general, the imaging area is heterogeneous, then $\sigma$ is no longer a constant from one resolution pixel to the next.
That much is clear.
Then, as reviewed in Chapter 2, there are many different ways to model $\sigma$ which in turns leads to many different models for the observable magnitude.
This changing $\sigma$ also leads to heteroskedasticity, 
  which the Examiner correctly pointed out in his comment
%``$var(L_A)$ and $var(L_I)$ cannot be homoskedastic'', 
  should we consider ``$var(L_A)$ and $var(L_I)$'' as the variation of the observables in an area.
However, if we consider ``$var(L_A)$ and $var(L_I)$'' as the deviation of the observables from their corresponding ``true signal'' at each physical resolution cell level, 
  then $\sigma$ is constant at each resolution pixel!
Consequently $var(L_A)$ and $var(L_I)$ will be independent of $\sigma$ (as described in Table 3.3).
In fact they are constant, which leads to homoskedasticity!
}

\replyToComment
    {If there are different models, the author should give a name to each of his models ...}
    {Yes there are several models proposed and the thesis is updated with names for each of them!}

%There are apparently two main concerns from Examiner No. 3's comments.
%The first is: ``how can the statement on page 44 line 1 ... be true?''.
%And second: ``why homoskedasticity is valid, and what are the assumption in this analysis?''
%
%First, the statement in page 44 line 1 should be intepreted strictly in the context of what preceeds it. 
%That is mathematically  speaking $\sigma$ is just a mathematical constant, which makes the statement true. 
%Any resemblence of it with the SAR signal is only established in the next paragraph onwards.
%In fact, the third line of the second paragraph in page 44 clearly indicated where $\sigma$ can be considered as constant (i.e. spatial homogeneous area or at each resolution cell).
%This is the assumption used in this analysis. 
%The assumption is acknowledged in the very next sentence, as the Examiner No. 3 pointed out precisely: ``Over heterogeneous area, ... $\sigma$ varies significantly ...''.
%If $\sigma$ is not a constant, then heteroskedastic ensues.
%The next section (i.e. Section 3.2) then outlines the effects of heteroskedasticity, 
%  which explains the needs for logarithmic transformation.  
%
%The assumption described above continues to hold in Section 3.3 and, of course, table 3.3.
%%In that very same vein, table 3.3 should be viewed under the assumption that $\sigma$ is constant (either in homogeneous area or in a single resolution cell).
%Under this condition, the variances of the log-transformed variables are then independent of $\sigma$.
%The point being made in the table is that: 
%  while variances are dependent on $\sigma$ in the original domain,
%  in the log transformed domain, they are independent of this value.
%
%The assumption actually is not very restrictive.
%It should be noted that at the level of each radar physical resolution cell, 
%  the value measured in SAR is not deterministic.
%For all practical purposes, the single measured value is to be considered as the result of a stochastic process, with one ``true'' signal $\sigma$.
%This model actually is applicable to both homogeneous and heterogeneous areas.
%
%Now, if the area is heterogeneous, then $\sigma$ is no longer a constant.
%In fact, as reviewed in Chapter 2, there are many different ways to model $\sigma$ which in turns leads to many different models for the observable magnitude.
%%This, of course, leads to different expression for the observable variances.
%This leads to heteroskedasticity, 
%  which the Examiner correctly pointed out ``$var(L_A)$ and $var(L_I)$ cannot be homoskedastic'', 
%  should we consider ``$var(L_A)$ and $var(L_I)$'' as the variation of the observables in an area.
%However, if we consider ``$var(L_A)$ and $var(L_I)$'' as the deviation of the observables from corresponding its ``true'' signal at each physical resolution cell level, 
%  then they will be independent of $\sigma$ (as described in Table 3.3).
%In fact they are constant, even as $sigma$ changes.
%This leads to homoskedasticity!
%
%The second point concerns the models and their names.
%This chapter has been updated and model names are given for each to clarify the concepts.
%The confusion is understandable, and partially my fault, in not highlighting the following points.
%First there are different models being proposed, in Section 3.4.
%And second, they are all however derived based on ONE basic model: the one based on ``the base-2 log-transformation of the SAR data.''
%Finally, there are differences in each of the models, even though, whether these differences are significant or not is, to be honest, rather subjective at my current state of understanding.

\subsubsection*{6. Chapter 4}

\replyToComment
    {When you have highly correlated data (i.e. homogeneous areas), the determinant $|\Sigma|$ will be very small, leading to a very narrow PDF in Equation 4.1.
Also the inverse $\Sigma^{-1}$ is ill-defined, leading to large errors in your model.
Please justify the situations, if any, your model will not work well.}
    {Equation 4.1 is the PDF for the circular complex Gaussian distribution, which is widely used in POLSAR.
Its form, as repeated from equation 4.1, is written as:

%\begin{align}
\begin{center}
  $pdf(s;\Sigma)=\frac{1}{\pi^d |\Sigma|} e^{-s^{*T}\Sigma^{-1}s}$
\end{center}
%\end{align}  

The equation is also ill-defined where $|\Sigma|=0$, which is also the same time that $\Sigma^{-1}$ is ``ill-defined''.
In other words, the models proposed has the same assumption and validity with the widely-accepted circular complex Gaussian distribution model.
In POLSAR, $|\Sigma|=0$ most commonly happens when the dataset is in Single-Look format.
This restriction is clearly stated in the sentence that follows Equation 4.1: ``the covariance matrix is only defined on multiple data-points''.

It is partially for the ``narrow PDF and large errors'' concern that log-transformation is proposed.
Since the original domain is multiplicative, the range of small values is, as also observed by Examiner No. 3, commonly found but extremely limited ($|\Sigma|$ ranges from $0+$ to $1$).
Log transformed domain not only changes the nature from multiplicative to additive, but also give this ``small'' range $(0,1)$ a much widely space $(-\infty,0)$.
In other words, it helps to expand the ``narrow'' distribution depicted in Equation 4.1 (when $|\Sigma|$ is small) to become another distribution whose shape does not dependent on $|\Sigma|$, as depicted by the Equation below:

%\begin{align*}
\begin{center}
  $pdf( \ln | \langle ss^{*T} \rangle|, \Sigma) = \ln |\Sigma| - \ln(2L)d + \sum^{d-1}_{i=0} \Lambda (2L-2i)$
\end{center}
%\end{align*}  
}

\replyToComment
    {On page 66, $3^{rd}$ paragraph, last sentence, the author declares that ``A visual match is clearly observable ...''.
However, in Fig 4.2(a) ...}
    {I understand that the Examiner is concerned about the subjective quality of the ``visual match'' claim. 
For this, the section have been updated to include now a quantitative, and hence objective, measure of match.
These objective numbers should completely addresses his concerns, I believe.}

\replyToComment
    {The section heading 4.5.3 should be ``Effective Number-of-Looks'' instead of ``Effect Number-of-Looks''}
    {Updated accordingly!}

%Examiner No.3 expresses his doubt on two points.
%First, that ``the determinant $|\Sigma|$ will be very small, leading to a very narrow PDF in Equation 4.1''.
%And second that $\Sigma^{-1}$ is ill-defined.
%I wish to highlight that Equation 4.1 is the PDF for the circular complex Gaussian distribution, which is widely used in POLSAR.
%Its form, as repeated from equation 4.1, is written as:
%
%\begin{align*}
%  pdf(s;\Sigma)=\frac{1}{\pi^d |\Sigma|} e^{-s^{*T}\Sigma^{-1}s}
%\end{align*}  
%
%This equation is also ill-defined where $|\Sigma|=0$, which is also the only time that $\Sigma^{-1}$ is ``ill-defined''.
%In other words, the models proposed has the same assumption and validity with the widely-accepted circular complex Gaussian distribution model.
%In POLSAR, $|\Sigma|=0$ most commonly happens when the dataset is in Single-Look format.
%This restriction is clearly stated in the sentence that follows Equation 4.1: ``the covariance matrix is only defined on multiple data-points''.
%
%It is partially for the second concern that log-transformation is proposed.
%Since the original domain is multiplicative, the range of small values is, as also observed by Examiner No. 3, commonly found but extremely limited ($|\Sigma|$ ranges from $0+$ to $1$).
%Log transformed domain not only changes the nature from multiplicative to additive, but also give this ``small'' range $(0,1)$ a much widely space $(-\infty,0)$.
%In other words, it helps to expand the ``narrow'' distribution depicted in Equation 4.1 (when $|\Sigma|$ is small) to become another distribution whose shape does not dependent on $|\Sigma|$, as depicted by the Equation below.
%
%\begin{align*}
%  pdf( \ln | \langle ss^{*T} \rangle|, \Sigma) = \ln |\Sigma| - \ln(2L)d + \sum^{d-1}_{i=0} \Lambda (2L-2i)
%\end{align*}  
%
%The Examiner also expresses his concern about the quality of the ``visual match'' on page 66.
%For this, I had updated the section to include a quantitative, and hence objective, measure of match instead of the current subjective evaluation.
%I believe, such a change should completely addresses his concerns.
%I also made other small updates, to addresses his other various minor concerns.

\subsubsection*{7. Chapter 5}

\replyToComment
    {Page 94, section 5.2.2.2 MSE is first used here. What is MSE? Is it Mean Squared Error? If yes, what is the reference value for calculating the MSE?}
    {MSE does stand for Mean Squared Error, and the page is updated accordingly.
The reference for computation in the section is the ``true signal'' $\sigma$.
}

\replyToComment
    {Fig 5.11  shows two curves that are almost, if not exactly the same. ...
What is the difference between them?}
    {Fig. 5.11 shows two curves that are essentially the same.
One of them is ``simulated result'' and the other is ``analysis formula''.
The difference is that the former is computed through a Monte-Carlo simulation and the other is a simple plotting of the mathematical calculated values.
The purpose is to show that the heuristic formula given as

\begin{center}
  $MSE = \frac{1}{(ENL-0.5)\ln^2(2)}$
\end{center}  

closely tracks observable values.
The argument is presented in a much more detailed manner in Section 5.3.2.1.
}

Page 94, section 5.2.2.2 is updated to indicate MSE as ``Mean Squared Error''.
The reference for computation in the section is the ``true signal'' $\sigma$.

Fig. 5.11 shows two curves that are essentially the same.
One of them is ``simulated result'' and the other is ``analysis formula''.
The difference is that the former is computed through a Monte-Carlo simulation and the other is a simple plotting of the mathematical calculated values.
The purpose is to show that the heuristic formula given as

\begin{align*}
  MSE = \frac{1}{(ENL-0.5)\ln^2(2)}
\end{align*}  

closely tracks observable values.
Readers should also note that this argument is presented in a much more detailed manner in Section 5.3.2.1.

Examiner No 3 also expresses his curiosity to see ``how the f-MLE filter will compare with other state-of-the-art-filters''.
While, I am also eager to see, and to some extend to prove, the good results of my proposed f-MLE filter,
  I have decided not to include it in the section 5.3, where the performance of many different filters are reviewed.
There are a few reasons for such a decision.
First, in the section I am proposing a new way to evaluate speckle filters.
And instead of at the same time discussing the performance of f-MLE filter, I wish to focus on that topic exclusively.
Second, assuming the performance of f-MLE filter is included and found to be superior than others,
  such a result can be easily refuted.
That is because I was proposing both new speckle filters and a new approach to evaluate speckle filters,
  which I wish to, as much as possible, keeping them independently of each other.
Last but not least, as noted by all Examiners and myself included, speckle filtering is not the main topic of the thesis.
It is only the avenue to demonstrate the benefits of the proposed models for SAR \& POLSAR data of the thesis.

\subsubsection*{8. Chapter 6}

In this chapter, the Examiner wonder: ``if the proposed models are far from complete, how can it be an accurate representation of the data''.
To clarify, the proposed models in this thesis do not aim to become accurate representation, or fully representative, of the data.
Rather, they are proposed as highly representative, probably the most representative scalar models, for the multi-dimensional data.
They are, however, useful as scalar models are needed quite often, when scalar decision are to be taken,
  for example: what is the best speckle filter for the given data set?, or what type of surface does the region of interest belong to? ...

The Examiner's next question center around this very point: ``.. definitely not full representative of the data''.
He went on to ask: ``What kind of eror will it cause?''.
To this the answer is that the model inherently suffers from loss of dimension
  where the full data is multi-dimensional and the proposed models are scalar.

Besides these main points, the chapter is updated to increase its pursuasive power.
For example: the sentence in line 1, page 133, section 6.3 is removed.
Or a list of achievements has been included in this chapter.

\subsubsection*{9. Appendix}

While the conference papers have been published and the journal paper submitted,
  their copies have been removed from this section.

\bibliographystyle{apalike}
\bibliography{answer_document}
