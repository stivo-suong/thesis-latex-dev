\section*{Reply to Examiner No. 3}

\begin{replyheader}
\end{replyheader}  

I would like to thank Examiner No. 3 for the constructive comments and suggestions. Since these were presented in the order of the thesis' chapters, I will follow this sequence in my responses below:

\subsubsection*{1. Overall Organization and presentation}

\replyToComment
    {The presentation needs to be considerable improved.}
    {The presentation has been improved according to the Examiners' suggestions. I greatly appreciate the Examiners' help in making this thesis its improved version!}

\replyToComment
    {Chapter 3 and 4 talks about models, but these models were not organized into sections with appropriate headings ... (see my comments in the respective chapter below).}
    {Since the organization for the Examiner's comments as well as for my answer are based on chapters, please see my reply in the corresponding chapters below.}

\replyToComment
    {Page ii of the Contents section is missing}
    {Page numbers have been added for the Contents section}

\replyToComment
    {Usually, a Glossary is put at the end of the document}
    {Updated (now the Glossary section stays at the end of the thesis)}

\replyToComment
    {Why are some page numbers (ii,iii ... ix,xi) missing?}
    {They were missing due to slight different in behavior in Latex systems on Windows and Linux. A fix has been applied to ensure consistent behavior and the page numbers should be printed out correctly now.}

\replyToComment
    {Page xiii: List of source codes. These are actually pseudocodes and not source codes.}
    {Updated (now called ``List of Pseudocodes'')}

\subsubsection*{2. Abstract}

\replyToComment
    {
      Since these models are the key contributions of the thesis, it would be good to give a separate name to each of these models.
      Also a short description of what each of them is suitable for should also be included in the abstract.
    }
    {The models' names and descriptions have been added into the abstract}

\replyToComment
    {The abstract lists the benefits of these models but not the shortcomings. Why?}
    {
      The shortcomings were discussed in the final chapters of the thesis.
      They have also been included into the abstract now.
    }

\subsubsection*{3. Chapter 1}

\replyToComment
    {Table captions appears at the top of a table, rather than below it}
    {This is typical of IEEE publications, so we adopted it in the thesis. However we tried this stylistic suggestion and it looks good, so it has been adopted for the revied thesis.}

\replyToComment
    {No references throughout this chapter.}
    {References have been included in this chapter, increasing its persuasive power.}

\replyToComment
    {you are trying to propose an additive and homoskedastic model for the data. What makes you think that this model is appropriate?}
    {Before an answer is laid out, I wish to make a few points.
Firstly, I do think that the Examiner has raise a good and important question here.
However %the goodness of
the models are, at the end of the day, judged mainly by their performance in practice. In turn, the performance is found from the outcome of various experimental evaluations and after developing some mathematics. But the experiments and mathematics require quite a lot of theoretical explanation and background discussion before they are presented -- and this is what forms the main bulk of the thesis. 
Because of this necessary sequence, I don't think that the demonstration of the models appropriateness can be tackled in the very first chapter, although the thesis can (and does) reveal early on that the models will (later) be shown to work well.
%**IVM: Hai, I rewrote this quite a lot. Can you please ensure that somewhere at the front of the thesis it says something like "this model will be shown, by experimental evaluation presented later in this thesis, to be an appropriate and useful model"

To directly address the Examiner's concern, let me mention a few points that have been noted earlier in the thesis.
First the generic model proposed in this thesis is applicable to multi-dimensional POLSAR data.
Second, when the multi-dimensional data is collapsed into one dimension, 
  the specific case of the proposed model matches perfectly with the model proposed for SAR by Arsenault \cite{Arsenault_JOptSocAm_1976}. 
Thus -- even before performing a detailed evaluation -- there is good evidence that the proposed generic model is at least as appropriate as the widely-accepted model cited above.
All of this discussion is in the revised thesis.
}

\replyToComment
    {In section 1.4, it may be worthwhile to mention that the author's contribution had been published in various venues, together with a list of author's publications.}
    {The list of publication is now included in this section as well.}

\subsubsection*{4. Chapter 2}

\replyToComment
    {Perhaps the theory section can be moved to a separate chapter, so that more space can be dedicated to the discussion of related work}
    { In the end I could not move the theory section away from a discussion of related work, because so much of the theory comes from the related work (i.e. it is not possible to cleanly separate these topics). 
    But actually I believe the examiners request is really that "more space can be dedicated to the discussion of related work". So instead I heavily updated the chapter to do exactly what the examiner requests. For example: 
\vspace{-3mm}
\begin{enumerate}
  \item Section 2.1 (basic theory about the data and older related work) is shortened to 12 pages and 
  \item Section 2.2 (newer related work and extended theory) is extended significantly to 13 pages. 
\end{enumerate}
I believe that this rewriting leads to a better balance between describing the nature of the data and reviewing the related work.
}

\subsubsection*{5. Chapter 3}

\replyToComment
    {how can the statement on page 44 line 1 ``... intensity are equal to a scaled version of these unit variables, specifically $A=\sigma A_1$ and $I=\sigma^2 I_1$'' be true?}
    {the statement in page 44 line 1 should be interpreted strictly in the context that precedes it. 
Mathematically speaking, $\sigma$ is just a constant, which makes the statement true.
Any resemblance of this to the SAR signal will only established in the next paragraph onwards.
In fact, line 7 page 44 clearly indicates under what circumstances that $\sigma$ can be considered constant (i.e. namely across spatially homogeneous areas).
}

\replyToComment
    {why homoskedasticity is valid, and what are the assumption in this analysis?}
    {Spartial homogeneity is the assumption used in the analysis thus far. 
Its limitation is acknowledged in the very next sentence, as the Examiner No. 3 pointed out: ``Over heterogeneous areas, ... $\sigma$ varies significantly ...''.

This assumption is actually not very restrictive.
It should be noted that at the level of each physical radar resolution cell, the value measured in SAR is not deterministic.
For all practical purposes, the single measured value is considered to be the result of a stochastic process, which has one ``true'' signal, namely $\sigma$.
In that sense, the assumption (of a single stochastic process) is actually applicable to both homogeneous and heterogeneous areas. 
}

\replyToComment
    {If $\sigma$ is not a constant, then $var(L_A)$ and $var(L_I)$ cannot be homoskedastic}
    {If the imaging area is heterogeneous, then $\sigma$ is no longer a constant from one resolution pixel to the next.
That much is clear.
Then, as reviewed in Chapter 2, there are many different ways to model $\sigma$ which in turns leads to many different models for the observable magnitude.
This non-constant $\sigma$ also leads to heteroskedasticity, 
%``$var(L_A)$ and $var(L_I)$ cannot be homoskedastic'', 
  should we consider ``$var(L_A)$ and $var(L_I)$'' as the variation of the observables in an area.
And the Examiner correctly pointed out in his comment. 
However, if we consider ``$var(L_A)$ and $var(L_I)$'' as the deviation of the observables from their corresponding ``true signal'' at each physical resolution cell level, 
  then $\sigma$ is constant at each resolution pixel!
Consequently $var(L_A)$ and $var(L_I)$ will be independent of $\sigma$ (as described in Table 3.3).
In fact they are constant, which leads to homoskedasticity!
}

\replyToComment
    {If there are different models, the author should give a name to each of his models ...}
    {Yes there are several models proposed and the thesis has been updated with names for each of them!}

\subsubsection*{6. Chapter 4}

\replyToComment
    {When you have highly correlated data (i.e. homogeneous areas), the determinant $|\Sigma|$ will be very small, leading to a very narrow PDF in Equation 4.1.
Also the inverse $\Sigma^{-1}$ is ill-defined, leading to large errors in your model.
Please justify the situations, if any, your model will not work well.}
    {Equation 4.1 is the PDF for the circular complex Gaussian distribution, which is widely used in POLSAR.
Its form, as repeated from equation 4.1, is written as:

\begin{center}
  $pdf(s;\Sigma)=\frac{1}{\pi^d |\Sigma|} e^{-s^{*T}\Sigma^{-1}s}$
\end{center}

The equation is also ill-defined where $|\Sigma|=0$, which is also the same time that $\Sigma^{-1}$ is ill-defined.
In other words, the proposed model has the same assumption and validity of the widely-accepted circular complex Gaussian distribution model.
In POLSAR, $|\Sigma|=0$ most commonly happens when the dataset is in Single-Look format.
Actually, this restriction is clearly stated in the sentence that follows Equation 4.1: ``the covariance matrix is only defined on multiple data-points''. %**IVM: Hai, maybe you can change the sentence in the thesis to say something like ``in common with other models based on complex Gaussian distribution, the model will be ill-defined where $|\Sigma|=0$''.

In fact, it was partially for the ``narrow PDF and large errors'' concern that I originally proposed the use of log-transformation.
Since the original domain is multiplicative, the range of small values is, as also observed by Examiner No. 3, commonly found to be extremely limited ($|\Sigma|$ ranges from $0+$ to $1$).
The log transformed domain not only changes the nature of the noise from multiplicative to additive, but also give this ``small'' range $(0,1)$ a much wider space $(-\infty,0)$.
In other words, it helps to expand the ``narrow'' distribution depicted in Equation 4.1 (when $|\Sigma|$ is small) to become another distribution whose shape does not dependent on $|\Sigma|$, as depicted by the Equation below:

\begin{center}
  $pdf( \ln | \langle ss^{*T} \rangle|, \Sigma) = \ln |\Sigma| - \ln(2L)d + \sum^{d-1}_{i=0} \Lambda (2L-2i)$
\end{center}
}

\replyToComment
    {On page 66, $3^{rd}$ paragraph, last sentence, the author declares that ``A visual match is clearly observable ...''.
However, in Fig 4.2(a) ...}
    {I understand that the Examiner is concerned about the subjective quality of the ``visual match'' claim. 
Because of this, the section has now been updated so that every `visual match' also includes a quantitative, and hence objective, measure of similarity.
These objective scores -- which do indicate good matches -- should completely addresses these concerns, I believe.}

\replyToComment
    {
      In fact, at x-axis value, the real data has a value that is a sharp dip from its neighboring values.
      Perhaps, it is important to exaplain why the real data behaves in such a way?
    }
    {
      It is observable that the chart for the real data (AIRSAR case) is not behaving as good in comparison to other dataset (e.g. RADARSAT2 or simulated data).
      This is because the AIRSAR data set is much smaller in size than the other dataset (50x50 for AIRSAR vs. 300x300 for RADARSAT2).
      Naturally, real data sets have natural fluctuation in comparison to perfect theoretical assumptions, (e.g. the area is assumed to be homogeneous while such a fact is not known for sure, in practice).
      In small dataset (i.e. the AIRSAR dataset) these fluctuations are expected to be more pronouced than others.
    }
    
\replyToComment
    {The section heading 4.5.3 should be ``Effective Number-of-Looks'' instead of ``Effect Number-of-Looks''}
    {Updated accordingly!}

\subsubsection*{7. Chapter 5}

\replyToComment
    {Page 94, section 5.2.2.2 MSE is first used here. What is MSE? Is it Mean Squared Error? If yes, what is the reference value for calculating the MSE?}
    {MSE does stand for Mean Squared Error, and the page is updated accordingly.
The reference for computation in the section is the ``true signal'' $\sigma$.
}

\replyToComment
    {Fig 5.11  shows two curves that are almost, if not exactly the same. ...
What is the difference between them?}
    {Fig. 5.11 shows two curves that are essentially the same.
One of them is ``simulated result'' and the other is ``analysis formula''.
The difference is that the former is computed through a Monte-Carlo simulation and the other is a simple plotting of the mathematical calculated values.
The purpose is to show that the heuristic formula given as

\begin{center}
  $MSE = \frac{1}{(ENL-0.5)\ln^2(2)}$
\end{center}  

closely tracks observable values.
The argument is presented in a much more detailed manner in Section 5.3.2.1.
}

\replyToComment
    {Fig 5.12: shows the MSE and speckle suppression power of your f-MLE filters for homogeneous area. How does it compare with the other state-of-the-art speckle filters?}
    {
      It should be noted that the f-MLE filters are iterative filters, where the number of iterations is configurable.
      Thus assuming prior knowledge of a homogeneous area, by increasing the iteration number, the speckle suppression power can be improved.
      But of course, such knowledge is only theoretical and may be difficult to obtain for real-life scenarios.
      Concerning the comparison with other state-of-the-art filters, please see my answer below.
    }

\replyToComment
    {Please label the two curves in Fig. 5.13}
    {The labels have now been added into the figure.}    

\replyToComment
    {
      In Fig 5.15 you show a comparison between f-MLE filter and the box-car filter for heterogeneous patterns ...
      I am still curious about how f-MLE filter will compare with other state-of-the-art filters for heterogeneous area as well.
    }
    {
While, I am also eager to see, and to some extent to prove, the good results of my proposed f-MLE filter in a rigorous manner,
  I have decided not to include this in the section 5.3, where the performance of many different filters are reviewed.
There are a few reasons for such a decision.
First, the focus of this section is to propose a new way to evaluate speckle filters, \textbf{not} to propose any new speckle filter.
Instead of discussing the performance of one particular filter (f-MLE), I wish to focus more exclusively on the topic of \textit{how} to evaluate such filters.
Second, assuming the performance of the f-MLE filter is included and found to be superior than others,
  then such a result would be highly suspect
because I would in effect be proposing both a speckle filter as well as a new approach to evaluate speckle filters.
As much as possible, I would like to keep these issues independently of each other.
Last but not least, as noted by all Examiners and myself (in the thesis abstract, introduction and conclusion), speckle filtering is not the main topic of the thesis.
It is only one avenue to demonstrate the benefits of the proposed models for SAR \& POLSAR data.
    }

\replyToComment
    {
      Look at Fig 5.13 again, ... can I assume that (b) and (d) are the results of the boxcar filter for homogeneous area?
      Can these results be compared to the f-MLE results for homogeneous area in Fig 5.12?
    }
    {
      Yes, Fig 5.13 shows the results of applying the boxcar filter on two homogeneous areas 3dB apart.
      The purpose of Fig 5.13 is to illustrate how AUC (i.e. Area Under the Curve) and MSE can be used to evaluate the performance of speckle filters on heterogeneous area.
      Thus what it shows is not to be taken for comparison with what is shown in Fig. 5.12.
      Fig. 5.15 shows a comparison between the result between boxcar and f-MLE filters over various underlying patterns.
    }    

%\begin{mylist}{Something else}
%\item[Whatever]  \LongText 
%\item[Something else] \LongText \LongText
%\end{mylist}


    
\subsubsection*{8. Chapter 6}

\replyToComment
    {If the proposed models are ``far from complete'', how can it be an accurate representation of the data.}
    {The proposed models in this thesis do not aim to become accurate representations of all data, or to be fully representative of all data.
Rather, they are proposed as being highly representative (and possibly the \textit{most representative} scalar models) for the multi-dimensional POLSAR data.
Despite not being perfect, the models are very useful, since scalar models are often needed when scalar decisions are required,
  for example, in answering the question: what is the best speckle filter for the given data set?, or what type of surface does the region of interest belong to? ...
}

\replyToComment
    {
      Page 133, line 3: ``it is definitely not fully representative of the data.''.
      So what kind of error it cause?
      How will it affects the use of your model?
      %It is necessary to point out what are the shortcomings and how it affects the use of your model!
    }
    {
      The thesis very clearly states %**IVM: Hai, can you remember where? Good to put a referene here
       that the proposed models inherently suffer from loss of dimension where the full data is multi-dimensional and the proposed models are scalar.
      This restricts the use of the model to a class of problems where a single scalar decision, probability, distance or number is required to represent the complex dataset.
    }

\replyToComment
    {
      Page 133, line 1: ``... its potential still mostly stays undiscovered''.
      So what have you done with your thesis?
      Do 10\% of the work and leave the other 90\% for other people to do it for you?
      A thesis is supposed to persuade others to believe what you have proposed and to use it.
      How can you expect others to believe in you, when you don't know what most of what it is supposed to do?
    }
    {
     To persuade people into believing its proposed models, this thesis tries to maintain and follow a very rigorous scientific methodology. 
     This includes theoretical mathematical transformations and realistic hypothesis-testing experiments.
     The theory, mathematics, objective analysis and subjective evaluations all agree, and demonstrate the validity of the models.
      The thesis also includes a chapter detailing several different applications for the proposed models, but with the disclaimer that the examiner has taken exception to.
      
      To explain this, firstly it is completely impossible for one thesis to include a discussion on \textit{all possible uses} of something as general as a model such as this. Therefore, the thesis concentrates on what are likely to be the most important uses. These also happen to coincide with the applications that the author was working on when he originally derived the models.
      Secondly, if other researchers find these models to be useful, they will apply them in many different situations. Maybe more than we can imagine at present. Therefore it is only being honest to acknowledge this fact in the thesis.

Actually, the two questions of (1) what else can the proposed models be used for and (2) should the proposed models be taken as true, are not really related.
%      In the end, Einstein probably did not foresee that his theories were used to create atomic bombs and killed thousands of people.
%This lack of foresight however does not cast any doubt on his contribution to our scientific progress.
%**IVM: Hai, best not to bring up Einstein here...

In summary, the thesis contains a thorough investigation of the models, and presents results that are both consistent, positive and indicative. This should be highly persuasive.
At the same time, the thesis makes an effort to evaluate the application of the models in the most important usage domains that the author has identified, as well as to address the use of the models beyond that.
    }

\replyToComment
    {
      On page 10, Chapter 1, the author specified a list of ``results to be obtained''.
      However, in the conclusion chapter, there is no corresponding list of achievements ...
    }
    {
      A corresponding list of achievements has been included in the conclusion, showing that all objectives were met.
    }
    
\subsubsection*{9. Appendix}

\replyToComment
    {It is not usual to attach copies of academic papers in the Appendix.}
    {They were included for the convenience of the examiners and other readers. They have now been removed from the Appendix.}

\bibliographystyle{apalike}
\bibliography{answer_document}
