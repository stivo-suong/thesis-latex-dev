\documentclass{article}

\usepackage{titlesec}
\newcommand{\sectionbreak}{\clearpage} %to give each section a new page

\usepackage{amsmath} %to have equation with no number

\title{Reply To Examiners}
\date{\today}
\author{Thanh-Hai Le}

\begin{document}
\maketitle

\newenvironment{replyheader}{
	\vspace*{10mm}
\begin{tabular}{l l}
\bf{Name of Student:} & Le Thanh Hai \\
\bf{Degree:} & Doctor of Philosophy \\
\bf{Thesis Title:} & Scalar \& Homoskedastic Models \\ 
 & for SAR \& POLSAR data
\end{tabular}  
	\vspace*{10mm}
}

\section*{Reply to Examiner No. 1}

\begin{replyheader}
\end{replyheader}  

First and foremost, I would like to express my appreciation towards Examiner No. 1 for his positive overall comments.
I also appreciate the fact that examiner No 1 includes 
  not only his comments in the examiner's report 
  but also his corrective suggestions inside the original thesis.
There was no direct question in the report, 
  and followings are my replies to his comments in the returned thesis.


\section*{Reply to Examiner No. 2}

\begin{replyheader}
\end{replyheader}  

Examiner No. 2 suggests to review Section 1.1 and Section 1.2 so that the significance of the research motivation is highlighted.
Similarly the conclusion chapter should also include the research objectives.
These have been updated in the revised thesis,
  even though for the latter point I would wish to point out that the research objectives have been clearly spelled out in the Introduction chapter.
The examiner also suggested to change the word ``theory'' to ``model'', which has also been applied through out the thesis.

The second main comment is to further highlight the novelty claim, which can be splitted into two parts.
In the first part, the examiner suggests that ``the proposed model is derived based on the existing statistical models for SAR and extended to POLSAR''.
Our main approach actually differs slightly and in a very subtle way.
The proposed model is derived based on a generic result for multi-dimensional random-walk in mathematics (cite Goodman), which is applicable to POLSAR.
Subsequently we pointed out that our model is actually an extension of existing statistical models for SAR,
  since our multi-dimensional model, when collapsed into single-dimension, matches perfectly with existing SAR models.
In the second part, the exminer express his doubt in ``the use of MSE (mean squared error) as the single unified evaluation criteria'',
  as ``the proposed log-transform model will introduce an inevitable bias error''.
While I have total empathy with the examiner, since it is ultimately tricky to evaluate evaluation criteria.
Still, I wish to point out that MSE evaluation has two components: 1. bias evaluation and 2. variance evaluation,
  which for SAR speckle filter equates to 1. radiometric preservation and 2. noise suppression capabilities.

( \textbf{ Hai note: }  the examiner suggests further clarifications to support the claim of novelty in Part II of his report.
Unfortunately I do not have it at hand.
It would be great if I can have a look at it!)

The examiner also suggestion inclusion of several references into the literature review topics of ``relevant SAR/POLSAR speckle filters'' and  ``evaluation of SAR speckle filters''.
I really appreciate the examiner effort in pointing out specific references.
They, and some other articles, have been included in the thesis.
I would like however to seek his empathy towards a couple of points.
First, a number of the references cited has already been included into the original thesis (e.g Touzi (TODO:CITE) or 2-11-12.). 
Some of them (for example Argenti (TODO:CITE)) are actually published (Sept 2013) after the submission of the original thesis (Aug 2013).
Second, please also note that the topics of SAR/POLSAR speckle filters and to some extend, their evaluation, may be considered as part of the thesis's contribution.
While they do results in published scientific articles, 
  they are not, however, to be considered as the main contributions of the thesis.
Rather they serves as application, or demonstration, of the benefits of the thesis's main proposal.
As such, while I understand the need for comprehensive review, due to various constraint, I seek his understanding that they are probably not to be explored with the same depth and coverage as the main topic of the thesis.

I appreciate the fact that the examiner also shares our thoughts and put it in a very concise sentence: ``the derivation of the statistical model for POLSAR data is the main contribution of the thesis''.
As suggested, we have tried to extract portions of this thesis as scientific articles and submitted them for review.

Last but not least, I have gone through the various errata exposed by Examiner No.2 and have updated the thesis accordingly.
Finally, I wish to express my appreciation towards Examiner No.2 for his carefully and detailed review.
  
\section*{Reply to Examiner No. 3}

\begin{replyheader}
\end{replyheader}  

Since Examiner No. 3 presneted his comments and suggestions in the order of the thesis' chapters, 
  I would follow his suite.

\subsubsection*{1. Overall Organization and presentation}

I appreciate the fact that the examiner noted: ``The overall organization of the thesis is appropriate''.
Besides the issue of appropriate headings for chapter 3 and 4, which will be discussed in their respective section, minor editions have been included to address his other concerns.

\begin{itemize}
  \item \textbf{Page ii of Contents page is missing:} This is probably an administrative mistake and oversight from us. Please accept our appologies!
  \item \textbf{Usually, a Glossary is put at the end of the document: } Updated (now the Glossary section stays at the end of the thesis)
  \item \textbf{Some page number missing: } Updated (TODO: mostlikely this is an option that has been accidently turned on in the Latex Class file.)
  \item \textbf{List of source codes: } Updated (now called ``List of Pseudocodes'' as suggested by the Examiner.)
\end{itemize}

\subsubsection*{2. Abstract}

The abstract has been updated so that: 1. each of the proposed models have their own name and 2. not only the benefits are listed, but also their shortcomings.

\subsubsection*{3. Chapter 1}

The examiner suggests that ``Table captions appears at the top of a table, rather than below it'' and the chapter as well as the thesis has been updated accordingly.
The chapter is also updated, so that new and existing references are added to provide direct support.
The author's publication list is now also included in this, as well as in the final conclusion, chapter.

At this juncture, I would also wish to address one of Examiner No. 3's main concern.
It is termed as: ``it doesn't mean that by fitting a homoskedastic model to heteroskedastic data, these algorithm will give correct results''.
The approach described in this thesis differs slightly from ``fitting a homoskedastic model to heteroskedastic data''.
Here, I proposed to used log-transformation which has long been noted as converting the data from mutiplicative to additive domain.
The model for the data in this additive domain has been derived and accepted in (TODO:CITE Arsenault).
My work, in simple terms, suggests that in addition to additive properties, the transformed domain also exhibits homoskedastic properties as contrast to the heteroskedastic properties in the original domain.
These properties are investigated in this thesis to show that it does offers several benefits (for example: in designing and evaluating speckle filters).
It should be noted that the model for SAR used in this thesis is exactly the same as the additive model proposed by Arsenault (TODO:CITE) and is mathematically equivalent to the model multiplicative model for SAR data in its original domain.

\subsubsection*{4. Chapter 2}

Chapter 2 is also updated to address Examiner No. 3's concerns.
In a bid of not confusing other examiners, I have decided not to move the theory section into a separate chapter.
However, this chapter is heavily updated so that 1. Section 2.1 is shortened considerably and 2. Section 2.2 is extended significantly. 
This I believe leads to a better balance between describing the nature of the data, and reviewing the related work.
This also help growing the space dedicated to the discussion of ``Other approaches that have recently been pursued in the research community (SAR speckle filters)''.

\subsubsection*{5. Chapter 3}

There are apparently two main concerns from Examiner No. 3's comments.
First, page 44 line 1 should be intepreted strictly in the context of the preceeding paragraph (page 43). 
That is mathematically  speaking $\sigma$ is just a mathematical variables, thus making the statement true. 
Any resemblence of it with the SAR signal is established in the next paragraph.
The third line of the second paragraph in page 44 clearly indicated where $\sigma$ can be considered as constant (i.e. spatial homogeneous area or at each resolution cell).
It is also acknowledged in the thesis, as the Examiner No. 3 pointed out precisely: ``Over heterogeneous area, ... $\sigma$ varies significantly ...''
This is the assumption used in this analysis, 
  and a portion of subsequent work is devoted to extend this early analysis into heterogeneous area.
The very next section (i.e. Section 3.2) starts the discussion on this very topic.  

In that very same vein, table 3.3 should be viewed under the assumption that $\sigma$ is constant (either in homogeneous area or in a single resolution cell).
Under this condition, then the variances are independent of $\sigma$, i.e. it is homoskedastic.
In fact, if the area is heterogeneous as reviewed in Chapter 2, there are many different ways to model $\sigma$ which in turns leads to many different models for the observable magnitude.
This, of course, leads to different expression for the observable variances.
In fact, it leads to heteroskedasticity, which the Examiner correctly pointed out ``$var(L_A)$ and $var(L_I)$ cannot be homoskedastic'', should we consider $var(L_I)$ as the variation of the observables in an area.
It should be noted, however, that at the level of each radar physical resolution cell, the values measured in SAR are not deterministic.
Then if we consider $var(L_A)$ as the deviation of the observables from its true signal, we will also have $var(L_A)$ as being constant (assuming that the effective number of looks or ENL is constant across the area)!

The second point concerns the models and their names.
This chapter has been updated and model names are given for each to clarify the concepts.
The confusion is understandable, and partially my fault, in not highlighting the following points.
First there are different models being proposed, in Section 3.4.
And second, they are all however derived based on ONE basic model: the one based on ``the base-2 log-transformation of the SAR data.''
Finally, there are differences in each of the models, even though, whether these differences are significant or not is, to be honest, rather subjective at my current state of understanding.

\subsubsection*{6. Chapter 4}

Examiner No.3 expresses his doubt on two points.
First, that ``the determinant $|\Sigma|$ will be very small, leading to a very narrow PDF in Equation 4.1''.
And second that $\Sigma^{-1}$ is ill-defined.
I wish to highlight that Equation 4.1 is the PDF for the circular complex Gaussian distribution, which is widely used in POLSAR.
Its form, as repeated from equation 4.1, is written as:

\begin{align*}
  pdf(s;\Sigma)=\frac{1}{\pi^d |\Sigma|} e^{-s^{*T}\Sigma^{-1}s}
\end{align*}  

This equation is also ill-defined where $|\Sigma|=0$, which is also the only time that $\Sigma^{-1}$ is ``ill-defined''.
In other words, the models proposed has the same assumption and validity with the widely-accepted circular complex Gaussian distribution model.
In POLSAR, $|\Sigma|=0$ most commonly happens when the dataset is in Single-Look format.
This restriction is clearly stated in the sentence that follows Equation 4.1: ``the covariance matrix is only defined on multiple data-points''.

It is partially for the second concern that log-transformation is proposed.
Since the original domain is multiplicative, the range of small values is, as also observed by Examiner No. 3, commonly found but extremely limited ($|\Sigma|$ ranges from $0+$ to $1$).
Log transformed domain not only changes the nature from multiplicative to additive, but also give this ``small'' range $(0,1)$ a much widely space $(-\infty,0)$.
In other words, it helps to expand the ``narrow'' distribution depicted in Equation 4.1 (when $|\Sigma|$ is small) to become another distribution whose shape does not dependent on $|\Sigma|$, as depicted by the Equation below.

\begin{align*}
  pdf( \ln | \langle ss^{*T} \rangle|, \Sigma) = \ln |\Sigma| - \ln(2L)d + \sum^{d-1}_{i=0} \Lambda (2L-2i)
\end{align*}  

The Examiner also expresses his concern about the quality of the ``visual match'' on page 66.
For this, I had updated the section to include a quantitative, and hence objective, measure of match instead of the current subjective evaluation.
I believe, such a change should completely addresses his concerns.
I also made other small updates, to addresses his other various minor concerns.

\subsubsection*{7. Chapter 5}

Page 94, section 5.2.2.2 is updated to indicate MSE as ``Mean Squared Error''.
The reference for computation in the section is the ``true signal'' $\sigma$.

Fig. 5.11 shows two curves that are essentially the same.
One of them is ``simulated result'' and the other is ``analysis formula''.
The difference is that the former is computed through a Monte-Carlo simulation and the other is a simple plotting of the mathematical calculated values.
The purpose is to show that the heuristic formula given as

\begin{align*}
  MSE = \frac{1}{(ENL-0.5)\ln^2(2)}
\end{align*}  

closely tracks observable values.
Readers should also note that this argument is presented in a much more detailed manner in Section 5.3.2.1.

Examiner No 3 also expresses his curiosity to see ``how the f-MLE filter will compare with other state-of-the-art-filters''.
While, I am also eager to see, and to some extend to prove, the good results of my proposed f-MLE filter,
  I have decided not to include it in the section 5.3, where the performance of many different filters are reviewed.
There are a few reasons for such a decision.
First, in the section I am proposing a new way to evaluate speckle filters.
And instead of at the same time discussing the performance of f-MLE filter, I wish to focus on that topic exclusively.
Second, assuming the performance of f-MLE filter is included and found to be superior than others,
  such a result can be easily refuted.
That is because I was proposing both new speckle filters and a new approach to evaluate speckle filters,
  which I wish to, as much as possible, keeping them independently of each other.
Last but not least, as noted by all Examiners and myself included, speckle filtering is not the main topic of the thesis.
It is only the avenue to demonstrate the benefits of the proposed models for SAR \& POLSAR data of the thesis.

\subsubsection*{8. Chapter 6}

\end{document}
