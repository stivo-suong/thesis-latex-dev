\documentclass{article}

\usepackage{titlesec}
\newcommand{\sectionbreak}{\clearpage}

\title{Reply To Examiners}
\date{\today}
\author{Thanh-Hai Le}

\begin{document}
\maketitle

\newenvironment{replyheader}{
	\vspace*{10mm}
\begin{tabular}{l l}
\bf{Name of Student:} & Le Thanh Hai \\
\bf{Degree:} & Doctor of Philosophy \\
\bf{Thesis Title:} & Scalar \& Homoskedastic Models for SAR \& POLSAR data
\end{tabular}  
	\vspace*{10mm}
}

\section*{Reply to Examiner No. 1}

\begin{replyheader}
\end{replyheader}  

Examiner No 1 includes not only his comment but also his original thesis.
There was no direct question in the comment, and followings are my replies to his questions in the returned thesis.


\section*{Reply to Examiner No. 2}

\begin{replyheader}
\end{replyheader}  

Examiner No. 2 suggests to review Section 1.1 and Section 1.2 so that the significance of the research motivation is highlighted.
Similarly the conclusion chapter should also include the research objectives.
These have been updated in the revised thesis,
  even though for the latter point I would wish to point out that the research objectives have been clearly spelled out in the Introduction chapter.
The examiner also suggested to change the word ``theory'' to ``model'', which has also been applied through out the thesis.

The second main comment is to further highlight the novelty claim, which can be splitted into two parts.
In the first part, the examiner suggests that ``the proposed model is derived based on the existing statistical models for SAR and extended to POLSAR''.
Our main approach actually differs slightly and in a very subtle way.
The proposed model is derived based on a generic result for multi-dimensional random-walk in mathematics (cite Goodman), which is applicable to POLSAR.
Subsequently we pointed out that our model is actually an extension of existing statistical models for SAR,
  since our multi-dimensional model, when collapsed into single-dimension, matches perfectly with existing SAR models.
In the second part, the exminer express his doubt in ``the use of MSE (mean squared error) as the single unified evaluation criteria'',
  as ``the proposed log-transform model will introduce an inevitable bias error''.
While I have total empathy with the examiner, since it is ultimately tricky to evaluate evaluation criteria.
Still, I wish to point out that MSE evaluation has two components: 1. bias evaluation and 2. variance evaluation,
  which for SAR speckle filter equates to 1. radiometric preservation and 2. noise suppression capabilities.

( \textbf{ Hai note: }  the examiner suggests further clarifications to support the claim of novelty in Part II of his report.
Unfortunately I do not have it at hand.
It would be great if I can have a look at it!)
  
\section*{Reply to Examiner No. 3}

\begin{replyheader}
\end{replyheader}  

\end{document}
