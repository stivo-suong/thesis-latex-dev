\section*{Reply to Examiner No. 2}

\begin{replyheader}
\end{replyheader}  

\replyToComment
    {Sec 1.1 and 1.2 should be reviewed again ... with proper citations on the background studies.}
    {The sections were reviewed and several citations were added}

\replyToComment
    {Paragraph 1 of Sec 1.3.3 should be removed or rewritten}
    {The paragraph were removed?}
    
\replyToComment
    {The achievement of objectives should (also) be stated in the conclusion chapter. }
    {These were added into the conclusion chapter (``with proper justfications based on experimental results and findings'').}

\replyToComment
    {Replace the word ``theory'' with ``model'' throughout the thesis}
    {Search and replace operation done!}
    
\replyToComment
    {The proposed model is derived based on the existing statistical models for SAR and extended to POLSAR}
    {The main approach, as described in the thesis, differs in a subtle but important way.
The proposed model is derived based on generic mathematical results for multi-dimensional random-walk \cite{Goodman_JOptSocAm_76, Goodman_Springer_1975}.
That is: the proposed models by nature are applicable to multi-dimensional data and thus they are applicable to POLSAR data.
To show that the proposed models are also applicable to SAR data, the dimension parameters in these models are collapsed into 1.
The thesis subsequently shows that they match perfectly with existing SAR models.
}

\replyToComment
    {The proposed log-transform model will introduce an inevitable bias error, which may not be able to measure by MSE.}
    {
      The thesis clearly recognizes that log-transformation will introduce bias.
      It also shows that MSE evaluation does inherently include bias evaluation.
      In fact, the thesis argues that MSE evaluation has two components: 1. bias evaluation and 2. variance evaluation, which for evaluating the performance of SAR speckle filters translates into 1. radiometric preservation and 2. noise suppression respectively.
}

\replyToComment
    {A more comprehensive review of the relevant SAR/POLSAR speckle filters should be included.}
    {The most recent review \cite{Argenti_GRSM_2013} has been included together with several other publications \cite{Lee_RSReviews_1994, Cetin_ProcSPIE_2000, White_ProcSPIE_1994, Sattar_TIP_1997, Wang_TIP_2004, Nielsen_2012_ICASSP}}

\replyToComment
    {Whereever possible, samples of SAR/POLSAR images with these three types of features (i.e. homogeneous, textured and strong scatterer) should be used to evaluate the effectiveness of the proposed models.}
    {In the section to evaluate speckle filters, various different patterns had been studied. Also the MSE evaluation which includes bias and variance evaluation are shown to be able to evaluate the effectiveness of different speckle filters in terms of radiometric preservation and speckle noise reduction respectively.}
    
\replyToComment
    {The work done has great potential to be published in high impact factor journals such as JSTAR, IJRS and PIER}
    {I appreciate the Examiner's vote of confidence. Portions of the thesis has been extracted out as academic papers and submitted for peer-review.}    

\replyToComment
    {Define terms heteroskedastic and homoskedastic in the context of SAR imagery.}
    {Heteroskedastic and homoskedastic are terms defined in statistical context to respectively denote the fact that different sub-populations of a given samples can have different or similar variabilities. In the context of SAR (and POLSAR) imagery they denote the fact that different areas in a given image can have different or the same variances.}

\replyToComment
    {Add references to the following statement: ``for example speckle filtering, target detection, image segmentation and other cluster, classification techniques''}
    {TODO:CITE}

\replyToComment
    {Add references to the following statement: ``most of these data processing techniques are traditionally designed for additive and homoskedastic data.''}
    {TODO:CITE}

\replyToComment
    {Add references to the following statement: ``Such use, however, is known to be not very robust for these so-called heavy detailed distributions.''}
    {TODO:CITE}
    
\replyToComment
    {Add references to the following statement: ``it is known that such use should be avoided in preference to a ratio-based descrimination measure.''}
    {TODO:CITE}

\replyToComment
    {Add references to the following statement: ``The Ordinary Least Square (OLS) is widely used as the best evaluation criteria, which is probably due to the Gauss Markov theorem.''}
    {TODO:CITE}

\replyToComment
    {Add references to the following statement: ``violates the homoskedastic assumption of the theorem and thus many different ways to evaluate SAR speckle filters were proposed.''}
    {TODO:CITE}

\replyToComment
    {It would be easier to read and refer if all the Equations are labelled accordingly.}
    {All equations in the thesis has been updated with labels.}


\textbf{Errata Sheet}
    
\rowcolors{2}{gray!25}{white}    
\noindent
\begin{longtable}[c]{p{0.1\textwidth}|p{0.1\textwidth}|p{0.3\textwidth}|p{0.3\textwidth}}
\textbf{Page} & \textbf{Line} & \textbf{Gramatical Errors} & \textbf{Corrections} \\
 \hline
 \endhead
ToC & - & missing page numbers & Rectified \\
1 & 13 & single SAR channel & single-channel SAR \\ 
3 & 12 & Similarly speaking, & Similarly, \\
3 & 19 & criteria & criterion \\ 
4 & 14 & MMSE criteria & MMSE criterion \\
6 & 9 & Last but certainly not least, & Thirdly, \\ 
6 & 9 & such a framework allow & such a framework allows \\
9 & 7 & multidimensional & Multi-dimensional \\
10 & 12 & The model os & The model is \\
11 & 28 & chapter 5 & Chapter 5 \\
12 & 6 & chapter 6 & Chapter 6 \\
22 & 15 & RadarSat & RadarSat-2 \\
24 & 14 & SVM & SVM (Support Vector Machine) \\
35 & 6-7 & Rician distribution [48]... & Rician distribution [48]. \\
35 & 11 & back scattering & backscattering \\
36 & 10 & literatured & literatures \\
37 & -2 & Proposed bu & proposed by \\
39 & 4 & dependence & dependency \\
44 & -1 & The nature of SAR is ... heteroskedastic heterogeneously & the sentence is rephrased \\
46 & 4 & most known common & most commonly known \\
56 & 6 & POLSAR And & POLSAR. And \\
65 &5,7 & an homogeneous area & a homogeneous area \\
77 & 3 & the objective then is & the objective is \\
77 & 5,6 & it has already been proven & it has already been proven [?] \\
87 & -1 & Fig 5.6a & Fig 5.6. \\
89 & - & Figure 5.6 - subtitle (a) & (a) is removed \\
91 & - & Figure 5.8: Legends are too small to read & They are enlarged \\
95 & - & Figure 5.12 & Use the same scale for  y axis (i.e MSE 0:1) across all the plots \\
103 & 22 & related to the ENL index. over & related to the ENL index over \\
104 & 13 & Equation 5.3.2.1 & The number is corrected to (?) \\
109 & 8 & Figs. 5.21c and 5.21d allows & Fig 5.21(a) and Fig 5.21(d) allow \\
112 & 13 & these smaller requirements & these requirements \\
132 & 9 & evaluation criteria & evaluation criterion
\end{longtable}    
\rowcolors{1}{white}{}

\textbf{Overall Comments \& Replies}

In general, I wish to express my appreciation towards Examiner No. 2 for his careful, detailed and concise review.
I am amazed by the fact that the Examiner is able to precisely summarize the long thesis into a single sentence.
I also feel encouraged by the Examiner's suggestion to extract out portions of this thesis as scientific papers for peer review, which has also been taken up and acted upon.

%=====
%
%Examiner No. 2 suggests to further emphasize the significance of the research motivation in \textbf{Section 1.1 and Section 1.2}.
%In a similar fashion, the conclusion chapter should also include the research objectives.
%These have been updated in the revised thesis.
%Even though I would wish to note that the research objectives have been spelled out in the Introduction chapter.
%The examiner also suggested to change the word ``theory'' to ``model''.
%That has also been applied through out the thesis.
%
%\textbf{The second main comment} is to further highlight the novelty claim, which can be splitted into two parts.
%
%\textit{In the first part}, the examiner appears to discount the novelty claim by suggesting that:
% ``the proposed model is derived based on the existing statistical models for SAR and extended to POLSAR''.
%The main approach, in actuality, differs in a subtle but important way:
%The proposed model is derived based on generic mathematical results for multi-dimensional random-walk \cite{Goodman_JOptSocAm_76, Goodman_Springer_1975}.
%These results are well known to be applicable to POLSAR.
%Thus, the derived models by nature are multi-dimensional model.
%However, when collapsed into single-dimension, they match perfectly with existing SAR models.
%As such, the proposed models can be considered as an extension of existing statistical models for SAR.
%
%\textit{In the second part}, the Examiner expresses his doubt in ``the use of MSE (mean squared error) as the single unified evaluation criteria''.
%His main concern: ``the proposed log-transform model will introduce an inevitable bias error''.
%It is no doubt tricky to evaluate evaluation criteria.
%And I can empathize with his feelings.
%Still, I wish to point out that MSE evaluation inherently includes bias evaluation. 
%In fact it has two components: 1. bias evaluation and 2. variance evaluation, which for SAR speckle filter evaluation translates into 1. radiometric preservation and 2. noise suppression respectively.

%( \textbf{ Hai note: }  The Examiner mentions further clarifications to support the claim of novelty in Part II of his report.
%Unfortunately I do not have it at hand.
%It would be great if I can have a look at it!)

%The examiner also suggestion inclusion of several references into \textbf{the literature review} topics of ``relevant SAR/POLSAR speckle filters'' and  ``evaluation of SAR speckle filters''.
%I really appreciate the examiner effort in pointing out specific references.
%They, and some other articles, have been included in the thesis.
%I would like however to seek his empathy towards a couple of points.
%First, a number of the references cited has already been included into the original thesis (e.g \cite{Touzi_2002_TGRS} or \cite{Xie_2002_TGRS}.). 
%Some of them (for example \cite{Argenti_GRSM_2013}) are actually published (Sept 2013) after the submission of the original thesis (Aug 2013).
%Second, please also note that the topics of SAR/POLSAR speckle filters and to some extend, their evaluation, may be considered as part of the thesis's contribution.
%While they do results in published scientific articles, 
%  they are not, however, to be considered as the main contributions of the thesis.
%Rather they serves as application, or demonstration, of the benefits of the thesis's main proposal.
%As such, while I understand the need for comprehensive review, due to various constraint, I seek his understanding that they are probably not to be explored with the same depth and coverage as the main topic of the thesis.

%\textbf{To conclude}, I appreciate the fact that the Examiner is able to a summarize the long thesis into a single sentence.
%Also, portions of this thesis has been extracted out as scientific articles and submitted for peer review.
%Various errata exposed by Examiner No. 2 and have also been updated in the thesis.
%Last but not least, I wish to express my appreciation towards Examiner No.2 for his careful and detailed review.

%\bibliographystyle{plainnat}
\bibliographystyle{apalike}
\bibliography{answer_document}

