\section*{Reply to Examiner No. 2}

\begin{replyheader}
\end{replyheader}  

Examiner No. 2 suggests to further emphasize the significance of the research motivation in \textbf{Section 1.1 and Section 1.2}.
In a similar fashion, the conclusion chapter should also include the research objectives.
These have been updated in the revised thesis.
Even though I would wish to note that the research objectives have been spelled out in the Introduction chapter.
The examiner also suggested to change the word ``theory'' to ``model''.
That has also been applied through out the thesis.

\textbf{The second main comment} is to further highlight the novelty claim, which can be splitted into two parts.

\textit{In the first part}, the examiner appears to discount the novelty claim by suggesting that:
 ``the proposed model is derived based on the existing statistical models for SAR and extended to POLSAR''.
The main approach, in actuality, differs in a subtle but important way:
The proposed model is derived based on generic mathematical results for multi-dimensional random-walk \cite{Goodman_JOptSocAm_76, Goodman_Springer_1975}.
These results are well known to be applicable to POLSAR.
Thus, the derived models by nature are multi-dimensional model.
However, when collapsed into single-dimension, they match perfectly with existing SAR models.
As such, the proposed models can be considered as an extension of existing statistical models for SAR.

\textit{In the second part}, the Examiner expresses his doubt in ``the use of MSE (mean squared error) as the single unified evaluation criteria''.
His main concern: ``the proposed log-transform model will introduce an inevitable bias error''.
It is no doubt tricky to evaluate evaluation criteria.
And I can empathize with his feelings.
Still, I wish to point out that MSE evaluation inherently includes bias evaluation. 
In fact it has two components: 1. bias evaluation and 2. variance evaluation, which for SAR speckle filter evaluation translates into 1. radiometric preservation and 2. noise suppression respectively.

( \textbf{ Hai note: }  The Examiner mentions further clarifications to support the claim of novelty in Part II of his report.
Unfortunately I do not have it at hand.
It would be great if I can have a look at it!)

The examiner also suggestion inclusion of several references into \textbf{the literature review} topics of ``relevant SAR/POLSAR speckle filters'' and  ``evaluation of SAR speckle filters''.
I really appreciate the examiner effort in pointing out specific references.
They, and some other articles, have been included in the thesis.
I would like however to seek his empathy towards a couple of points.
First, a number of the references cited has already been included into the original thesis (e.g \cite{Touzi_2002_TGRS} or \cite{Xie_2002_TGRS}.). 
Some of them (for example \cite{Argenti_GRSM_2013}) are actually published (Sept 2013) after the submission of the original thesis (Aug 2013).
Second, please also note that the topics of SAR/POLSAR speckle filters and to some extend, their evaluation, may be considered as part of the thesis's contribution.
While they do results in published scientific articles, 
  they are not, however, to be considered as the main contributions of the thesis.
Rather they serves as application, or demonstration, of the benefits of the thesis's main proposal.
As such, while I understand the need for comprehensive review, due to various constraint, I seek his understanding that they are probably not to be explored with the same depth and coverage as the main topic of the thesis.

\textbf{To conclude}, I appreciate the fact that the Examiner is able to a summarize the long thesis into a single sentence.
Also, portions of this thesis has been extracted out as scientific articles and submitted for peer review.
Various errata exposed by Examiner No. 2 and have also been updated in the thesis.
Last but not least, I wish to express my appreciation towards Examiner No.2 for his careful and detailed review.

%\bibliographystyle{plainnat}
\bibliographystyle{apalike}
\bibliography{answer_document}

