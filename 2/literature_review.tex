\chapter{Current Methods in Processing SAR and POLSAR Data} %chapter 2
\label{chap:lit_survey}

This chapter 
reviews the published literature for SAR and POLSAR data processing.
Specifically, its first section describes the multi-dimensional and stochastic nature of the (POL)SAR data.
The second section then illustrates the different approaches that have been published to address the grand challenge of modelling, processing and ultimately understanding these types of data.

\section{The Stochastic and Multivariate Nature of (POL)SAR data}

\subsection{The Stochastic Nature of SAR data}

%\subsubsection{ Traditional Single-Channel SAR and the Speckle phenomena} 

The stochastic nature of SAR data arises due to the process of SAR image formation. Specifically, since one SAR resolution cell is very big in comparison to the radar wavelength, it normally consists of many elementary scatterers. The data point of each SAR pixel then arises from the combined interference effect of the back-scattered waves from these different scatterers. This effect causes natural variation in SAR intensity image even when the underlying scene is highly homogeneous.

SAR speckle phenomena are explained as the interference of many coherent but dephased back-scattering components;
  each reflecting from different and distributed elementary scatterers \cite{Oliver_ProcIEEE_1963, Leith_ProcIEEE_1971}. 
The random nature of the SAR data arises
  due to the unknown and varying location, height and thus distance of each elementary scatterer;
  which, in turn, generates random phases in their back-scattering responses.
%Assuming the number of these elementary back scatterers is sufficiently large,
Assuming these elementary back-scatterers are large in number and independent in nature,  
  then the Central Limit Theorem is applicable \cite{Goodman_Springer_1975}.
Consequently the interference of these randomly phased responses can be considered as
  a random walk on the 2D complex plane \cite{Goodman_JOptSocAm_76}.  
Specifically the real part $A_r$ as well as the imaginary part $A_i$ of the observed SAR signal $A$ can be considered as
  random variables from uncorrelated Gaussian distributed stochastic processes with zero means and indentical variances $\sigma^2/2$  \cite{Lee_CRCPress_2009}. 
Their probability density functions (PDF) are given as:

\begin{equation}
\label{eqn:SAR Real and Imaginary Components PDF}
\caption{eqn:SAR Real and Imaginary Components PDF}
pdf(A_x)=\frac{1}{\sqrt{\pi} \sigma} e^{- \left( \frac{A_x^2}{\sigma^2} \right) }
\end{equation}

It can then be shown that the measurable amplitude $A=\sqrt{A_r^2+A_i^2}$ is a random variable of Rayleigh distribution.
Subsequently the SAR intensity $I=A^2=(A_r^2+A_i^2)$ is a random variable of a negative exponentially distributed random process. The PDFs of these two properties can then be expressed as:

%\begin{eqnarray}
%pdf(A) &=& \frac{2A}{\sigma^2}e^{ \left( -\frac{A^2}{\sigma^2} \right) }\\
%pdf(I) &=& \frac{1}{\sigma^2}e^{\left( -\frac{I}{\sigma^2} \right) }
%\end{eqnarray}
\begin{equation}
\label{eqn:SAR Amplitude PDF}
\caption{eqn:SAR Amplitude PDF}
pdf(A) = \frac{2A}{\sigma^2}e^{ \left( -\frac{A^2}{\sigma^2} \right) }
\end{equation}
\begin{equation}
  \label{eqn:SAR Intensity PDF}
  \caption{eqn:SAR Intensity PDF}
pdf(I) = \frac{1}{\sigma^2}e^{\left( -\frac{I}{\sigma^2} \right) }
\end{equation}

%\subsubsection{The Stochastic nature of SAR  and SAR speckle filtering }

%The stochastic nature SAR arises from the speckle phenomena.
As SAR speckle phenomena arises from the stochastic nature of SAR, it seriously hinders the interpretation and understanding of SAR images.
Speckle noise on SAR images are hence routinely reduced through a speckle filtering process for easier human interpretation and more accurate post-processing.
%***IVM: Hai, I added "human" before 'interpretation'. Also added the text afterwards, please check that the sense of the edited text is still correct in terms of what you wanted to say.

\subsection{The Multivariate Nature of POLSAR data}

\subsubsection{The polarization property of Electro-Magnetic waves}

The propagation of Electro-Magnetic (EM) wave in space is governed by Maxell’s equations. 
The general solution to these equations is a linear superposition of waves, each having the form of:
\begin{eqnarray}
 \vec{E}(\vec{r},t) &=& g(\phi(\vec{r},t)) = g(\omega t - \vec{k} \cdot \vec{r}) \\
 \vec{B}(\vec{r},t) &=& g(\phi(\vec{r},t)) = g(\omega t - \vec{k} \cdot \vec{r})
\end{eqnarray}
where
	$g$ is any well-behaved function of a dimensionless argument $\phi$,
	%for virtually any well-behaved function g of dimensionless argument φ, where
	$\omega$ is the angular frequency (in radians per second), and
	$\vec{k}= (k_x,k_y,k_z)$ is the wave vector (in radians per meter).

Because of the divergence between the electric and the magnetic fields, there will be no fields in the direction of wave propagation.
Thus if we define the z direction as the travelling direction of the EM wave,
  then both the electric and the magnetic fields are fluctuating in the x,y plane.
Furthermore, when considering wave propagation, it is conventional that only the electric field vector is described and the magnetic field is ignored since it is perpendicular to the E field and is proportional to it.
Projecting this vector into x,y directions, we have:

\begin{equation}
\vec{E}(z,t) 
= 
\left(
	\begin{array} {c}
		E_x^0 \cos{ \left( wt - kz  + \phi_x^0 \right) } \\
		E_y^0 \cos{ \left( wt - kz  + \phi_y^0 \right)} \\
		0
	\end{array}
\right)
= 
\left(
	\begin{array} {c}
		\xi_x(t) \\
		\xi_y(t) \\
		0
	\end{array}
\right)
\end{equation}
where $\phi_0$ is the initial phase angle, that is dependent on the choice of space-time origin.

In other words, the EM vector can oscillate in more than one orientation, and this denote the polarization property of EM wave.
The exact shape traced out in a fixed projective plane by the tip of the electric vector as the plane propagate is a description of the polarization state.
This shape is characterized by the formula: 
\begin{equation}
 \frac{\xi_x^2(t)}{|E_x^0|^2} + \frac{\xi_y^2(t)}{|E_y^0|^2} - 2 \frac{\xi_x \xi_y}{E_x^0 E_y^0} \cos (\Delta \phi) = \sin^2(\Delta \phi)
\end{equation}
where
	$\Delta \phi = \phi_y^0 - \phi_x^0$
The formula characterizes what is termed as the polarization ellipse.

The parameters that describe the polarization ellipse are:
  the intensity orientation angle $\theta$,
  the ellipticity angle $\Psi$,
  and the orientation angle $\chi$.
They are defined as:
\begin{equation}
\tan (\theta ) = \frac{E_y^0}{E_x^0}
\end{equation}
with
	$0 \leq \theta \leq \pi/2$. 
The orientation angle, which is the angle between the major semi-axis of the ellipse and the x-axis, is given by:
\begin{equation}
\tan (\Psi) = \tan (2 \theta) \cos(\Delta \phi)
\end{equation}
with
	$0 \leq \Psi < \pi$. 
The ellipticity, $\epsilon$ is the major-to-minor-axis ratio.
The ellipticity angle which is defined as $\chi = \arctan (1/\epsilon)$ can be calculated as:
\begin{equation}
\sin (2 \chi) = \sin(2 \theta) \sin( \Delta \phi)
\end{equation}
with
	$0 \leq \chi \leq \pi/2$, and $0 \leq \epsilon \leq \infty$

Ignoring the scale of the electrical fields, the elementary wave polarization can be completely described by two parameters, i.e. $E_y^0/E_x^0$ and $\Delta \phi = \phi_y^0 - \phi_x^0$. 
Similarly, ignoring the scale of the ellipse, the polarization ellipse can be completely described by two angles $\Psi$ and $\chi$.

Other parameters includes:
\paragraph{ Complex polarization ratio }
This ratio is important in explaining polarization basis changes.

It is defined as:
\begin{equation}
\rho = \frac{E_y^0}{E_x^0} e^{i \Delta \phi} = \frac{\cos(2 \chi) + i \sin(2 \chi)}{1 - \cos(2\Psi) \cos(2\chi)}
\end{equation}

\paragraph{Jones Vector}

%Let the total wave power:
%\begin{equation}
%|E|^2 = (E_x^0)^2  + (E_y^0)^2
%\end{equation}
%and phase 
%\begin{equation}
%\theta = \tan^{-1} \left( \frac{E_y^0}{E_x^0} \right)
%\end{equation}

The Jones vector is defined as 
\begin{equation}
J = 
\left(
\begin{array}{c}
	\xi_x(t) \\
	\xi_y(t) 
\end{array}
\right)
= \xi_x(t)
\left(
\begin{array}{c}
	1 \\
	\rho
\end{array}
\right)
\end{equation}

Thus the normalized Jones vector can be given as:
\begin{equation}
J = 
\left(
\begin{array}{c}
	\cos(\theta) \cdot e^{i \phi_x^0} \\
	\sin(\theta) \cdot e^{i \phi_y^0}
\end{array}
\right)
=
\left(
\begin{array}{c}
	1\\
	\tan(\theta) \cdot e^{i \Delta \phi}
\end{array}
\right)
=
\left(
\begin{array}{c}
	\cos(\Psi) \cos(\chi) - i \sin(\Psi) \sin(\chi) \\
	\sin(\Psi) \cos(\chi) + i \cos(\Psi) \sin(\chi)
\end{array}
\right)
\end{equation}

The two parameters in the Jones vector can completely describe the polarization state (ellipse) of any elementary wave.
This same Jones vector finds extensive use in quantum machenics, and is interpreted as quantum state vector.
Its bra-ket notation formula, which is normally used in quantum computation contexts is:
\begin{equation}
J = | \Psi \rangle 
= 
\left(
\begin{array}{c}
	\cos(\theta) \cdot e^{i \phi_x^0} \\
	\sin(\theta) \cdot e^{i \phi_y^0}
\end{array}
\right)
\end{equation}

By definitions, the horizontal linear polarized wave is defined as: $J_h = \left(
\begin{array}{c}
 1 \\
 0
\end{array}
\right)$, similarly the vertical linear polarized wave: $J_v = \left(
\begin{array}{c}
 0 \\
 1
\end{array}
\right)$, the $\pm \pi/4$ linear polarized waves $J_{\pm} = \pm \frac{1}{\sqrt{2}} \left(
\begin{array}{c}
 1 \\
 1
\end{array}
\right)$, the left circular polarized wave: $J_l = |L\rangle = \frac{1}{\sqrt{2}} \left(
\begin{array}{c}
 1 \\
 -i
\end{array}
\right)$ and the right circular polarized wave: $J_r = |R\rangle = \frac{1}{\sqrt{2}} \left(
\begin{array}{c}
 1 \\
 i
\end{array}
\right)$

\paragraph{ Stokes Vector}

Another way to look at polarization is through the coherency matrix, and equivalently the Stokes matrix.
The wave coherency matrix is defined as:
\begin{equation}
W_c = \left<
\left(
\begin{array}{c}
 \xi_x \\
 \xi_y
\end{array}
\right)
\left(
\begin{array}{c}
 \xi_x \\
 \xi_y
\end{array}
\right)^{\dagger}
\right>
= \left<
\left(
\begin{array}{c c}
 \xi_x \xi_x^* 	& \xi_x \xi_y^* \\
 \xi_y \xi_x^* 	& \xi_y \xi_y^* 
\end{array}
\right)
\right>
= \left<
\left(
\begin{array}{c c}
 |\xi_x|^2 	& |\xi_x| |\xi_y| e^{-i \Delta \phi} \\
 |\xi_x| |\xi_y| e^{i \Delta \phi} 	& |\xi_y|^2
\end{array}
\right)
\right>
\end{equation}

The Stokes vector is then defined as:
\begin{equation}
S_t = 
\left(
\begin{array}{c}
 S_0 \\
 S_1 \\
 S_2 \\
 S_3
\end{array}
\right)
=
\left(
\begin{array}{c}
 |\xi_x|^2 + |\xi_y|^2 \\
 |\xi_x|^2 - |\xi_y|^2 \\
 2\Re(\xi_x \xi_y^*) \\
 2\Im(\xi_x \xi_y^*)
\end{array}
\right)
\end{equation}
where $\Re$ and $\Im$ respectively denotes the real and imaginary parts of a complex number.

In the case of elementary (or fully polarized) waves, the Stokes vector can be written as:
\begin{equation}
S_t = 
\left(
\begin{array}{c}
 S_0 \\
 S_0 \cos(2\Psi) \cos(2\chi) \\
 S_0 \sin(2\Psi) \cos(2\chi) \\
 S_0 \sin(2\chi)
\end{array}
\right)
\end{equation}

%It should be noted that: the factor of two before the polarization ellipse's angles indicate the fact that
%	any polarization ellipse is practical indistinguishable from on rotated by $180^o$ 
%	or with on with the axis lengths swapped accompanied by a $90^o$ rotation.

At this point, a distinction needs to be highlighted. %explaination is warranted.
%The naive view is that Stokes vector, which has four values are redundant in describing wave's polarization. 
%In contrary, Stokes vector are preferred to Jones vector, as Jones vector is based on the assumption that the wave is elementary and very well behaved.
In comparison to the two-element Jones vector, Stokes vectors have more data, namely four elements. 
The Jones vector can completely describe the polarization states of an elementary wave, whose formulation is based on the assumption that the wave is very well behaved.
In practice, that is not normally the case, because reflected waves are generally combinations of multiple waves which ranged over certain areas of time and frequency. 
As such, the polarization of the waves are normally not fully polarized.
One simple analogy is that the field ratio, i.e. $\xi_x/\xi_y$, or the phase difference, i.e. $\Delta \phi$, are not constant at all times.
%The phenomena can be loosely called wave depolarization. %****IVM: Hai, is this your term or is it also used by others for the same thing? In that case, it might be good to put a reference here. I think the word "loosely" is one of those trigger words that causes the examiner to sit up and read it critically.

The analysis above is especially prevalent in the remote sensing community, where the area of a single resolution cell (pixel) is very large compared with the radar's wavelength.
In such cases, the wave field is very stochastic and only statistical information about the variations and correlations between polarization components can be gathered. 

%(comment: I think it would be better if you can link the two paragraphs together, such as
%�Closely related to the Stoke vector is the concept of Degree of polarization p, which is a quantity used to describe��, or something like this )

\paragraph{Degree of polarization}

Closely related to the Stokes vector is the concenpt of ``Degree of polarization'' $p$, which is a quantity used to describe the portion of an EM wave which is polarized.
A perfectly polarized wave has $p=100\%$, while a fully depolarized wave is characterized by $p=0\%$.
A wave which is partially polarized can be represented as a superposition of a polarized and another unpolarized component.
Thus $0 \leq p \leq 1$.

Depolarization phenomena are explained from the fact that most sources of EM radiation contain a large number of ``elementary'' sources.
The polarization of the electric fields produced by the emitters may not be correlated, which leads to depolarization effects.
If there is partial correlation among the ``elementary'' sources, the light is partially polarized. 
One may then describe the received EM wave polarization in terms of the degree of polarization and the parameters of the polarization ellipse.

In fully polarized wave, the following equation holds:
\begin{equation}
S_0^2 = S_1^2 + S_2^2 + S_3^2
\end{equation}

In depolarized wave, it becomes an in-equation
\begin{equation}
S_0^2 > S_1^2 + S_2^2 + S_3^2
\end{equation}

Then the degree of polarization is defined as
\begin{equation}
p = \frac{\sqrt{ S_1^2 + S_2^2 + S_3^2 }}{S_0}
\end{equation}

In such general case, the Stokes Vector is written as:
\begin{align}
S_t &= 
\left(
\begin{array}{c}
 I \\
 I p \cos(2\Psi) \cos(2\chi) \\
 I p \sin(2\Psi) \cos(2\chi) \\
 I p \sin(2\chi)
\end{array}
\right) \notag \\
 &= I (1 - p)
\left(
\begin{array}{c}
 1 \\
 0 \\
 0 \\
 0
\end{array}
\right) 
+ I p
\left(
\begin{array}{c}
 1\\
 \cos(2\Psi) \cos(2\chi) \\
 \sin(2\Psi) \cos(2\chi) \\
 \sin(2\chi)
\end{array}
\right)
\end{align}

Alternatively, if the Stokes vector is measured and known, then the polarization characteristics of a wave are given as:
$I$, the total power of the measured beam is $I=S_0$,
$p$, the degree of polarization is calculated as $p = \frac{\sqrt{ S_1^2 + S_2^2 + S_3^2 }}{S_0}$,
$2\Psi = \tan^{-1} (S_2/S_1) $ and $2 \chi = \tan^{-1} (S_3 / \sqrt{S_1^2 + S_2^2})$ are the angles that characterize the wave polarization state.

\subsubsection{Full Polarimetry and the Polarimetric Signatures}

This section describes the core concept of POLSAR, i.e. the target's polarimetric signature, 
	and illustrates why research into partial polarimetry is warranted.

POLSAR radar measures the back-scattered echo received as the result of transmitting a signal.
The scattering matrix relates the received signal $E^{Rx}$ with the transmitted $E^{Tx}$ signal as:
\begin{equation}
 E^{Rx} = \frac{e^{ikr}}{r} S E^{Tx}
\end{equation}
%***IVM: Hai, do you need to explain the k & r terms?

Normally, horizontal and vertical linear polarization are used for transmission, and thus the scattering matrix, i.e. $S$, would normally have the following expanded form:
\begin{equation}
 \left( 
\begin{array}{c}
 E_h^{Rx} \\
 E_v^{Rx}
\end{array}
 \right) = \frac{e^{ikr}}{r} 
\left( 
\begin{array}{c c}
 S_{HH} & S_{HV} \\
 S_{VH} & S_{VV}
\end{array}
 \right) 
\left( 
\begin{array}{c}
 E_h^{Tx} \\
 E_v^{Tx}
\end{array}
 \right) 
\end{equation}

The scattering matrix can be called the Jones matrix (relating to the Jones vector) in the forward scattering alignment.
The matrix is normally called the Sinclair matrix in SAR literature, as backward scattering alignment is normally adopted for SAR \cite{Sinclair_1950_ProcsIRE}.
In the case of mono-static SAR, i.e. both radar transmitter and receiver can be considered as located in a single place, physical reciprocity leads to $S_{HV}=S_{VH}$ \cite{Nghiem_1992_RadioSci}.

The matrix can also be ``stratified'' to define the target scattering vector as: 
\begin{equation}
  \label{eqn:Covariance Target Vector of Full Polarimetric SAR}
  \caption{eqn:Covariance Target Vector of Full Polarimetric SAR}
T_{cv} = \left( 
\begin{array}{c}
 S_{HH} \\
 \sqrt{2} S_{HV} \\
 S_{VV}
\end{array}
 \right) 
\end{equation}
or 
\begin{equation}
  \label{eqn:Coherence Target Vector of Full Polarimetric SAR}
  \caption{eqn:Coherence Target Vector of Full Polarimetric SAR}
T_{ch} = \frac{1}{\sqrt{2}} \left( 
\begin{array}{c}
 S_{HH} + S_{VV}\\
 S_{HH} - S_{VV} \\
 S_{HV}
\end{array}
 \right) 
\end{equation}

Similar to the reason that Stokes vectors are preferred to Jones vector in wave polarization, 
	second order statistics of the target scattering matrix are normally preferred.
The covariance matrix is defined as:
\begin{equation}
  \label{eqn:Covariance Matrix of Full Polarimetric SAR}
  \caption{eqn:Covariance Matrix of Full Polarimetric SAR}
\tiny{
C_v = \left< T_{cv} T_{cv}^{*T} \right> = 
\left( 
\begin{array}{c c c}
 S_{HH}S_{HH}^* 	& \sqrt{2}S_{HH}S_{HV}^*	& S_{HH}S_{VV}^* \\
 \sqrt{2}S_{HV}S_{HH}^* & 2 S_{HV}S_{HV}^* 		& \sqrt{2}S_{HV}S_{VV}^* \\		
 S_{VV}S_{HH}^*		& \sqrt{2}S_{VV}S_{HV}^*	& S_{VV}S_{VV}^*
\end{array}
 \right) 
}
\end{equation}
while the coherency matrix is defined as:
\begin{equation}
  \label{eqn:Coherence Matrix of Full Polarimetric SAR}
  \caption{eqn:Coherence Matrix of Full Polarimetric SAR}
\tiny{
C_h = \left< T_{ch} T_{ch}^{*T} \right> = \frac{1}{2}
\left( 
\begin{array}{c c c}
 (S_{HH}+S_{VV})(S_{HH}+S_{VV})^* 	& (S_{HH}+S_{VV})(S_{HH}-S_{VV})^*	& (S_{HH}+S_{VV})S_{HV}^* \\
 (S_{HH}-S_{VV})(S_{HH}+S_{VV})^* 	& (S_{HH}-S_{VV})(S_{HH}-S_{VV})^* 	& (S_{HH}-S_{VV})S_{HV}^*\\		
 2S_{HV}(S_{HH}+S_{VV})^*		& 2S_{HV}(S_{HH}-S_{VV})^*		& S_{HV}S_{HV}^*
\end{array}
 \right) 
}
\end{equation}


Evidently the two matrices are related:
\begin{equation}
C_h = N C_v N^T
\end{equation}
where
	$N = \frac{1}{\sqrt{2}}
\left( 
\begin{array}{c c c}
 1 & 0		& 1\\
 1 & 0		& -1\\
 0 & \sqrt{2}	& 0
\end{array}
\right)$ 

The off-diagonal values in both covariance and coherence matrices are normally complex numbers. 
Equivalent matrices with all real value entries which carry the same information are called the Mueller and the Kennaugh matrix, which are to be used in forward and backward scattering alignment respectively\cite{Guissard_1994_TGRS}. 
They are defined as the matrix that relates the Stokes matrix of sent and received signals.
\begin{equation}
S_t^{Rx} = K S_t^{Tx}
\end{equation}

Since the Stokes and Jones matrices are related, the Kennaugh and Sinclair matrices are also related, and this relationship is given by:
\begin{equation}
K = 1/2 Q^* S \otimes S^* Q^{T*}
\end{equation}
where
	$\otimes$ denotes the Kronecker product, and
	$Q = 
\left( 
\begin{array}{c c c c}
 1 & 0 & 0 &  1 \\
 1 & 0 & 0 & -1\\
 0 & 1 & 1 &  0\\
 0 & i & i &  0
\end{array}
\right)$ 

POLSAR aims to measure the target's polarimetric response.
%Depending on the satellite instrument and processing, either Sinclair matrix or Kennaugh matrix elements are provided.
Typically, the radar transmitter will separately send different horizontal and vertical polarized waves, and measure the response also in horizontal and vertical polarization directions.
The full polarimetric data, will then have four channels, i.e. $\{ S_{HH}, S_{VV}, S_{HV}, S_{VH} \}$. 
The four channel, or quad-pol data, is considered full polarimetric data since it allows one to synthesize the target's response for any given pair of transmitted and received polarization.
Such a process is called polarization synthesis.
The formula is given as:
\begin{equation}
  \label{eqn:Polarization Synthesis Process of Full-Pol data}
  \caption{eqn:Polarization Synthesis Process of Full-Pol data}
I^{Rx} = 
\left( 
\begin{array}{c }
 1 \\
 \cos(2\chi^{Rx}) \cos(2\Psi^{Rx}) \\
 \cos(2\chi^{Rx}) \sin(2\Psi^{Rx}) \\
 \sin(2\chi^{Rx}) 
\end{array}
\right)^T
K
\left( 
\begin{array}{c }
 1 \\
 \cos(2\chi^{Tx}) \cos(2\Psi^{Tx}) \\
 \cos(2\chi^{Tx}) \sin(2\Psi^{Tx}) \\
 \sin(2\chi^{Tx}) 
\end{array}
\right)
\end{equation}
where $I^{Rx}$ is the intensity of the target's response in a given combination of transmitted and received polarizations.

In a full polarimetric implementation, an alternate pulsing scheme is normally used to share a single time slot for transmitting two different polarizations' pulses.
In partial polarimetry modes only a single polarization is sent.
Since the transmitting time-slot no longer needs to be shared in the partial polarimetry modes, higher operating frequency become possible which leads to better (almost double) resolution or coverage.
Clearly in such cases, only two out of four data channels is available, i.e. either $\{S_{HH},S_{HV}\}$ or $\{S_{VH},S_{VV}\}$.
%The limitation in payload is expected to be around for years to come, with new polarimetric projects being planned by various space agencies.
This trade-off is based on limitations of physics, and hence is likely to be around for years to come.

\subsubsection{Partial Polarimetry and the Stokes Vector}

%***IVM: Hai your citations are wrong in this paragraph, they should be \cite (I think I've fixed them all though)
Early works from the radar community had focused on the theoretical introduction of the scattering matrix such as the Sinclair matrix or Mueller matrices, for full, i.e. quad, polarimetric measurements \cite{Sinclair_1950_ProcsIRE}.
%(Sinclair, 1950), (TODO) 
These works were continued by Boerner, and the first polarimetric images were captured by NASA JPL at Caltech in 1991 \cite{Zebker_1991_ProcsIEEE}.

Systems were then engineered and commercial POLSAR satellites, e.g. RadarSat-2, TerraSAR, Alos-PalSAR were launched.
However due to operational restrictions on payload, power and data-downlink, the satellites also offer dual-channel polarimetric SAR modes as a compromise between theoretical polarimetric requirements and practical limitations.
Initially, to the end user, while dual-channel partial polarimetry may not provide the complete polarimetric signature of the target, the loss is compensated for by either better resolutions or higher coverage and lower cost.  

Until very recently, these partial polarimetry modes were not formally taken up by scientists and academia. 
In his pioneer paper in 2005, Souyris\cite{Souyris_2005_TGRS} introduced the so-called compact polarimetry mode and suggested that it may be possible to reconstruct full polarimetric information from compact polarimetric data.
Raney\cite{Raney_2006_IGARSS} also showed that dual-channel polarimetry is not only capable of measuring the complete polarization information of the back-scattered EM wave (via Stokes parameters), it is also capable of detecting the basic target's polarimetric characteristic, e.g. odd or even bounce phenomena.

While the dual-polarization mode sensing has been available operationally for quite some time, partial polarimetry has only recently been investigated in the scientific community. 
In comparison, 
	partial polarimetry measures dual-polarized response from incoming pulses of a single polarization channel,
	while full polarimetry measures the same response from two (generally orthogonal) polarisation channels.
Mathematically, the process is modelled as:
\begin{equation}
 \left( 
\begin{array}{c}
 E_h^{Rx} \\
 E_v^{Rx}
\end{array}
 \right) = \frac{e^{ikr}}{r} 
\left( 
\begin{array}{c}
 S_{H} \\
 S_{V}
\end{array}
 \right) E^{Tx}
\end{equation}

\cite{Souyris_2005_TGRS} pioneered the field with a paper on compact polarimetry.
This essentially describes a partial polarimetry mode in which the incoming pulse is of linear polarization, 
	but is of neither fully horizontal nor vertical angle.
%	rather it is of an odd angle of 
Instead, its incoming polarization of linear polarisationhas an odd angle of $\pi/4$.
%The $\pi/4$ mode has not been implemented in real-hardware, but only been simulated using polarization basis change principles.
The partial polarimetric covariance matrix is then given as
%\begin{equation}
%J = 
%\left(	
%\begin{array}{c c}
%	|s_{HH}|^2 + |s_{HV}|^2 + 2 \Re{(s_{HH}s_{HV}^*)}
%& 	s_{HH}s_{VV}^* + |s_{HV}|^2 + s_{HH}s_{HV}^* + s_{HV}s_{VV}^*\\
%	s_{HH}^*s_{VV} + |s_{HV}|^2 + s_{HV}s_{HH}^* + s_{VV}s_{HV}^*
%& 	|s_{VV}|^2 + |s_{HV}|^2 + 2 \Re{(s_{VV}s_{HV}^*)}
%\end{array}
%\right)
%\end{equation}

%****IVM: Hai, the equation doesn't fit - can you do something like this instead? (and same for the next eqn too)-> 
%\begin{equation}
%J = 
%\left(	
%\begin{array}{c c}
%	|s_{HH}|^2 + |s_{HV}|^2 }              		& 	{s_{HH}s_{VV}^* + |s_{HV}|^2} \\
%	{+ 2 \Re{(s_{HH}s_{HV}^*)}          		&  {+ s_{HH}s_{HV}^* + s_{HV}s_{VV}^*}\\
%\vspace{0.1}
%	{s_{HH}^*s_{VV} + |s_{HV}|^2}         		& 	{|s_{VV}|^2 + |s_{HV}|^2}\\
%	{+ s_{HV}s_{HH}^* + s_{VV}s_{HV}^*} 	& { + 2 \Re{(s_{VV}s_{HV}^*)}}
%\end{array}
%\right)
%\end{equation}

\begin{equation}
J = 
\left(	
\begin{array}{c c}
	|s_{HH}|^2 + |s_{HV}|^2               		& 	{s_{HH}s_{VV}^* + |s_{HV}|^2} \\
	+ 2 \Re{(s_{HH}s_{HV}^*)}          		&  {+ s_{HH}s_{HV}^* + s_{HV}s_{VV}^*}\\
        \vspace{0.01 in} & \vspace{0.01 in}\\
	{s_{HH}^*s_{VV} + |s_{HV}|^2}         		& 	{|s_{VV}|^2 + |s_{HV}|^2}\\
	{+ s_{HV}s_{HH}^* + s_{VV}s_{HV}^*} 	& { + 2 \Re{(s_{VV}s_{HV}^*)}}
\end{array}
\right)
\end{equation}


%*** IVM, I changed the following reference to a \citet
In \cite{Raney_2006_IGARSS}, Raney demonstrated a different perspective by proposing a hybrid architecture.
He argued that the dual polarisation measurements effectively capture the polarisation states of the single back-scattered signal resulting from a single incoming pulse.
Thus the polarisation choice of the receiving antenna is trivial (as long as they are orthogonal).
The important parameter is the choice of the polarisation to be used for the incoming pulse.
His hybrid architecture is again a kind of partial polarimetry in which the incoming pulse is of circular polarisation, instead of the normally encountered linear polarisation.
The partial covariance matrix then becomes:
\begin{equation}
J = \frac{1}{2} \left(	
\begin{array}{c c}
	|s_{HH}|^2 + |s_{HV}|^2 
& 	s_{HH}s_{HV}^* - s_{HV}s_{VV}^* \\
        + i (s_{HH}s_{HV}^* - s_{HV}s_{VV}^*)
&       - i (|s_{HV}|^2 + s_{HH}s_{VV}^*) \\ 
        \vspace{0.01 in} & \vspace{0.01 in}\\
	s_{HH}s_{HV}^* - s_{HV}s_{VV}^* 
& 	|s_{VV}|^2 + |s_{HV}|^2  \\
        + i (|s_{HV}|^2 + s_{HH}s_{VV}^*)
&       + i (s_{HV}s_{VV}^* - s_{VV}s_{HV}^*)  
\end{array}
\right)
\end{equation}

The benefits of this hybrid architecture, according to Raney are that it can measure certain physical polarimetric phenomena such as the odd and even bounce phenomena.
The suggestion is evidenced by calculating the response of known odd-bounce trihedral and even-bounce dihedral when the incident wave is right circular polarized.
\begin{equation}
\Gamma_{tri} = \left(
\begin{array}{c c}
1 & 0 \\
0 & -1
\end{array}
\right)
\end{equation}
Then the response will be reversed as it can be calculated as:
\begin{equation}
J_r = \left(
\begin{array}{c c}
1 & 0 \\
0 & -1
\end{array}
\right) \cdot
\left(
\begin{array}{c}
1 \\
-i
\end{array}
\right) = \left(
\begin{array}{c}
1 \\
i
\end{array}
\right) 
\end{equation}
with
	$ \left( \begin{array}{c} 1 \\ -i \end{array} \right) $ indicating right circular polarization transmitted, and
	$ \left( \begin{array}{c} 1 \\ i \end{array} \right) $ indicating left circular polarization received.

For a dihedral target, whose main axis is at an angle $\theta$ with respect to the horizontal direction, the Sinclair matrix is given as:
\begin{equation}
\Gamma_{dih} = 
\left(
\begin{array}{c c}
\cos(2\theta) & \sin(2\theta) \\
\sin(2\theta) & \cos(2\theta)
\end{array}
\right)
\end{equation}
The even bounce phenomena is detectable as its polarimetric response, which is then:
\begin{equation}
J_r = \left(
\begin{array}{c c}
\cos(2\theta) & \sin(2\theta) \\
\sin(2\theta) & \cos(2\theta)
\end{array}
\right) \cdot
\left(
\begin{array}{c}
1 \\
-i
\end{array}
\right) = \left(
\begin{array}{c}
1 \\
-i
\end{array}
\right) 
\end{equation}
In \cite{Raney_2007_IGARSS}, Raney shows that hybrid circular modes bring many conveniences to hardware calibration.
Also in a follow up study \cite{Raney_2007_TGRS}, he demonstrated the classification power of hybrid polarimetry.
Furthermore, Lardeux \cite{Lardeux_2007_ESASP} found that for target classification using Support Vector Machine (SVM) techniques, compact polarimetry mode shows comparable results to full polarimetric data.
Notably, the hybrid architectural concept has been adopted and brought towards implementation by NASA's upcoming Desdynl satellite \cite{Raney_2009_RadarConf}.

The research into partial polarimetry is convenient as its data can be simulated using polarisation basis transformations such as those described above.
Hardware realisation may not be needed until the final physical prototype stages.
This methodology has been demonstrated and is widely used in the community.
%We hypothesize that the partial polarimetric covariance matrix, which is essentially a variant of the Stokes vector, can be speckle filtered, using the familiar intensity speckle filtering algorithms.
The focus into partial polarimetry allows analysis with a smaller matrix, while its implications can be extended not only to full polarimetry modes but also towards bi-static polarimetric scenarios.
Besides that, a number of research questions in this new field are still open, relevant and exciting.

Souyris, in 2005, published a paper \cite{Souyris_2005_TGRS} which claimed and presented an algorithm to reconstruct full polarimetric information from the given partial compact polarimetric data.
The algorithm is claimed to work on natural surfaces.
In \cite{Nord_2009}, Nord reviewed the reconstruction problem for various different partial polarimetry modes, and an improved algorithm is given. %****IVM: Hai are Nord / Ainsworth single author? If not, you need an "et. al."
In \cite{Ainsworth_2009_ISPRS}, Ainsworth reviewed performance of target classification on different partial modes, reconstructed pseudo-full mode and full polarimetric data.
Chapter 3 will review this sub-topic further and introduce our work on expanding the reconstruction to urban man-made structures and surfaces (from the original system which was relevant to natural cover only).

For the tasks of target detection and classification, we believe that partial dual polarization provides a unique proposition.
Compared with single-channel systems, the dual channel provides additional and useful polarimetric information. 
Compared with full polarimetric data, the better spatial resolution of dual channel systems could compensate for the loss of reconstructed polarimetric information, especially for the purpose of target detection and classification.

%In fact, this project is intended to provide help for a new project at EADS Innovation Works Singapore to investigate the use of polarimertic data in ship monitoring over the oceans.
%It is hypothesized that using polarimetric information would improve the possibilities of detecting ships travelling in the open seas.
%Evidently partial polarimetric modes are good candidates for better detection rates.
%
%The scientific application of POLSAR include: 
%	oil slick detection \cite{Nunziata_2009_IGARSS, Gambardella_2007_ESASP}, 
%	ship monitoring \cite{Kozai_2004_OCEANS}, 
%	sea ice monitoring \cite{Wakabayashi_2004_TGRS, Nakamura_2005_IGARSS}, 
%	volcano lava-flow monitoring \cite{Dierking_1998_IGARSS}, 
%	agricultural crop monitoring \cite{Bouvet_2009_TGRS, Hajnsek_2008_IGARSS}, 
%	planetary exploration \cite{Rosenbush_2009_JQuantSpecRadTransfer} 
%	and other surface parameter estimations \cite{TruongLoi_2009_TGRS, Touzi_2009_TGRS, Hajnsek_Thesis_2001}. 
%The remaining text of this chapter briefly look at recent literature in the context of ship detection and monitoring.
%The topic of ship detection and monitoring has already been tackled quite extensively for single channel SAR data \cite{Tello_2005_IGARSS, Tello_2006_ISPRSJPhotogrammetryRemoteSensing}.
%In fact, a number of countries have reported to implement such systems, and some uses them in everyday operations.
%The countries includes: USA \cite{Montgomery_1998_IGARSS}, Canada \cite{Vachon_2000_JHUApplPhysics}, France \cite{Losekoot_2005_IGARSS}, Norway \cite{weydahl_2007_RemoteSenseGISGeology} and Spain \cite{Margarit_2009_RemoteSensing}.
%The common scenarios in use are normally: 
%	a designated area is continuously monitored through the use of some automatic ship detection algorithm.
%If an unknown ship is detected, some manual intervention is to be applied, normally in the form of a navy ship being sent to further investigate.
%The reports however did not elaborate about the accuracy of the detection algorithms.
%
%A recent trend that is reported in the literature is the use of AIS (Automatic Identification System).
%In 2007, Vachon \cite{Vachon_2007_IGARSS} reported his study in Canada, which aimed to match ships detectable in SAR images with AIS reported locations.
%In 2009, Grasso \cite{Grasso_2009_ISDA} evaluated the performance of ship detection algorithms using AIS as ground truth.
%In our planned approach, we are prepared to study the performance improvement through using partial polarimetry data in comparison to normal single-channel SAR approaches.
%Ground truth for such study will also be using synchronized AIS data.

%\subsubsection{The Polarization Basis Transformation}
%
%Assuming that the polarization states are known, the wave representation can be given in terms of $\xi_x$ and $\xi_y$ as:%****IVM, you need to mention A_0.... as well as \phi
%\begin{equation*}
%\left( 
%\begin{array}{c}
% \xi_x \\
% \xi_y
%\end{array}
%\right) 
%= A_0 e^{-i \phi_0} \left(
%\begin{array}{c c}
% \cos(\Psi) & -\sin(\Psi) \\
% \sin(\Psi) & \cos(\Psi)
%\end{array}
%\right) 
%\left(
%\begin{array}{c}
% \cos(\chi) \\
% i \sin(\chi)
%\end{array}
%\right) 
%\end{equation*}
%The representation of a wave, however, depends on the choice of orthogonal reference frame basis.
%
%Consider two different orthogonal basis, $\{ x,y \}$ and $\{ h,v \}$.
%A given Jones vector can be measured in both basis as:
%$J_{xy}= 
%\left(
%\begin{array}{c}
% \xi_x \\
% \xi_y
%\end{array}
%\right) $ and 
%$J_{hv}
%= \left(
%\begin{array}{c}
% \xi_h \\
% \xi_v
%\end{array}
%\right)$�
%.
%The transformation from $\{h,v\}$ to $\{x,y\}$ is governed by a unitary transformation matrix $U_2$, which is given as:
%\begin{equation*}
%\left(
%\begin{array}{c}
% \xi_x \\
% \xi_y
%\end{array}
%\right)
%= U_2
%\left(
%\begin{array}{c}
% \xi_h \\
% \xi_v
%\end{array}
%\right)
%\end{equation*}
%
%Boerner \cite{Boerner_1991_ProcsIEEE} has given $U_2$ as:
%\begin{equation*}
%U_2 = \frac{1}{\sqrt{1+\rho\rho^*}}
%\left(
%\begin{array}{c c}
% 1 & -\rho^* \\
% \rho & 1
%\end{array}
%\right)
%\left(
%\begin{array}{c c}
% e^{-i\delta} & 0 \\
% 0 & e^{i\delta}
%\end{array}
%\right)
%\end{equation*}
%where
%	$\rho$ is the complex polarisation ratio of the Jones vector for the first new basis, and
%	$\delta = \tan^{-1}(\tan(\Psi) \tan(\chi)) - \phi_0$ is the phase reference for the new basis, 
%	and $\delta$ is required for the determination of the initial phase of the Jones vector in the new reference basis.
%
%In the same vein, \cite{Pottier_2008_IGARSS} gives %****IVM pls change this to a proper \cite{}
%\begin{equation*}
%U_2 = \left(
%\begin{array}{c c}
% \cos(\Psi) & -\sin(\Psi) \\
% \sin(\Psi) & \cos(\Psi)
%\end{array}
%\right)
%\left(
%\begin{array}{c c}
% \cos(\chi) & i\sin(\chi) \\
% i\sin(\chi) & \cos(\chi)
%\end{array}
%\right)
%\left(
%\begin{array}{c c}
% e^{-i \phi_0} & 0 \\
% 0 & e^{i \phi_0}
%\end{array}
%\right)    
%\end{equation*}
%
%From these results, it is then possible to derive equations for the transformation of the scattering matrix.
%Noting that by definition, $J_{xy}^r=S_{xy} J_{xy}^t$, $J_{hv}^r=S_{hv} J_{hv}^t$ 
%and that $J_{xy}^*= U_2 J_{hv}^*$, also that $(U_2)^{-1} = (U_2)^T$,
%it is then trivial to show that:
%\begin{equation*}
%S_{xy} = U_2 S_{hv} U_2^T
%\end{equation*}
%
%%***IVM: again, I think you need to introduce some of the variables below, like delta, rho. Is rho* the complex conjugate? Better to mention that too....
%Using the result above, we then have:
%\begin{equation*}
%S_{xx} = \frac{1}{1+\rho \rho^*} 
%\left( 
%S_{HH} e^{-2i\delta} - \rho^* (S_{HV}+S_{VH}) + (\rho^*)^2 S_{VV} e^{2i\delta}
%\right)
%\end{equation*}
%\begin{equation*}
%S_{xy} = \frac{1}{1+\rho \rho^*} 
%\left( 
%\rho S_{HH} e^{-2i\delta} + S_{HV} -\rho \rho^* S_{VH} - \rho^* S_{VV} e^{2i\delta}
%\right)
%\end{equation*}
%\begin{equation*}
%S_{yx} = \frac{1}{1+\rho \rho^*} 
%\left( 
%\rho S_{HH} e^{-2i\delta} - \rho \rho^* S_{HV} + S_{VH} - \rho^* S_{VV} e^{2i\delta}
%\right)
%\end{equation*}
%\begin{equation*}
%S_{yy} = \frac{1}{1+\rho \rho^*} 
%\left( 
%\rho^2 S_{HH} e^{-2i\delta} + \rho (S_{HV} + S_{VH}) + S_{VV} e^{2i\delta}
%\right)
%\end{equation*}
%
%In the case of mono-static POLSAR, i.e. $S_{HV}=S_{VH}=S_{XX}$ and $S_{xy}=S_{yx}$, the equations become
%\begin{equation*}
%S_{xx} = \frac{1}{1+\rho \rho^*} 
%\left( 
%S_{HH} e^{-2i\delta} - 2 \rho^* S_{XX} + (\rho^*)^2 S_{VV} e^{2i\delta}
%\right)
%\end{equation*}
%\begin{equation*}
%S_{yx} = S_{xy}= \frac{1}{1+\rho \rho^*} 
%\left( 
%\rho S_{HH} e^{-2i\delta} + (1-\rho \rho^*) S_{XX} - \rho^* S_{VV} e^{2i\delta}
%\right)
%\end{equation*}
%\begin{equation*}
%S_{yy} = \frac{1}{1+\rho \rho^*} 
%\left( 
%\rho^2 S_{HH} e^{-2i\delta} + 2 \rho S_{XX} +  S_{VV} e^{2i\delta}
%\right)
%\end{equation*}
%
%The results here, together with Stokes Vector's formula, can be used to prove the following identity,
%  which can form the basis of a novel approach in POLSAR speckle filtering: %used throughout this thesis:
%\begin{equation*}
%S_t = 
%\left(
%	\begin{array} {c}
%		S_hS_h + S_vS_v \\
%		S_hS_h - S_vS_v \\
%		2 \Re{(S_hS_v^*)} \\
%		2 \Im{(S_hS_v^*)}
%	\end{array}
%\right)
%= 
%\left(
%	\begin{array} {c}
%		S_hS_h + S_vS_v \\
%		S_hS_h - S_vS_v \\
%		S_+S_+ - S_-S_- \\
%		S_lS_l - S_rS_r
%	\end{array}
%\right)
%\end{equation*}
%where $S_t$ is the Stokes vector and $S_+S_+$, $S_-S_-$, $S_rS_r$, $S_lS_l$ are the intensity of received signals in $+45^o$ linear, $-45^o$ linear, right circular and left circular polarization respectively.
%
%%The transformation from $\{h,v\}$ to $\{+\pi/4,-\pi/4\}$ as well as from $\{h,v\}$ to circular basis $\{l,r\}$ are special cases of the above formula.
%%Ignoring the determination of initial phases, we have: 
%%	the measured signal at $\pi/4$ linear polarizer / projection as: $\xi_+ = \frac{\xi_h + \xi_v}{\sqrt{2}}$,
%%	at $-\pi/4$ linear polarizer as: $\xi_- = \frac{\xi_h - \xi_v}{\sqrt{2}}$, 
%%	at left circular polarization as: $\xi_l = \frac{\xi_h + i \xi_v}{\sqrt{2}}$, 
%%	and at right circular polarization as $\xi_r = \frac{\xi_h - i \xi_v}{\sqrt{2}}$.
%%
%%The Stokes vectors, when aligned at these transformed projections are given by the following well-known equations
%%\begin{equation}
%%S_t = \left(
%%\begin{array}{c}
%% |\xi_h|^2 + |\xi_v|^2 \\
%% |\xi_h|^2 - |\xi_v|^2 \\
%% 2 \Re( \xi_h \xi_v^* ) \\
%% 2 \Im( \xi_h \xi_v^* ) 
%%\end{array}
%%\right)
%%= \left(
%%\begin{array}{c}
%% |\xi_+|^2 + |\xi_-|^2 \\
%% -2 \Re( \xi_+^* \xi_- ) \\
%% |\xi_+|^2 - |\xi_-|^2 \\
%% 2 \Im( \xi_+^* \xi_- ) 
%%\end{array}
%%\right)
%%= \left(
%%\begin{array}{c}
%% |\xi_l|^2 + |\xi_r|^2 \\
%% 2 \Re( \xi_l^* \xi_r ) \\
%% -2 \Im( \xi_l^* \xi_r ) \\ 
%% |\xi_l|^2 - |\xi_r|^2 
%%\end{array}
%%\right)  
%%\end{equation}


%\subsection{The Stochastics Nature of SAR and POLSAR}
\section{Current methods in SAR and POLSAR Data Processing}

\subsection{Current Statistical Models for SAR and POLSAR data}

%In practice, discrimination measures are not the only way to carry out classification tasks in POLSAR.
%Another common approach is to use POLSAR observables identified by different target decomposition theorems.
Different target decomposition theorems have identified many possible scalar observables for the complex POLSAR data.
In \cite{Alberga_2008_IJRS_4129}, the performance of different scalar POLSAR observables for classification purposes are evaluated.
While many scalar observables for POLSAR were presented, their corresponding statistical models and classifiers were not available.
Furthermore, at its conclusion, the paper indicated that ``it is impossible to identify the best one''
  as these ``representations ... do not show a precise trend''.
%Instead an MLP classification approach were employed. 
Although, to be fair, these observables are identified to describe a decomposed portion of the complex POLSAR data,
  and thus they were not precisely defined to be used individually for the purpose of single-handedly representing POLSAR data.

Given that the joint distribution for POLSAR is known to be the multi-variate complex Wishart distribution,
  it is possible to derive the scalar statistical models for some univariate POLSAR observables.
However, such derivations are no trivial task,
  as the data is much more complex and the field is much less mature than that of traditional SAR statistical modelling.
Thus the strategy of the pioneering works tend to be to tackle simpler versions of polarimetry.

In the literature, POLSAR data is commonly represented by its covariance matrix.
For full polarimetric SAR data (full-pol) the covariance matrix has the form of
\begin{equation}
C_3 = 
\begin{bmatrix}
S_{hh}^2 & \sqrt{2}S_{hh}S_{hv}^* & S_{hh}S_{vv}^* \\ 
\sqrt{2}S_{hh}^*S_{hv} &2S_{hv}^2  & \sqrt{2}S_{hv}S_{vv}^*\\ 
S_{hh}^*S_{vv} & \sqrt{2}S_{hv}^*S_{vv} & S_{vv}^2
\end{bmatrix}
\end{equation}
For partial polarimetric SAR data (part-pol), it reduces to either of the two following equations:
\begin{equation}
  C_{2,h} =
  \begin{bmatrix}
    S_{hh}^2 & S_{hh}S_{hv}^* \\
    S_{hh}^*S_{hv} & S_{hv}^2
  \end{bmatrix};
  C_{2,v} =
  \begin{bmatrix}
    S_{hv}^2 & S_{hv}S_{vv}^* \\
    S_{hv}^*S_{vv} & S_{vv}^2
  \end{bmatrix}
\end{equation}
Obviously, the statistical behaviour of $C_3$ can be understood by understanding the behaviour of $C_{2h},C_{2v}$ and
\begin{equation}
  C_{2,c} = 
\begin{bmatrix}
    S_{hh}^2 & S_{hh}S_{vv}^* \\
    S_{hh}^*S_{vv} & S_{vv}^2
\end{bmatrix}
\end{equation}

While $C_{2v}, C_{2h}$ are physically related to the Stokes parameters of EM waves,
  $C_{2c}$ is %a totally made-up mathematical term from the full-pol covariance matrix.%***IVM changed it to:
  an artificial term derived from the ful-pol covariance matrix.
In fact, in the field of polarimetric SAR, $C_{2v}, C_{2h}$ are called cross-pol and $C_{2c}$ is called the co-pol covariance matrice.
Their main difference lies in investigating the so-called complex correlation coefficient, defined as:
\begin{equation}
  \rho = \frac{E \left( S_{pq}S_{rs}^* \right) }{\sqrt{E \left( S^2_{pq} \right) E \left( S^2_{rs} \right) }} = |\rho| e^{j \phi}
\end{equation}
with $\phi=arg(\rho)$ is the phase difference angle.
In $C_{2v}$ and $C_{2h}$ that are commonly used in optical physics, $|\rho_v|,|\rho_h|$ are insignificant,
  and $\phi_v,\phi_h$ are very random, due to little correlation existing between orthogonal polarizations in monochromatic EM waves.
%By analogy, in POLSAR ``natural surfaces'' data, it is common to consider: $|\rho_v|=|\rho_h|=0$
Their equivalent lies in the ``natural surface'' case of POLSAR,
  where  it is common to assume $|\rho_v|=|\rho_h|=0$.
This is, however, not true in the general case of POLSAR (i.e. including over built-up areas).
Specifically for $C_{2c}$, since there is correlation between the two co-pol channels,
  $|\rho_c|$ is almost always significant and $\phi_c$ is also not entirely random.

In the field of optical physics,
  under the assumption of no correlation between cross-polarized channels,
  \cite{Barakat_1985_IJOptics}, \cite{Eliyahu_1993_PhysRevE_2881} and \cite{Brosseau_1995_AppliedOptics_4788} derived the statistical models for $C_{2h},C_{2v}$.
The observables that have been studied are:
\begin{enumerate}
\item the polarization ellipse parameters \cite{Barakat_1985_IJOptics}: i.e.
  \begin{enumerate}
  \item the total power $S_0=C_{11}^2 + C_{22}^2$,
  \item the ratio of minor to major axis of the polarization ellipse  $\epsilon=\lambda_2/\lambda_1$,
  \item and $\psi$ the angle that the ellipse major axis makes with horizontal axis,
  \end{enumerate}
\item the Stokes parameters \cite{Eliyahu_1993_PhysRevE_2881}: i.e.
  \begin{enumerate}
  \item $S_0 = C_{11}^2 + C_{22}^2$
  \item $S_1 = C_{11}^2 - C_{22}^2$
  \item $S_2 = \Re (C_{12})$
  \item $S_3 = \Im (C_{12})$
  \end{enumerate}
\item and the normalized Stokes parameters $s_i=S_i/S_0, i>0$ \cite{Brosseau_1995_AppliedOptics_4788}
\end{enumerate}
In the field of POLSAR, under the assumption of significant correlation between polarized channels,
  \cite{Joughin_1994_TGRS_562}, \cite{Lee_1994_TGRS_1017} and \cite{Touzi_1996_TGRS_519} derived the statistical models for not only $C_{2h},C_{2v}$ but also $C_{2c}$.
Specifically the following observables were studied:  
  \begin{enumerate}
  \item cross-pol ratio $r_{HV/HH} = |S_{HV}|^2/|S_{HH}|^2$ \cite{Joughin_1994_TGRS_562},
  \item co-pol ratio $r_{VV/HH} = |S_{VV}|^2/|S_{HH}|^2$ \cite{Joughin_1994_TGRS_562},
  \item co-pol phase difference $\phi_{VV/HH} = arg(S_{VV}S_{HH}^*) $ \cite{Joughin_1994_TGRS_562,Lee_1994_TGRS_1017},
  \item magnitude $g=|avg(S_{pq}S_{rs}^*)|$ \cite{Lee_1994_TGRS_1017},
  \item normalized magnitude $\xi = \frac{|avg(S_{pq}S_{rs}^*)|}{\sqrt{avg(|S_{pq}|^2) avg(|S_{rs}|^2)}}$ \cite{Lee_1994_TGRS_1017},
  \item intensity ratio $w = avg(|S_{pq}|^2)/avg(|S_{rs}|^2)$ \cite{Lee_1994_TGRS_1017},
  \item and the Stokes parameters $S_i,0 \leq i \leq 3$ \cite{Touzi_1996_TGRS_519}. 
  \end{enumerate}

Furthermore, other studies have offered
  %\cite{Lopez-Martinez_2003_TGRS_2232} and \cite{Erten_2012_Sensors_2766} gives
  statistical model for
  each element of the POLSAR covariance matrix $S_{pq}S_{rs}^*$ \cite{Lopez-Martinez_2003_TGRS_2232}
  as well as the largest eigen-value of the covariance matrix $\lambda_1$ \cite{Erten_2012_Sensors_2766}.
While these models undoubtedly help in understanding SAR data,
  individually none of them statisfy the dual criteria of
  1) resulting in statistically consistent discrimination measures and
  2) being an representative observable for the complex POLSAR data, which are the objectives of this thesis.
  %that they model have the power to represent the multi-demensional POLSAR covariance matrix.

\subsection{Existing Discrimination Measures for (POL)SAR data}

The commonly used measure of distance for matrices are either the Euclidean or Manhattan distances, defined as:
\begin{align}
  d(C_x,C_y) &= \sum_{i,j} |\mathbb{R} (C_x - C_y)_{i,j}| + \sum_{i,j} |\mathbb{I} (C_x - C_y)_{i,j}| \\
  d(C_x,C_y) &= \sqrt{\sum_{i,j} |C_x - C_y|_{i,j}^2 }
\end{align}
where $C_{i,j}$ denotes the (i,j) elements of the POLSAR covariance matrix C,
 $||$ denotes absolute values
and $\mathbb{R},\mathbb{I}$ denote the real and imaginary parts respectively.
However, in the context of POLSAR% covariance matrix
, these dis-similarity measures are not widely used 
  mainly due to the multiplicative nature of the noisy data.

In the field of POLSAR, the Wishart distance is probably the most widely used, as part of the well-known Wishart classifier \cite{Lee_1999_TGRS}.
It is defined \cite{Lee_1994_IJRS_2299} as:
\begin{equation}
  d(C_x,C_y) = \ln|C_y| + tr(C_xC_y^{-1})
\end{equation}
where $tr(C)$ denotes the trace of the matrix C. 
As a measure of distance, its main disadvantage is that $d(C_y,C_y) = \ln|C_y| \neq 0$.

Recent works have suggested alternative dissimilarity measures including the symmetric and asymmetric refined Wishart distance \cite{Anfinsen_2007_ESA_POLINSAR},
\begin{align}
  d(C_x,C_y) &= \frac{1}{2} tr(C_x^{-1}C_y + C_y^{-1}C_x) - d \\
    d(C_x,C_y) &= \ln|C_x| - \ln|C_y| + tr(C_xC_y^{-1}) - d
\end{align}
the Bartlett distance \cite{Kersten_2005_TGRS_519},
  \begin{align}
  d(C_x,C_y) &= 2 \ln |C_{x+y}| - \ln |C_x| - \ln |C_y| - 2d\ln2
  \end{align}
the Bhattacharyya distance \cite{Lee_2011_IGARSS_3740},
\begin{equation}
  r(C_x,C_y) = \frac{|C_x|^{1/2} |C_y|^{1/2}}{|(C_x+C_y)/2|}
\end{equation}
and the Wishart Statistical test distance \cite{Cao_2007_TGRS_3454}
\begin{equation}
  d(C_x,C_y) = (L_x + L_y) \ln|C| - L_x \ln|C_x| - L_y\ln|C_y|
\end{equation}


Closer investigation of these dis-similarity measures reveals that most are related in some ways. %to each other.
The Bhattacharyya distance is easily shown to be related to the Barlett distance.
At the same time the Barlett distance can be considered a special case of the Wishart Statistical Test distance,
  when the two data sets have the same number of looks, i.e. $L_x=L_y$.
The close relation among the measures is further supported by the fact that
  all of their proposing papers refered the same statistical model developed in \cite{Conradsen_2003_TGRS_4} as a foundation.
In \cite{Conradsen_2003_TGRS_4}, to determine if the two scaled multi-look POLSAR covariance matrix $Z_x$ and $Z_y$,
  which have $L_x$ and $L_y$ as the corresponding number of looks,
  come from the same underlying stochastic process,
the likelihood ratio statistics for POLSAR covariance matrix is considered:  
\begin{equation}
  Q = \frac{(L_x+L_y)^{d \cdot (L_x+L_y)}}{L_x^{d \cdot L_x} L_y^{d \cdot L_y}} \frac{|Z_x|^{L_x} |Z_y|^{L_y} }{|Z_x+Z_y|^{(L_x+L_y)}}
\end{equation}

Taking the log-transformation of the above equation, and note that $C_{vx} = Z_x / L_x$, $C_{vy} = Z_y / L_y$ and $C_{vxy} = (Z_x + Z_y)/(L_x + L_y)$ then:
\begin{equation}
  \label{eqn:Complex Wishart Distribution Likelihood Test Statistics}
  \caption{eqn:Complex Wishart Distribution Likelihood Test Statistics}
  Q = \frac{|C_{vx}|^{L_x} \cdot |C_{vy}|^{L_y} }{|C_{vxy}|^{L_x + L_y}}  
\end{equation}
\begin{equation}
  \label{eqn:Complex Wishart Distribution Likelihood Test Statistics (Log Domain)}
  \caption{eqn:Complex Wishart Distribution Likelihood Test Statistics (Log Domain)}
  \ln Q = L_x \ln |C_{vx}| + L_y \ln |C_{vy}| - (L_x + L_y) \ln |C_{vxy}| 
\end{equation}

To detect changes, a test statistic is developed based on this measure of distance.
This means a distribution is to be derived for the dissimilarity measure.
However, in the original work \cite{Conradsen_2003_TGRS_4}, only an asymptotic distribution has been proposed.

\subsection{Current Methods for SAR speckle filtering}

%This section starts with a description of the stochastic nature of SAR data. (Comment: as mentioned � this should be at the beginning of the chapter)
%SAR speckle filtering is then reviewed within a statistical estimation theory framework. 
%Various different approaches to SAR speckle filtering are reviewed, 
%	ranging from Generalized Least Squared Error algorithms
%	%ranging from a discrete cartoon like algorithm 
%	to the more convoluted Maximum A Posteriori (MAP) approaches.
%	%to recent more continous multi-resolution wavelet or partial differental equations based algorithms. 

Although speckle has been extensively studied for decades, speckle reduction remains one of the major issue in SAR imaging process. 
Many reconstruction filters have been proposed, and they can be classified into two main categories: 
	Weighted Least-Square Error de-speckling using the speckle model; 
	and maximum a posteriori (MAP) de-speckling using the product model. 
In this section, various different approaches to SAR speckle filtering are reviewed.
A recent review of this topic is available in \cite{Argenti_GRSM_2013}.
%The famous Lee, Kuan, and Frost ����ए�lters in the ����ए�rst category provide MMSE reconstructions based on measured local statistics.
%In the second category, di����ए�erent scene distributions are used: Gaussian, Gamma, and parameter-based distributions.

In the first category the speckle random process is assumed to be stationary over the whole image. 
Then speckle models are proposed, such as the multiplicative speckle model is first approximated by a linear model.
A class of speckle reduction filters is formulated 
as follows: 
\begin{equation}
  \label{eqn:Weighted Least-Square Error}
  \caption{eqn:Weighted Least-Square Error}
\hat{R}(t) = I(t) \cdot W(t) + \bar{I}(t) \cdot (1 - W(t)) 
\end{equation}
where
	$\hat{R}(t)$ is the filter response,
	$I(t)$ is the intensity at the centre of the moving window,
	$\bar{I}(t)$ is the averaged intensity for the whole processing window,
	$W(t)$ is a weight function,

For the Lee filter \cite{Lee_PAMI_1980}, the weight function is proposed to be:
\begin{equation}
W(t) = 1 - \frac{C_u^2}{C_I^2} 
\end{equation}
where
	$C_u$ is the variational coefficient of noise, i.e. $C_u=std(u)/avg(u)$, and
	$C_I$ is the variational coefficient of the true image, i.e. $C_I=std(I)/avg(I)$.

In the approach of the Kuan filter \cite{Kuan_1985_PAMI}, the multiplicative speckle model is first transformed into a single-dependent additive noise model, and then the MMSE criterion is applied. 
The speckle filter thus has the same form as the Lee filter but with a different weighting function 
\begin{equation}
W(t) = \frac{1 - \frac{C_u^2}{C_I^2} }{1+ C_u^2} 
\end{equation} 
The Kuan filter takes into account the dependency of noise on the signal. 
In the special case of noise being independent of the underlying signal, the Kuan filter would be exactly the same as the Lee filter.
From this point of view, it can be considered to be superior to the Lee filter.

The Frost filter \cite{Frost_PAMI_1982} is different from the Lee and Kuan filters, where the scene reflectivity is estimated by convolving the observed image with the impulse response of the SAR system. 
The impulse response of the SAR system is obtained by minimizing the mean square error between the observed image and the scene reflectivity model which is assumed to be an autoregressive process. 
The filter impulse response can, after some simplification, be written as:
\begin{equation}
H(t) = K_1 e^{-K_2 C_I^2(t)}
\end{equation}
where
	$K_1$ is some normalizing coefficient to allow preservation of signal mean values, and
	$K_2$ is a user-selected filter parameter.

As can be seen from these three most recognised algorithms in this field, they share similar principles.
When the variation coefficient $C_I$ is small, the filter behaves like a low pass filter smoothing out the speckles. 
When $C_I$ is large, it has a tendency to preserve the original observed image.
The choice of functions relating speckle suppression power to local variations may be different, 
	the weighted average as well as the weighted least square principles are apparently common.

As will be discussed at length in Chapter 4, SAR data is heteroskedastic.
Our literature search yielded only a single article \cite{Amirmazlaghani_2009_TIP} tackling SAR data's heteroskedastic feature, suggesting a lack of appropriate attention within the research community about such characteristic and its effects.
%However, the community appears to be oblivious of the fact, evidenced by the fact that we could only find very few researcher (Just a comment: it could also imply that the concept is not significantly useful)  \cite{Amirmazlaghani_2009_TIP} tackling heteroskedasticity in SAR data
Under such conditions, Ordinary Least Square methods may not provide optimal results \cite{Woods_PAMI_1984}. 
In fact, as is shown by Eqn. \ref{eqn:Weighted Least-Square Error}), weighted average formulae were being used.
This is consistent with the knowledge that, when variance of heteroskedastic data is available, weighted least squares is an optimal estimator.
However, in SAR data, this variance is not available directly and can only be estimated. 
Since heteroskedasticity means that this variance is dependent on the under-lying signal, thus estimating this variance is evidently as hard as estimating the un-speckled signal itself.
%However, it will also be shown 
% that log transformation would help to break  this circle of ambiguity.

%***IVM: Hai, to be clear, when you mention "the second approach" below, do you mean "In our second approach"??? In which case, please say so!!
In the MAP approach, the filters assume that speckle is not stationary globally but is only stationary locally within the moving processing window.
Such filters operate based on the MAP principle, which is given as:
\begin{equation}
f(x|z) = \frac{f(z|x) f(x)}{f(z)}
\end{equation}
where
	$x$ is the underlying signal that needs to be estimated, and
	$z$ is the observed signal.
The underlying signal is found by
\begin{equation}
x = argmax \{ \log(f(z|x)) + \log(f(x))  \} 
\end{equation}

As evidenced, these classes of filter require the knowledge of the \textit{a-priori} PDF $f(x)$. 
As such, various PDFs for the underlying back-scattering coefficient have been assumed. 

If a normal distribution is assumed, then \cite{Medeiros_1998_IAI}:
\begin{equation}
f(x) = \frac{1}{\sigma_x \sqrt{2 \pi} } e^{\frac{1}{2} \left( \frac{x-\mu}{\sigma_x} \right)^2 }
\end{equation}
and the underlying signal is found by solving the equation
\begin{equation}
x^4 \Gamma^2(N) - x^3 \Gamma^2(N) \mu_x + x^2 \Gamma^2(N) 2N \sigma^2_x - 2 \sigma^2_x z^2 \Gamma^2(N+1/2) = 0
\end{equation}

If a Gamma distribution is assumed, then \cite{Lopes_1990_IGARSS}: 
\begin{equation}
f(x) = \frac{\sigma_x}{\Gamma(\lambda)} (x \sigma_x )^{\lambda-1} e^{-x \sigma_x}
\end{equation}
For $\lambda=1$, the Gamma distribution is identical to the exponential distribution. 
For $\lambda=n/2$ and $\sigma=1/2$ the Gamma distribution is equivalent to the chi-square distribution.
The underlying signal is found by solving the equation:
\begin{equation}
x^3 \Gamma^2(N) \sigma_x + x^2 \Gamma^2(N) (2N - \lambda + 1) - 2 z^2 \Gamma^2(N+1/2) = 0
\end{equation}

If an exponential distribution is assumed, then \cite{Lopes_1990_IGARSS}:
\begin{equation}
 f(x) = \sigma_x e^{-x \sigma_x}
\end{equation}
and the underlying signal is found by solving the equations
\begin{equation}
 x^3 \Gamma^2(N) \sigma_x + x^2 \Gamma^2(N) 2N - 2 z^2 \Gamma^2(N+1/2) = 0
\end{equation}

If a Chi squared distribution is assumed, then \cite{Lopes_1990_IGARSS}:
\begin{equation}
 f(x)=\frac{1}{2^{n/2} \Gamma(n/2)} x^{n/2-1} e^{(-x/2)}
\end{equation}
and the true signal is found by solving the equation
\begin{equation}
 x^3 \Gamma^2(N) + x^2 \Gamma^2(N) (4N-n+2) - 4 z^2 \Gamma^2(N+1/2) = 0
\end{equation}

If a log-normal distribution is assumed, then \cite{Medeiros_2003_IJRS}:
\begin{equation}
 f(x) = \frac{1}{\sigma_x \sqrt{2 \pi}} x^{-1} e^{\frac{-1}{2 \sigma_x} (\ln(x) - \mu_x)^2}
\end{equation}
and
\begin{equation}
 x^2 \Gamma^2(N) (-2N \sigma^2_x - \sigma^2_x -\log(x) + \mu_x) + 2 z^2 \Gamma^2(N+1/2)=0
\end{equation}
%Some of them being: Gaussian PDF, Gamma PDF, chi square, Rayleigh, beta, log-normal ... 

%Clearly should such PDF being estimatable from the data it-self, that would potentially presents an good opportunity for this class of speckle filters.
Other distributions that have been proposed include the heavy tailed Rayleigh distribution \cite{Sun_2006_ICVES}, Weibull distribution \cite{Nezry_1997_IGARSS}, beta distribution \cite{Nezry_1997_IGARSS}, Rician distribution \cite{Lewinski_1983_TAntennaPropagation}. 
Evidently, the performance of these filters are dependent on the underlying distribution of the back-scattering coefficient on the particular surface being imaged. 
In this context, the community appears to focus on using Bayesian inference to choose the most suitable distributions \cite{Walessa_2000_TGRS}.
One of the additive and homoskedastic SAR model's benefits is the possible hypothesis that the underlying distribution of the back-scattering coefficient is estimable discussed in subsequent chapters.
%In our approach, we hypothesize that the underlying distribution of the back scattering coefficient is estimable, at least in the log-transformed domain, as will be described later.

Other approaches that have recently been actively pursued in the research community and in published literature include: 
	%recently there has been an increase publications about the applicability of
\begin{itemize}
\item \textbf{multi-resolution methods:}
 In this approach, speckle filtering is carried out in a scale-space representation.
The homomorphic transformation is normally applied here.
This transformation converts the original multiplicative model into a more familiar additive model.
In this transformed domain, the assumption of linear-time invariant (LTI) system appears to hold.
This simplifies the measurement and filtering of the speckle noise in each sub-band of the scale-space representation.
\begin{itemize} 
\item	\textbf{wavelets:} The wavelet analysis transforms a given signal / image into a new multi-resolution representation.
The original signal / image can then be obtained using a synthesizing scheme symmetrical to that of the analysis \cite{Gagnon_SPIEProc_1997, VidalPantaleoni_2004_IJRS, Moulin_1993_JMathImageVision, Hervet_1998_SPIE, Hebar_2009_TGRS, Chen_2007_IET, Bianchi_TGRS_2008, Argenti_2006_TGRS}, 
\item	\textbf{directional-lets:} Combining Laplacian pyramid decomposition and different schemes of directional filter banks results in the derivation of curvelets \cite{Wang_2007_ICWAPR, Saevarsson_2003_IGARSS, Guo_2008_ICARCV}, contourlets \cite{Foucher_2006_IGARSS} and ridgelets \cite{Saevarsson_2004_IGARSS},
\item	\textbf{autoregressive conditional heteroskedasticity:} 
 In this approach, the wavelet coefficient of SAR images are assumed to follow 2D GARCH models.
GARCH is a term borrowed from the field of econometric. 
It stands for Generalized Auto-Regressive Conditional Heteroskedasticity.
The advantage of this approach is that it takes into account both SAR's heteroskedasticity and the relations among its wavelet coefficients \cite{Amirmazlaghani_2009_DSPWorkshop}. 
\end{itemize}
\item	\textbf{fuzzy logic:} The contribution of each neighboring pixel is weighed using a fuzzy probabilistic function \cite{Cheng_ETCS_2009}, 
\item	\textbf{anisotropic kernels:} Speckle reducing anisotropic diffusion make uses of the local coefficient of variation ($C_v$). 
In that sense, it has the same approach as that of the Frost or Lee filter.
Yet, it has an interesting development.
It shows that local $C_v$ is a function of the local gradient value and the applicable Laplacian operator \cite{DHondt_2006_TGRS},
\item   \textbf{simulated annealing:} Simulated annealing (SA) is an optimization method.
In SAR speckle filtering, it aims to find the global minimum of a multi-dimensional energy function.
There, it is equal to finding the global MAP of a multi-variate distribution.
The local maxima / minima is the common stumbling block of global optimization algorithms.
SA overcomes this problem by introducing a temperature variable to control the optimization process.
At first few iterations, the temperature is kept high allowing a  better chance of accepting configurations that results in an energy increase.
This hopefully allows SA to get out of local minima.
At the end of the minimization process, no further energy increases will be accepted.
This ideally should leads to global minimal results \cite{White_ProcSPIE_1994},
\end{itemize}
%***IVM it might be good to make the above into an un-numbered list, I think it will look nicer!!	

Central to wavelet techniques is the discovery of a consistent and optimal choice for the wavelet coefficients.
Similarly, the development of kernel methods depends strongly on the discovery of consistent predictive kernels.
Evidently the choice of such coefficients are dependent on the underlying statistical characteristics of SAR signals.
This variety of approaches further emphasises the importance of investigating consistent statistical properties of SAR signals.
Not only is statistical analysis important in speckle filtering, the technique is also important for  information extraction \cite{Oliver_1991_JPhysDApplPhys}, target detection \cite{Luttrell_1986_JPhysDApplPhys} and classification \cite{Nyoungui_2002_IntlJRemoteSense}. 

\subsection{Current Methods for POLSAR speckle filtering }

Compared to SAR technologies, most POLSAR techniques are relatively recent. 
Published literature on POLSAR speckle filtering indicates that filtering the covariance matrix off-diagonal elements is still an open problem.
The problem can be traced back to the lack of a statistical model for the covariance matrix in polarimetry.
%This has the consequences that very few methodologies to statistically simulate POLSAR is porposed in literature and validations for such proposals are pretty primatives.
%is the application of such framework is shown to be beneficial academically. 

In the first systems that captured POLSAR data, multi-look processing was used as a method of choice for speckle filtering, at the cost of spatial resolution.
Novak \cite{Novak_1990_TAES} pioneered the concept of speckle reduction, and proposed the polarimetric whitening filter (PWF) to produce a single speckle reduced intensity image. 
The main disadvantages of the PWF filter is that polarimetric information is not well-preserved.
Multi-look processing polarimetric results also exhibit bias and distortion \cite{Lee_2008_TGRS}.

Lee \cite{Lee_1991_TGRS} proposed a new filtering algorithm which does suppress speckle for the diagonal elements, however the off-diagonal elements remain un-filtered.
Later, Touzi et. al. \cite{Touzi_1994_TGRS} drew out the principles of speckle filtering of POLSAR data, which they summarised as:
\begin{enumerate}
	\item Speckle can be filtered if ALL elements of the covariance matrix are filtered.
	\item Single-channel SAR speckle filtering is only applicable to diagonal elements of the covariance matrix.
	\item Speckle filtering needs also to be performed on off-diagonal elements, as correlation among the elements needs to be exploited, and the elements do not exist independently.
\end{enumerate}

In 1999, Lee \cite{Lee_1999_TGRS} extended his refined-Lee filters to POLSAR data, and argued that all covariance matrix elements should be filtered by the same amount to preserve polarimetric properties. 
He then proposed to apply the weighted averaging on the full covariance matrix instead of single-channel intensities values:
\begin{equation}
C_{out} = C_{avg} + w (C_{curr} - C_{avg})
\end{equation}
where
	$C_{out}$ is the filtered covariance matrix output,
	$C_{avg}$ is the average covariance matrix using an edge-aligned window, and
	$C_{curr}$ is the covariance matrix of the current processing pixel.
The weight $w$ is computed within the edge aligned window from the total power image $y$ as:
\begin{equation}
w = \frac{\sigma^2_y - \mu^2_y \sigma^2_v}{\sigma^2_y - \sigma^2_y \sigma^2_n}
\end{equation}
where 
	$\sigma^2_y$ and $\mu_y$ are the variance and mean computed within the edge aligned window, respectively,
	$\sigma^2_n$ is the user specified noise variance. 

Further attempts have been made towards statistically modelling the off-diagonal elements but the application of these results for speckle suppression of off-diagonal element have not been conclusive.
For instance Martinez et. al.\cite{Lopez_2008_TGRS} proposed a model based polarimetric filter.
They first proposed to model $S_hS_v^*$ as having both additive and multiplicative components:
\begin{equation}
S_pS_q^* = \psi |\rho | e^{i \theta} + \psi \bar{z}_n N_c (1-n_m) e^{i \theta} + \psi (n_{ar} + i n_{ai}) 
\end{equation}
where
	$\psi = \sqrt{ E \left( |S_p|^2 \right) E \left( |S_q|^2 \right) }$ represents the average joint power in the two channels, 
	$\rho = |\rho| e^{i \theta} = \frac{E \left( S_pS_q^* \right)}{ \sqrt{ E \left( |S_p|^2 \right) E \left( |S_q|^2 \right) } }$ is the complex correlation coefficient that characterises the correlation among the channels,
	$N_c = \frac{\pi}{4} |\rho| F_{1,2} ( 1/2,1/2,2,|\rho|^2 )$ basically contains the same information as the complex correlation coefficient.
	$F_{1,2} ( a,b,c,d )$ is the Gauss hyper-geometric function,
	$n_m$ is the first speckle noise component that is multiplicative with $E(n_m)=1$ and $var(n_m)=1$,
	$n_{ar} + i n_{ai}$ is the second additive speckle noise component with $E(n_{ar}) = E(n_{ai}) = 0$ and $var(n_{ar}) = var(n_{ai}) = \frac{1}{2} (1-|\rho|^2)^{1.32}$.
The model is overly complicated and to do speckle filtering, various approximations have to be used.

Another different approach is proposed by Lee \cite{Lee_CRCPress_2009}  to model $S_hS_v^*$.
In this method, the phase difference is defined as:
\begin{equation}
\phi = \arg{ \left( S_pS_q^* \right) }
\end{equation}
and the PDF for the phase difference as:
\begin{equation}
pdf(\phi) = \frac{\Gamma(3/2) (1-|\rho|^2) \beta}{2 \sqrt{\pi} \Gamma(1) (1-\beta^2)} + \frac{(1-|\rho|^2)}{2 \pi} F_{1,2}(1,1,1/2,\beta^2)
\end{equation}
where
	$\beta = |\rho| cos(\phi-\theta)$ and
	$F_{1,2} ( a,b,c,d )$ is the Gauss hyper-geometric function.

The normalized magnitude, is defined as:
\begin{equation}
\xi = \frac{E \left( S_pS_q^* \right)}{\sqrt{ E \left( |S_p|^2 \right) E \left( |S_q|^2 \right) }}
\end{equation}
and the PDF is found to be:
\begin{equation}
pdf(\xi) = \frac{4\xi}{\Gamma(1)(1-|\rho|^2)} I_0 \left( \frac{2 |\rho| \xi}{1-|\rho|^2} \right) K_0 \left(  \frac{2 \xi}{1-|\rho|^2} \right)
\end{equation}
	$I_0$ and $K_0$ are modified Bessel functions.

Again the equations looks overwhelmingly complex, and as admitted by the authors themselves:
	this does not allow for easy speckle filtering as the model is too complex and 
	the complex correlation coefficient $\rho$ is not easily estimated \cite{Lee_CRCPress_2009}

There have been several other attempts to filter POLSAR data.
Vasile, proposed a region growing adaptive neighbourhood approach in 2006, where averaging is performed on an adaptive homogeneous neighbour.
The main drawback of the approach is that only the intensity was used to determine homogeneity and a primitive averaging filter is performed on off-diagonal elements \cite{Vasile_TGRS_2006}.
%J.S. Lee is also very active in POLSAR speckle filtering.
In \cite{Lee_2006_TGRS}, J.S. Lee proposed to group homogeneous pixels based on their underlying scattering model. 
While the work is interesting, it is believed that pixels having similar underlying models, for example volume scattering, can still be of a very different nature.
Similar to Lee's previous work, a simple weighted average filtering is applied on both diagonal and off-diagonal elements.
Vasile, in 2010, published a complicated spherically invariant random vector estimation scheme to estimate the underlying target POLSAR signature \cite{Vasile_2010_TGRS}.
While polarimetric information is estimated independently from the power span, its output results no longer follow established Wishart distribution.

%In summary, it is evident that further research into designing speckle filters for POLSAR data, especially for off-diagonal elements is warranted. 
%We will discuss our approach in the next section.
In summary, while off-diagonal elements, which carry phase information, are known to be important in interpreting POLSAR data \cite{Rodionova_2009_ESASP}, their estimation and filtering is under-developed in the current research literature.
%We will discuss our approach ??) based on polarization basis transformation in the next section. (comment: This chapter is about background. So your �approach� should not be included in this chapter. From our today conversation, I now know that the various sections are �assembled� from the various conference papers you published. But you will need to re-arrange the various parts into the appropriate Chapter for the Thesis)   

%\subsection{ Simulating Compact Polarimetry data and the reconstruction of POLSAR signature }

%\subsection{ Target Detection and Classification using SAR-based images }

\subsection{Current Techniques to Evaluate SAR Speckle Filters}

Because of the variaties of  speckle filtering available, it is hence crucial to have a framework for 
	(a) modelling and simulation, which provide the ground-truth data in different scenarios of underlying distribution, and
	(b) analysing the inter-dependency between noise and the underlying signal, as well as the dependency of each filter's performance on the different underlying distribution scenarios and
	(c) establishing consistent criteria to evaluate speckle suppression power as well as the adaptive capabilities in reconstructing the underlying signal.

This section reviews the published literature for the different techniques used to evaluate SAR speckle filters.
%In general, any metric to evaluate speckle filters should be relevant to the normal usage of such filters.
%Furthermore, the application of any speckle filter in a SAR processing framework
%   should enable an improvement in the measurement, 
%	detection or classification of the underlying radiometric features.
% (??? I suggest move these sections to Chapter 3 or 4, if they are really in this Chapter), 
%.
% (The above is your proposal. It should not appear here in Chpater 2)
The overall requirement of speckle filtering can be broken down into two smaller requirements.
On the one hand, speckle filters should preserve the underlying radiometric signal (namely the radiometric 
	preservation requirement).
On the other hand, they should reduce the variation of the additive noise (i.e. speckle suppression power).
One metric that can be used to detect radiometric distortion is the ratio between the estimated and the 
	original value $r = I_{est} / I_{org}$ \cite{Oliver_2004_SciTech,Medeiros_2003_IJRS}.
A somewhat similar metric is used in \cite{Wang_2005_MIPR} as $r_w = (I_{est} - I_{org} )/ I_{org}$.
%For homogeneous scenes, Shi et al. \cite{Shi_IGARSS_1994} found that 
%	in the original domain the ``standard'' filters (boxcar, Lee \cite{Lee_PAMI_1980}, 
%	Kuan \cite{Kuan_1985_PAMI}, Frost\cite{Frost_PAMI_1982}, 
%	MAP\cite{Lopes_IGARSS_1990}) and the enhanced version \cite{Lopes_TGRS_1990} can achieve negligible bias. 
%In this paper, %****IVM you need to search for the word "paper" in case you have more of these
Several metrics have also been developed to evaluate speckle suppression power.
The most common measure are: the Equivalent Number of Looks (ENL) index 
$ENL=avg(I)^2/var(I)$ \cite{Lee_1981_CGIP},
and the Coefficient of Variation index $C_v = \frac{\sqrt{var(I)}}{avg(I)}$ \cite{Touzi_2002_TGRS}.
Another very similar metric is the ratio of mean to standard deviation, $R=avg(I)/std(I)$ \cite{Gagnon_SPIEProc_1997} 
%****IVM: the section reference below is not found (is displayed as ??)

It is customary to divide the performance evaluation of speckle filters into two distinct classes of 
	homogeneous and heterogeneous region evaluation.
Across homogeneous areas, speckle filters are expected to be estimated with negligible radiometric bias.
As such, evaluating speckle filters over homogeneous areas has traditionally focused on evaluating the variation of the estimators'
	output (a.k.a the speckle suppression power).
%In contrast, the methodologies used to evaluate speckle filters over heterogeneous areas are much more complicated, 
%	due in part to the following difficulties.
%The first obstacle is to determine an absolute ground-truth against which quantitative criteria can be measured.
%The second challenge is to then define a quantifiable metric that allows the performance of different speckle filters 
%	to be measured and compared.
        
Real-life and practical images, however, are not homogeneous.
Thus there are a number of associated difficulties in evaluating speckle filters over heterogeneous scenes.
The first difficulty in evaluating speckle filters for heterogeneous scenes is to select the basis for comparison. 
It is trivially easy to estimate the underlying radiometric coefficient if an area is known to be homogeneous.
However, without simulation or access to solid ground-truth, it is practically impossible to do so for 
	real-life images.
And hence, the need for speckle filter estimation is warranted.

Without ground truth, one way to evaluate radiometric preservation of filters is to compute the ratio image mentioned above. 
Under a multiplicative model, the ratio image is expected to comprise solely of
    the noise being removed (i.e. it should be completely random). 
Being random, this should display as little ``visible'' structure as possible. 
However, when displaying such images for visual evaluation, the ratios that are smaller than unity are much 
harder to distinguish than those that are bigger \cite{Medeiros_2003_IJRS}. 


%There are two main ways that ground truths are used in evaluating speckle filters.
%A simple method is to embed speckle noise into an existing optical image, and then use the filters to estimate noise-free imagary from this. 
%Normally the image is large, with a number of different features and test conducted in once. 
%The benefit of this method is that a wide range of features can be tested, which is probably closer to the real-life situation. 
%The main drawback is that, since the noise embedding process is stochastic in nature, reporting only a single experimental is probably not providing a very representative result. 
%It is of course, also dependent upon the nature of the test image and noise embedded process employed.
%
%Another method is to evaluate using a patterned, structured and repeated ground truth which may be artificial or real.
%Since the structure is repeated many times, the combined results become more representative. Also, the patterns can be pre-designed and lead to the possibility of repeatable evaluations between research groups.
%The drawback is that only a single type of target can be tested per image. These types may normally be point targets, edges and lines (all features that are currently used to evaluate speckle filters).

Even with known ground-truth, evaluating metrics need to be defined for quantitative measurements.
This is the second difficulty in evaluating speckle filters for heterogeneous scenes.
Under the conditions of heterogeneity, the standard speckle filters still introduce radiometric loss, 
	normally at local or regional levels.
In fact, the common consensus is that a powerful speckle suppression filter (for example the boxcar filter) 
	is likely to perform poorly in terms of preserving underlying radiometric differences 
	(e.g. causing excessive blurring).
Furthermore, we have not found many published articles combining the evaluation
  of these two seemingly contradicting requirements.


%Similar to the ENL index in homogeneous area, in establishing an evaluation metric, 
%	it is desirable that the criteria should also be scene-independent\cite{Shi_IGARSS_1994}.
%The most common measure of difference in image processing methods is to compute some substractive distances, either in intensity or in amplitude domain. 
%However since SAR noise is multiplicative in the original domain, these differences are not only dependent on the noise level but also on the amplitude level of the signal itself.
%Shi \cite{Shi_IGARSS_1994} define sharpness as the ratio between
%Shi define the mid-point position of an edge as the intensity value between the nominal values of the two regions, $M= { E(X_t) + E(X_b)}/{2}$
%	and the sharpness of an edge as the slope between the two nominal intensities value, $S=E(X_t)-E(X_b)$, 
%	where $E()$ denotes expectation operator. 

Overall, many different methods and metrics have been proposed to evaluate various aspects of speckle filters. 
However it is clearly advantageous to have a single metric that is able to judge if one filter is better than another. 
Wang \cite{Wang_2005_MIPR} proposed using fuzzy membership to weight opinions of an expert panel. 
Although this provides a potential solution, we consider it to be tedious in implementation, fuzzy in concept and 
	subjective in nature.

Another approach is to apply a universal mean squared error criteria into the context of SAR data. 
However since SAR data is heteroskedastic, which violates the assumptions of the Gauss-Markov theorem, 
	the use of MSE is not straightforward. 
Thus Gagnon \cite{Gagnon_SPIEProc_1997} suggested the use of
  the ratio between the expected mean of the signal and the RMSE of the removed noise. 
The metric is argued to be similar in interpretation to the standard signal-to-noise ratio (SNR). 
Others have suggested the use of normalised MSE, which is essentially the ratio between MSE and the expected mean.

In general, any metric to evaluate speckle filters should be relevant to the normal usage of such filters.
Furthermore, the application of any speckle filter in a SAR processing framework
   should enable an improvement in the measurement, 
	detection or classification of the underlying radiometric features.
Thus, another way to evaluate speckle filters is by comparing the feature preservation characteristics 
of the original noised data and those of the filtered image.
This can be applied not only whether or not the ground-truth is available precisely.
When there is no ground-truth given, the feature are estimated in both the original noised and the filtered images.
Evaluation would then determine how closely related the two feature maps are. 

Various methods may be applied to extract features; examples of those that have been used are the 
	Hough Transform \cite{Medeiros_2003_IJRS}, Robert gradient edge detector \cite{Gagnon_SPIEProc_1997} 
	and edge map \cite{Frost_PAMI_1982} or edge-correlation index \cite{Sattar_TIP_1997}.
While significant effort has been spent on these evaluations, serious doubts remain concerning the precision 
of these methodologies.
This is because feature extraction algorithms are only approximations,
	whose accuracy is not only dependent upon the characteristics of the original image 
		but also heavily affected by 
		the inherent noise.
Unfortunately, speckle filters invariably alter the noise characteristics.
Thus, without a clear understanding of these dependencies and no absolute ground truth, 
	using feature extraction algorithms to evaluate speckle filters would leave serious questions on 
	how to interpret the results, especially in terms of accuracy.

There are also a variety of other approaches, which includes
\begin{itemize}
  \item The Target-To-Clutter ratio \cite{Cetin_ProcSPIE_2000}, whose applicability is limited to the special case of point-target preservation.
  \item The Structural-Change-Index \cite{Aiazzi_ProcSPIE_2011_81790D}, whose authors also admits that it is unable to measure the overall effectiveness of SAR speckle filtering \cite{Argenti_GRSM_2013}.
  \item The Mean Structural Similarity Index Measurement (MSSIM) \cite{Wang_TIP_2004}, where the proposed index is tuned for hypothesized overall functionalities of the Human Visual System (HVS) in a top-down manner.
%\item the B-index \cite{}, where the proposed metric only focuses on one single evaluation of bias.
\end{itemize}  
        
As reviewed above, the problem of establishing an single evaluation criteria to evaluate the overall performance of speckle filters remains open for improvement.

%Thus we propose the use of MSE in the homoskedastic log-transformed domain to evaluate eps. 
%This is similar to the investigation of residual image in log-transformed domain and also agrees with the log-transformed variance metric in homogeneous area (assuming that the filters have no problem preserving the mean). We also show experimentally that this MSE measure is inversely correlated with the AUC index mentioned earlier, suggesting that the lower the MSE index a filter can achieve, the better its feature preservation capability can be.

%In this paper, 
%	Section \ref{sec:log_transform} provides a brief discussion on logarithmic transformation for SAR images.
%****IVM:  Hai, please be consistent in using "Section" or "section". My advice would be to use the capitalised form whenever you talk about a section number, and the lower case form for discussing a section. For example: "look at Section 1.4 or consider the previous section". This instance is fine, but I noticed a few above that might need to be changed.

%\subsection{Existing Discrimination Measures for (POL)SAR data}
%
%(comment: you need to have a lead-in paragraph to link this section from the other)
%To detect if the two scaled multi-look POLSAR covariance matrix $Z_x$ and $Z_y$,
%  which have $L_x$ and $L_y$ as the corresponding number of looks,
%  come from the same underlying stochastic process,
%Conradsen \cite{Conradsen_2003_TGRS_4} considered the following POLSAR statistics
%\begin{equation*}
%  Q = \frac{(L_x+L_y)^{d \cdot (L_x+L_y)}}{L_x^{d \cdot L_x} L_y^{d \cdot L_y}} \frac{|Z_x|^{L_x} |Z_y|^{L_y} }{|Z_x+Z_y|^{(L_x+L_y)}}
%\end{equation*}
%
%Taking the log-transformation of the above statistics, and noting that $C_{vx} = Z_x / L_x$, $C_{vy} = Z_y / L_y$ and $C_{vxy} = (Z_x + Z_y)/(L_x + L_y)$ then we have;
%\begin{eqnarray*}
%  Q &=& \frac{|C_{vx}|^{L_x} \cdot |C_{vy}|^{L_y} }{|C_{vxy}|^{L_x + L_y}} \nonumber \\
%  \ln Q &=& L_x \ln |C_{vx}| + L_y \ln |C_{vy}| - (L_x + L_y) \ln |C_{vxy}| \nonumber
%\end{eqnarray*}
%
%Since both $Z_x$ and $Z_y$ follow complex Wishart distributions with $L_x$ and $L_y$ degrees of freedom,
%  $Z_x+Z_y$ also follows the complex Wishart distribution with $L_x + L_y$ degrees of freedom.
%In view of the models denoted later in Eqn. \ref{eqn:log_determinant_distribution},   
%%***IVM: Hai, can you please check this equation reference (it comes out as eqn. 4.9 which I don't have in this document)
%  it is evident that not only the bound for $\ln Q$, or equivalently $Q$, can be derived
%  but the whole statistical distribution for it can be simulated as well:
%\begin{eqnarray*}
%%  Q &\sim& \frac{(\chi^d_{L_x})^{L_x} \cdot (\chi^d_{L_y})^{L_y} \cdot (2(L_x+L_y))^{d (L_x + L_y)}}{(2 L_x)^{d \cdot L_x} \cdot (2 L_y)^{d \cdot L_y} (\chi^d_{L_x + L_y})^{L_x + L_y}} \\
%%    &=& \frac{(L_x+L_y)^{d (L_x + L_y)}}{(L_x)^{d \cdot L_x} \cdot (L_y)^{d \cdot L_y} } \frac{(\chi^d_{L_x})^{L_x} \cdot (\chi^d_{L_y})^{L_y}}{(\chi^d_{L_x + L_y})^{L_x + L_y}} \\
%  \ln{Q} &\sim&  q + L_x \Lambda^d_{L_x} + L_y \Lambda^d_{L_y} - (L_x + L_y) \Lambda^d_{(L_x + L_y)} \\
%  Q &\sim& e^q \frac{(\chi^d_{L_x})^{L_x} \cdot (\chi^d_{L_y})^{L_y}}{(\chi^d_{L_x + L_y})^{L_x + L_y}}  
%\end{eqnarray*}
%where $q = d \left[ (L_x + L_y) \ln(L_x + L_y) - L_x \ln{L_x} - L_y \ln{L_y} \right]$.
%
%In the special case of $L_x = L_y$, the Conradsen statistics become
%$\ln Q = \ln |C_{vx}| + \ln |C_{vy}| - 2 |\ln C_{vxy}|$.
%%Our approach suggest a slightly simpler statistics which probably will work just as well: the contrast
%The contrast model which is written as $\mathbb{C} = \ln C_{vx} - \ln C_{vy}$ and presented above should, under exactly the same assumptions, provides another consistent and equivalent approach,
%  with probably simpler conceptual model and computational derivations. 
%%a slightly simpler statistics.
%The detailed application of this for the task of edge-detection however is outside the scope of this thesis.%paper.  (Commet: I suggest this section to be removed since it would not be referred to in the remainig of the thesis)
%
%%\subsection{The need for Scalar Representative and Homoskedastic Models}
%%	We will show our originality and significance here!
%
%%This section surveys recent developments in the research landscape. 
%%First, the stochastic nature of SAR speckle phenomena is explained. 
%%Then, the topic of SAR speckle is reviewed from the perspective of statistical estimation theory. 
%%An opportunity for contribution in proposing a stochastic simulation and evaluation framework is identified. 
%%Next, the applications of such a framework into the problem of POLSAR speckle filtering is described. 
%%Recently, the practicality of Partial Polarimetry has been noticed and studied. 
%%The need for reconstruction of Full Polarimetric signatures is highlighted and the possible application of the proposed framework is described. 
%%Last but not least will be an illustration on the applicability of such a framework to the topic of target detection and classification.
%
%%\subsection{POLSAR speckle filter}
%        
%	%\section{The Need for Scalar Consistent Measures of Distance}
%
%	%\section{Research Framework: Originality and Significance}

\section{Summary }

This chapter has
  not only described the multi-dimensional and stochastic nature of the POLSAR data, but
  also illustrates the different published approaches to address the grand challenge of modelling, processing and understanding this type of data.   
Specifically it shows that
while the statistical model for SAR is quite well understood,
the understanding of the complex POLSAR statistical model trails behind.
The complex POLSAR data have been statistically modelled as following the complex Wishart distribution, which apparently is multi-dimensional, complex and thus un-intuitive.
There have been a few published scalar statistical models for POLSAR,
but none of them is highly representative of the multi-dimensional data.
None of them have so far led to published discrimination measures. 
%but none of them fit the dual criteria of being highly representative of the data and at the same time lead to scalar discrimination measures.
A few POLSAR discrimination measures have also been proposed, but all of them are based on likelihood test statistics, which so far have only shown to be based on an asymptotic distribution.

The chapters also illustrated many different ways to combat  the negative impact caused by multiplicative and heteroskedastic properties on the statistical estimation process,
  which includes statistical modelling of the data, estimator development and the estimator performance evaluation.
Specifically, it showed that when the sense of subtractive distance is not consistent for SAR data, the ratio is proposed as its discrimination measure \cite{Rignot_1993_TGRS_896}.
It also showed that while the MMSE criteria is not very suitable to evaluate SAR speckle filters, other criteria such as radiometric preservation and speckle suppression power evaluation are employed.
Unfortunately, these counter-intuitive measures are fragmented and thus not very popular outside the SAR community.

In the next chapter, theoretical models are developed to form the foundation required to systematically tackle these issues. 







