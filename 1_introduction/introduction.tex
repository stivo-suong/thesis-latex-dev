%: ----------------------- introduction file header -----------------------
\chapter{Introduction}

%What is the situation, the background context of your research? the BIG problem!
Synthetic Aperture Radar (SAR) is a technology with many advantages.
Compared to optical remote sensors - which are passive, SAR uses active sensors to provide weather independent and night-inclusive operational capabilities. 
Compared to other radar based active sensors,
for example Real Aperture Radar (RAR), SAR provides state-of-the-art resolution
and coverage.
In the past decades, the exponential growth in computational power has made the once overly excessive computationally demanding SAR technology to become a feasible and preferred earth observation solution \cite{Cumming_2005_Artech}.
Thus SAR has now become the preferred choice for continuous and autonomous large-scale
remote surveillance solutions.

As state-of-the-art technology, the SAR technique has been extended in a few directions, one of which is Polarimetric SAR (POLSAR).
POLSAR is the natural extension of SAR, exploiting the natural polarization property of Electromagnetic (EM) waves.
It extends the SAR acquisition information from the traditional single-channel SAR to multi-channel POLSAR data, corresponding to a different combination of transmitted and received polarizations. Similar to the extension from black-and-white images to colour photography, the polarimetric extension brought about
multi-channel POLSAR data as compared to the traditional one-channel SAR data.

With both SAR and POLSAR data becoming cheaper and more readily available, the recent research emphasis is on understanding and developing applications using these data. However, both SAR 
and POLSAR suffer from speckle phenomena, which hampers the human capabilities
to understand SAR images. Speckle arises due to the random interference of many
de-phased but coherent backscattered waves. Thus both SAR and POLSAR data are
stochastic by nature.

\section{Research Motivation}

%Given the BIG problem, Why it is important to solve this problem?
The
                big picture, as introduced above, is that POLSAR data is multi-dimensional,
                stochastic, multiplicative and heteroskedastic.
%Under
%                this context, the motivation is to gain a better
%                understanding of the data,
%which
%                hopefully will lead to better information that can be
%                extracted from these data.
Under this condition, it is very important to understand the statistical models for its
data. It is these models that form the foundation for different (POL)SAR data processing
techniques, for example speckle filtering, target detection, image segmentation, and
other clustering, classification techniques \cite{Lopez-Martinez_2003_TGRS_2232, Argenti_GRSM_2013, Conradsen_2003_TGRS_4}.
In fact, due to the random nature of speckle, all of these techniques are essentially based on statistical estimation theory,
which comprises statistical modelling and validation, estimator design and development
as well as evaluation of different estimators' accuracy. And central to this statistical
estimation theory is the need for statistically consistent discrimination measures which
are commonly derived from the statistical models.

That is, while different statistical models have been developed for different POLSAR
observables, for a statistical model to be useful, scalar discrimination measures need to
be derived from it. At the same time, the extension from SAR to POLSAR brings about
a representation difficulty in that: there exists not one, like the intensity in SAR, but
many observable quantities in POLSAR. Thus, on the one hand, practical applications for
POLSAR data processing require measures which are scalar, consistent and preferably
homoskedastic. On the other hand, the observable quantity being modelled needs to
be naturally representative of the high dimensional POLSAR data.

%Solving these dual-problem is both important and interesting.
%The statistical models are undoubtedly important in understanding the stochastic nature of the data. 
%At the same time, the statistical properties of the input data set greatly affect the choice and the performance of information extraction and statistical estimation techniques.
In addition, most of these data processing techniques are traditionally designed for additive and homoskedastic data \cite{Dutch_LNCS_2004}.
In fact, unless proven otherwise, it is both convenient and acceptable to assume that the additive white noise model is appropriate for any given set of data samples.
This is due to a number of reasons, for example the prevalence of the normal distribution which results from the central limit theorem, or the prevalence of the additive noise model which probably arises due to the linear and additive nature of digital data. % dear Ian I changed this back 
%of the Additive White Gaussian Noise (AWGN) model in natural processes.
%of the additive noise model which probably arises due to the linear and additive nature of digital data.%****IVM I'm not sure about this - you seem to assume (in this sentence) that digital data is linear and additive. However this is definitely not true - your POLSAR data is digital and yet is not additive. Maybe you can say "or the prevalence of AWGN in the majority of sensing or sampling systems, and electronic devices."
%OR PERHAPS THIS: "... the prevalence of AWGN in natural processes."
%****IVM2: I'm still not sure about it, but we can put it to the examiner, see if he picks it up.
It should be noted that even for the case of SAR data which is digital and not additive,
  the above statement is still true as SAR actually has two components, the real and imaginary parts,
  both of which follow the standard Additive White Gaussian Noise model.
%****IVM: I really don't understand what you're trying to say here.  Are you saying that "both parts are corrupted by AWGN"?

Both SAR and POLSAR data however are multiplicative and heteroskedastic by nature.
While there is no doubt that a number of data processing techniques have been proposed to handle certain aspects of the data, the breadth and depth of these studies trails in comparison to that of the more familiar additive and homoskedastic model.
This leads to the fact that a lot of times, existing statistical or computational techniques initially designed for additive and homoskedastic models are applied on multiplicative and heteroskedastic data, in ignorance of the incompatibility. 
An example is the use of mean, variance and squared error criteria, which are linear by nature and work excellently on additive and homoskedastic models, being applied on multiplicative and heteroskedastic distribution such as the gamma distribution.
Such use, however, is known to be not very robust for these so-called heavy tailed distributions \cite{McElroy_Econometric_2002}.

Similarly, heteroskedasticity poses numerous negative impacts on the usual framework of statistical estimation, which most of the time makes use of various different techniques and criteria known to work very well on the usual models.
For example, in modelling, it is very common to use subtractive differences to discriminate and to make decisional inferences. However, for heteroskedastic SAR, it is known that such use should be avoided in preference to a ratio-based discrimination measure \cite{Rignot_1993_TGRS_896}.
Another example is in the normal performance evaluation stage of statistical estimators, where the Ordinary Least Square (OLS) is widely used as the best evaluation criterion which is probably due to the Gauss Markov theorem \cite{Furno_1991_JStatCompSimul}.
For SAR data, however, its heteroskedasticity directly violates the homoskedastic assumption of the theorem and thus many different ways to evaluate SAR speckle filters were proposed \cite{Gagnon_SPIEProc_1997, Argenti_GRSM_2013}.

Furthermore, since POLSAR can be considered as the multi-dimensional extension of
traditional SAR, a common problem exists for both SAR and POLSAR data. That
is: the sense of subtractive distance is seriously flawed in the original multiplicative
and heteroskedastic domain of both types of data. 
In contrast, the digital technologies which range from image processing and display to artificial intelligence and machine learning algorithms used to process (POL)SAR data are linear
and additive in nature. 
This linear and additive nature is manifested, for example, in a
large number of algorithms in the field of computational intelligence which make use of
MSE as an objective function, as well as the extensive use of variance and
contrast in digital image processing. 
Thus it will be very useful to develop homoskedastic models for this heteroskedastic data
  so that the benefits of an additive and homoskedastic statistical estimation framework can be explored.

%Specifically, 
%  this thesis focuses on providing a solution for the
%                dual-problem of: 1) developing scalar and
%                representative statistical models for the
%                multi-dimensional and inter-correlated POLSAR data and
%                2) establishing at consistent measures of distance for both
%                SAR and POLSAR data which allow the benefits of the
%                homoskedastic statistical estimation framework to be
%                demonstrated.


\section{Research Objective}

%The research objectives 
%helps to 1) define and focus your effort, 2) identify variables to be measured 3) identify steps to be performed 4) establish the limits of the study 5) avoid any data collection, argument development that is not necessarily needed.
%Should be stated in action verbs specific enough to be measurable. Avoid vague, non-active verbs which is hard to evaluate if the objectives have been achieved.
%4. Research methodologies (How the research is/was/will be conducted)
%5. Result interpretation and evaluation (describe how the results is expected to look like, how they are going to be evaluated

%How do others solve this problem?
%Briefly What are their solution? what problems they cannot yet solve?
While
                the statistical model for SAR is quite well understood,
the
                understanding of the POLSAR statistical model trails behind.
The
                complex POLSAR data have been statistically modelled as
                following the complex Wishart distribution, which
                apparently is multi-dimensional, complex and
                un-intuitive.
There
                have been a few published scalar statistical models for
                POLSAR, \cite{Alberga_2008_IJRS_4129, Joughin_1994_TGRS_562, Lee_1994_TGRS_1017, Touzi_1996_TGRS_519, Barakat_1985_IJOptics, Eliyahu_1993_PhysRevE_2881, Brosseau_1995_AppliedOptics_4788} but none of them fit the dual criteria of being
                highly representative of the data and at the same time
                lead to scalar discrimination measures.
There
                are also a few POLSAR discrimination measures proposed \cite{Lee_1994_IJRS_2299, Kersten_2005_TGRS_519, Lee_2011_IGARSS_3740, Cao_2007_TGRS_3454, Conradsen_2003_TGRS_4},
                but all of them are based on the likelihood test
                statistics, which so far have only shown to be based on an
                asymptotic distribution.

 In
                another thread of study, to combat the negative
                impacts caused by multiplicative and heteroskedastic
                properties for SAR data, experienced researchers in the
                field have developed numerous ingenious ways to tackle
                these issues.
For
                example, when the sense of subtractive distance is not
                consistent for SAR data, the ratio is proposed as its
                discrimination measure \cite{Rignot_1993_TGRS_896}.
Or
                when the MMSE criterion is not very suitable to evaluate
                SAR speckle filters, other criteria likes radiometric
                preservation and speckle suppression power evaluation
                are employed. 
Unfortunately,
                these counter-intuitive measures are not very popular
                outside the SAR community, and newcomers to the field
                are frequently found being trapped by these pitfalls.

%In what way is your solution different ? What is the novelty in your approach and hence solution?
Some of the problems described above have been studied using different approaches in the literature.
However, they have not been presented in a systematic manner nor with precise closed-form mathematical models.
%precise problem statement
Hence the \textbf{main problem statement} of this thesis arises. %it is important to emphasize again the main problem statement of this thesis.
That is,
                this thesis focuses on providing a solution for the
                dual-problem of: 1) establishing scalar and
                representative statistical models for the
                multi-dimensional and inter-correlated POLSAR data and
                2) providing consistent measures of distance for both
                SAR and POLSAR data which allow the benefits of the
                homoskedastic statistical estimation framework to be
                demonstrated.

\section{Research Methodology}
%To achieve the stated objectives, what is your plan of attack? Describe How your actions are to be conducted? How your argument is developed?                
%****IVM Hai, I removed this sentence. I don't think it adds anything useful...
%The bottom up development of the research effort for this long-term study, admittedly is not as straightforward as the top down layout of the stated objectives.
The
                overall research methodology is to first focus on the
                simpler one dimensional SAR domain, and then extend the
                obtained insights and processing techniques 
towards the more complex multi dimensional
                POLSAR data domain.
%***IVM   what OPEN LOOP??? I'm not sure that this is an open loop!! Can't you just say "This is achieved when....."
This
                is achieved as the theoretical models for
                POLSAR are shown to be the multi dimensional extensions of
                the one dimensional SAR model.

The
                development of these models are based on
                rigorous statistical mathematics transformations. % and theory. 
The
                assumptions used are minimal and also widely
                acknowledged for this type of data.
Then, the variable change theorem is applied to derive the statistical model for these transformed observables. %log-transformed SAR data.
The
                derived theoretical models are validated against
                real life data.
The
                imperfect conditions found in practice (in
                comparison to the theoretical assumptions) are also
                investigated, and the models are shown to be robust, capable of
                handling these imperfections.
The
                theoretical properties of these models, as well as many
                of their benefits, are also proven using rigorous
                mathematical principles.

%The theoretical statistical models are derived from the commonly used statistical models for SAR and POLSAR.
%Then, as mentioned above, the variable change theorem is applied to derive the statistical model for log-transformed SAR data.
%A similar methodology is then followed for the multi-dimensional POLSAR data.
%The POLSAR theoretical foundation used is the Complex Wishart distribution applicable on the covariance matrix.
%In this thesis, the statistical model for the determinant of this covariance matrix,
%  is derived from this foundation.
%The variable change theorem is again applied to derive the discrimination measures from these derived basic models.
  
As both the SAR and POLSAR data are stochastic in nature, many of the associated data processing techniques, such as speckle filtering have been put into the framework of statistical estimation.
For such application, the research methodology is based extensively on the statistical estimation framework. 
The framework consists of three stages:

%***IVM Hai, I changed your list type to make it look better
\begin{description}
\item{Stage 1:} data statistical analysis. 
We propose a preferred data transformation and decomposition that may allow applications of existing computer science techniques. 
The statistical models built after such a transformation are validated against real-life data.
\item{Stage 2:} predictor and estimator development.
Various statistics-based computational intelligence and/or signal processing methods are to be applied. 
\item{Stage 3:} results analysis and performance evaluation.
Experiment results are evaluated both qualitatively on real images and quantitatively through simulated data. 
\end{description}

\subsection{Statistical Estimation Framework for (POL)SAR data}

Even though the research topic seems to spread across different sub-fields, the common framework that binds them all together is the application and development of statistical estimators.
For example, for speckle filtering, it is the estimation of underlying back-scattering coefficients being corrupted with stochastic noise.
For the problem of target detection or classification,
  it is the estimation of the likelihood that the target belongs to a certain category.

Thus a significant portion of this research effort focuses on developing a computational framework for simulation, evaluation, processing and classification of target polarimetric signatures.
The benefits of such a framework are manyfold.
Firstly, simulation allows one to quickly carry out experiments on small synthesized and well-designed scenarios, without the need for highly expensive satellite imagery.
Secondly, simulation provides ground-truth, without which quantitative evaluation of estimators would have been impossible.
Last but certainly not least, such a framework allow one to apply estimators repeatedly, and results from different estimators can be statistically and reliably compared. For example, this would allow different authors to compare results quantitatively.

The objective of such framework being:
\begin{enumerate}
\item To build, develop and validate statistical models for different stochastic processes that are inherent within the polarimetric SAR domain.
\item To allow easy, fast and repeatable simulations and experiments of various stochastic processes in designed, focused, as well as broad-based scenarios.
\item To allow easy application of different existing estimators as well as various transformations into focused and simulated scenarios as well as into practical real measured data.
\item To allow development and calculation of good performance indices for estimators. 
\item To allow visualization and analysis of the performance of different stochastic estimators.
\item Ultimately, to allow development of better and more accurate estimators, whose results can be verified: not only qualitatively on real life images, but also quantitatively through rigorous simulations.
\end{enumerate}

The possible benefits of this computational framework are:
\begin{enumerate}
\item Practical benefits
\begin{enumerate}
\item The computational frameworks allow for faster, quicker and more focused experiments. 
Often, a statistical hypothesis can be observed or rejected by designing and implementing an appropriate computational experiment.
\item The framework allows for easier analysis, and thus more insight to be gained, on the performance of an estimator. 
This, in turn, allows for design and development of better estimators.
\end{enumerate}
\item Academic benefits
\begin{enumerate}
\item The computational framework allows for quantitative evaluation and comparison of estimators.
With fast and repeated experiments, the framework allows one to not only compute a single performance value on a single experiment, but also to analyze the whole behaviour of such performance measures over repeated stochastic simulations.
\item The computational framework provides significant help in deriving new performance evaluation criteria for statistical estimators.
\end{enumerate}
\end{enumerate}

\subsection{Data Collection}

This work has made extensive use of simulated data, which have a number of advantages listed below:
\begin{enumerate}
\item Once the model is validated against real-life data, simulated experiments can be carried out on more focused scenarios.
This allows smaller data sets to be generated and analysis can become faster and more accurate.
\item Contrary to real-life images where underlying coefficients are to be estimated, ground truth is readily available in simulated experiments.
\item Quantitative evaluation is possible not only via a single experiment on a single real-image but via repeated stochastic runs.
  Thus not only single values of mean or mean-squared-error are to be evaluated. 
As the whole response PDF is available, statistical bias or heteroskedasticity can also be reported.
\end{enumerate}

Besides simulated  data, we have gathered a large amount of real-life data.
Table \ref{tbl:collected_data} lists data made available to us, courtesy of EADS Innovation Works Singapore. 

\begin{table}
\centering
\begin{tabular}{|c|c|c|}
\hline
Radar Platform & Polarization Mode & Data Capture Time \\
\hline
RadarSat 2 & Full Quad Pol & 29 Apr 2009 \\
RadarSat 2 & Full Quad Pol & 05 Apr 2009 \\
RadarSat 2 & Full Quad Pol & 12 Mar 2009 \\
RadarSat 2 & Full Quad Pol & 23 Jan 2009 \\
RadarSat 2 & Full Quad Pol & 30 Dec 2008 \\
RadarSat 2 & Full Quad Pol & 06 Dec 2008 \\
RadarSat 2 & Full Quad Pol & 19 Oct 2008 \\
RadarSat 2 & Full Quad Pol & 12 Nov 2008 \\
RadarSat 2 & Full Quad Pol & 25 Sep 2008 \\
TerraSar-X & Dual Pol (HH-VV) & 18 Apr 2008 \\
NASA-JPL AirSar & Full Quad Pol & 2001 \\
\hline
\end{tabular}
\caption{Collected imagery data}
\label{tbl:collected_data}
\end{table}

\subsection{Result Interpretation and Evaluation}

%How do you evaluate your solution? 
%The
%                achievement of the research objectives stated above not only provides
%                novel solutions to the stated problem, it also
%                illustrates the advantages and beneficial implications
%                of the proposed theory.  
Compared
                to existing published techniques, the advantage of the
                proposed models is qualitative, which does not require
                direct quantitative comparisons.
Instead
                a straightforward binary evaluation would be sufficient.
For
                the POLSAR scalar model, its advantages are that not
                only does it lead to consistent discrimination measures, it
                also generalizes the traditional model for one
                dimensional SAR intensity.
For
                the homoskedastic models, their advantage is that they
                enable an additive and homoskedastic statistical
                framework, which is not only capable of overcoming various
                negative impacts of heteroskedasticity, but also results
                in a single better evaluation criteria that can combine
                and represent many different existing performance
                criteria.
                
To validate the theoretical models,  real-life data are used in this work.
Specifically, estimated theoretical PDFs are plotted against the histogram of real data,
  and a good visual match indicates a good validation.
In cases where theoretical PDF cannot be derived, simulated data are then generated
  and their histograms are matched against real-life data for validation.

%***IVM This was all said before...
%Simulated experiments allows evaluation of estimators against ground-truth.
%Errors can be quantitatively analyzed and methods to reduce such errors can be developed.
%Thus simulated experiments could produce quick results, which may provide faster analysis and 
%further insights for developing more accurate estimators.

To evaluate the application of these models in a statistical estimation framework, a practical point of view is adopted.
Practically, all estimators would eventually need to be applied on and work for real images.
Contrary to experiments on simulated data, experiments on real images can often only be compared qualitatively.
This however is critical and will always be included in our investigations.
The reason is that this helps to guard against any systematic errors in the various assumptions and approximations made in developing the theoretical models.
Towards that end, real-data can help to build and validate statistical models.

\section{Research Contributions}
Interestingly, the dual-problems stated above are found to have a single solution.
They are the derivation of scalar, representative and homoskedastic statistical models for the multidimensional, inter-correlated and heteroskedastic POLSAR data.
%  which is 

%In what way is your solution different ? What is the novelty in your approach and hence solution?
%In
%                this thesis, the statistical model for the determinant
%                of the POLSAR covariance matrix is proposed and
%                validated.
These theoretical models are studied and validated in this thesis.
They are based on a highly representative observable, which is the determinant of the POLSAR covariance matrix.
Out
                of many different possible scalar projections of the
                multi-dimensional data, the proposed projection is
                highly representative.
When the dimension number is collapsed to one, the
                special case of the proposed model matches perfectly
                with the commonly used statistical model for SAR
                intensity.
Furthermore,
                the proposed statistical model also leads to the
                derivation of existing as well as the newly proposed
                discrimination measure: determinant-ratio which is the
                generalized multi dimensional version of the widely used
                SAR intensity ratio.

 Another
                novelty of the contributions in this thesis comes from the systematic
                treatment towards the problem of statistical estimation
                of heteroskedastic SAR data.
While
                many different ways to combat the negative impacts
                caused by multiplicative and heteroskedastic properties
                for SAR data have been proposed over the years, they have often been
                handled in a piecemeal manner.
Here,
                a systematic homoskedastic framework of statistical
                estimation is developed, where not only homoskedastic
                statistical models are proposed, the impacts of
                homoskedasticity on estimator development and
                performance evaluation of statistical estimators is also
                studied.
This
                comprehensive treatment not only explains why the
                original SAR data requires special treatment very
                different to the common practice of computational
                sciences, 
it
                also encourages a thorough and unique discussion, for example on
                evaluating different evaluation criteria for statistical
                estimators.  % ;-)).
And
                since the POLSAR data can be considered as the multi
                dimensional extension of traditional SAR data, these
                treatments are shown extensible towards POLSAR data
                which is definitely yet another novelty. 
                
% How do you know if the problem is solved?   State your research objectives
Specifically, the following results are to be obtained:
\begin{enumerate}
\item                Scalar and representative statistical models
                for the multi-dimensional and inter-correlated POLSAR
                data are to be derived.
                \begin{enumerate}
                \item \label{itm:polsar_model} A scalar statistical model for the
                POLSAR observable, being the determinant of the
                covariance matrix, is to be derived.
                \item \label{itm:polsar_discrimination_measure} The statistical models for different discrimination measures which are based on
                the observable are to be established together with its consistent property. %being modelled 
                \item \label{itm:polsar_include_sar} The representation power of these observables is to be illustrated by showing
                that when the multi-channel POLSAR data is collapsed
                into the traditional one-channel SAR data, this
                determinant is transformed into the representative SAR
                intensity.
                \item \label{itm:polsar_validation} The model os to be validated against practical real-life
                captured POLSAR data.
                \end{enumerate}
\item                Homoskedastic statistical models for the
                heteroskedastic (POL)SAR data are to be derived. 
                \begin{enumerate}
                \item \label{itm:log_transform} The fact that (POL)SAR data are heteroskedastic in
                the original domain and logarithmic transformation
                can convert this into a homoskedastic model are to be shown
                \item \label{itm:sar_measures_distance} Homoskedastic measures of distance for
                the multi dimensional POLSAR as well as for the one
                dimensional SAR data are to be derived.
                \item \label{itm:sar_validation} These statistical models are to be validated against both
                practical SAR and POLSAR data
                \end{enumerate}
\item The beneficial applications of these
                statistical models are to be demonstrated. Specifically:
                \begin{enumerate}
                \item \label{itm:filter_vicious_circle} To show how the consistent sense of variance
                helps break the vicious cycle in the problem of SAR
                speckle filtering: a new statistical test for
                homogeneity and a new speckle filter is derived.
                \item \label{itm:log_mse} To show how the consistent sense of subtractive
                distance helps in the problem of evaluating (POL)SAR
                speckle filters: the log-MSE is proposed to combine the
                evaluation of radiometric preservation and speckle
                suppression power.
                \item \label{itm:extend_sar_to_polsar} An example where an existing SAR
                processing technique is extensible towards POLSAR data is to be demonstrated
                \end{enumerate}
\end{enumerate}

In summary, the first main objective is to derive a scalar and representative statistical model for POLSAR data.
%The theoretical foundation is the commonly agreed Circular Complex Normal distributions proposed in \cite{Goodman_JOptSocAm_76}. %***IVM you need a reference here!!
%Theory development is based on applying the statistical variable change theorem on the proposed scalar observable.
%The theoretical model is validated by matching the functional PDF or the simulated histogram with real-life captured data.
The second main objective is to derive homoskedastic measures of distance despite its multiplicative and heteroskedastic nature.
This is first applied on the simpler case of SAR data. %devided into a few smaller objectives.
%For this purpose, first log transformation is shown to not only convert the multiplicative noise model into an adaptive one, 
%  it also results in a homoskedastic statistical model.
%From this consistent sense of variance, the second step illustrates the consistent sense of contrast and distance.
%After these models are validated against practical data, their beneficial applications are further explored.
To combine these two seemingly disconnected threads of research, the results for SAR are extended to POLSAR,
  since the thesis also shows that the POLSAR models are a generic multi-dimensional superset of the traditional SAR data. %***IVM I changed "version" to "superset" in the sentence above. Please check if you are comfortable with that change, and if not, change it back!
%Thus the homoskedastic theoretical models and their benefits are extended from SAR to POLSAR.   
%The 3 paragraphs are collapsed into 1: to address Prof Vun's comment:
%(comment: you are stating the objectives repetitively so far – I counted three time so far, and confuse the results obtained with the objectives. There is no need to re-emphasize the objective if the write-up is clear)

The work in this research results in the following list of articles that has been prepared for public communication: 

\begin{enumerate}
\item \textbf{Neural network-assisted reconstruction of full polarimetric SAR information.}
  Thanh-Hai Le, I. McLoughlin, Ken Yoong Lee, and T. Bretschneider.
  In \textit{Communications, Control and Signal Processing (ISCCSP), 2010 4th International Symposium on}, pages 1–5, 2010.  \cite{Le_2010_ISCCSP} 
\item \textbf{SLC SAR speckle filtering using homoskedastic features of logarithmic transformation.}
  T. H. Le, I. V. McLoughlin, K. Y. Lee, and T. Brestchneider.
  In \textit{Proceedings of the 31st Asian Conference on Remote Sensing (ACRS)}, Hanoi, Vietnam, Nov 2010. \cite{Le_2010_ACRS}
\item  \textbf{SAR Fuzzy-MLE Speckle Filter using the Distance Consistency Property in Homoskedastic Log-Transformed Domain.}
  T. H. Le and I. V. McLoughlin. 
  In \textit{Proceedings of the 32nd Asian Conference on Remote Sensing (ACRS)}, Taipei, Taiwan, Nov 2011 \cite{Le_2011_ACRS}
\item \textbf{Scalar and Representative Models for Polarimetric SAR Data.}
  T. H. Le, I. V. McLoughlin, and C. H. Vun,
  under revision for resubmission to \textit{IEEE Journal on Selected Topic in Applied Earth Observation and Remote Sensing.}
\item \textbf{Using MSE to evaluate SAR speckle filters.}
  T. H. Le, I. V. McLoughlin, and C. H. Vun.
  submitted to \textit{IET Radar, Sonar \& Navigation}
\end{enumerate}
  

\section{Organisation of the Thesis}
The rest of this thesis is organized into five chapters as follows.

Chapter \ref{chap:lit_survey} surveys the related publications
  and the state-of-the-art techniques in modelling, processing, filtering and evaluating (POL)SAR speckle filter. %will be presented.
It will also discuss the different measures of distance for both SAR and POLSAR data, together with their extensive applications. 
Through these discussions, the need for a homoskedastic and scalar model for the multivariate and heteroskedastic POLSAR is brought forward.

%\section{Research Contributions}

%1 show that the results is not only new but also useful
%2 let the reader read the contribution and says: “gosh, if they can really deliver this, that’s exciting; I’d better read on” 
%new theory or hypotheses, writers will need to explain shortcomings in existing theories.
%new solution, writers will need to explain 1) existing problems and clarify how their solutions solve these problems 2) they also need to compare and contrast a range of alternative solutions and explain why their solution is the best among these alternatives.
%new territory, writers need to make a case for why this new territory is important to study and show that it has been overlooked by previous research.
%new methodology, then writers need to 1) critique the methods of previous studies (or characterize them as inconclusive) and 2) explain how they will address these flaws.

%Show that you deliver on the objectives 
%****IVM   Hai, again I've deleted this sentence because it is slightly negative:
%Even
%                though there are small differences between the bottom up
%                development of the research efforts and the top down
%                stated objectives, the research results of each section
%                contribute to the achievement of the objectives.
%Specifically,

%The heart of is thesis contains several contributions which meet the stated objectives.
                Chapter \ref{chap:sar} begins by highlighting that SAR
                data are heteroskedastic in the original domain, whereas
                logarithmic transformation converts this into a
                homoskedastic model. 
It
                also drives consistent and homoskedastic measures of
                distance for the one dimensional SAR data.
These
                models for SAR are then validated against real-life
                practical data. (i.e. see Research Contribution \ref{itm:log_transform} to \ref{itm:sar_validation})

 Chapter
                \ref{chap:polsar} then derives several scalar statistical models for the
                determinant of the POLSAR covariance matrix and
                shows that this is heteroskedastic in the original
                domain as well as homoskedastic in the log-transformed
                domain.
It
                also derives the heteroskedastic discrimination measures
                 as well as several homoskedastic measures of
                distance for the data.
Then
                the models for SAR are shown to be a special case of
                the proposed models for POLSAR and finally it
                validates these proposed models (i.e. see Research Contribution \ref{itm:polsar_model} to \ref{itm:polsar_validation})

 The
                beneficial applications of these models is presented in
                chapter \ref{chap:applications}, where the property of a consistent sense of variance
                 helps to break the vicious cycle in SAR
                speckle filtering.
It
                also shows how MSE can again be applicable in the
                additive and homoskedastic domain to evaluate SAR
                speckle filters.
An
                example of extending existing techniques from SAR to
                POLSAR is also demonstrated in this chapter where the
                MSE evaluation criteria is also shown capable of
                evaluating POLSAR speckle filters.
(i.e. see Research Contribution \ref{itm:filter_vicious_circle} to \ref{itm:extend_sar_to_polsar})
Finally, the contribution of the thesis with respect to the research objectives and research impact is discussed in the concluding chapter \ref{chap:conclusions}.


%The heart of the thesis lies in chapters \ref{chap:sar} and \ref{chap:polsar}.
%While chapter \ref{chap:sar} presents the initial homoskedastic model for SAR data, 
%chapter \ref{chap:polsar} illustrates the more recent and more generic model for POLSAR.
%Chapter \ref{chap:polsar} also specifically describes how the initial model for SAR can be considered as a special case of the POLSAR model.
%
%Chapter \ref{chap:applications} presents several of our published techniques in handling both SAR and POLSAR data.
%Theses techniques can all be considered as applications of the theoretical model presented in chapters \ref{chap:sar} and \ref{chap:polsar}.
%Demonstrated applications include clustering algorithms, speckle filtering techniques and a novel methodology to evaluate (POL)SAR speckle filters.

