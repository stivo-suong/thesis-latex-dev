
% Thesis Abstract -----------------------------------------------------


%\begin{abstractslong}    %uncommenting this line, gives a different abstract heading
\begin{abstracts}        %this creates the heading for the abstract page

In the past decades, the exponential growth in computational power has made the once overly-excessive computationaly-demanding SAR technology now become a feasible and preferred earth observation solution.
As state-of-the-art technology, the SAR technique has been extended in a few directions, one of which is the polarimetric SAR or POLSAR.
POLSAR is the natural extension from SAR exploiting the natural polarization property of Electro-Magnetic (EM) waves.
Similar to the extension from black-and-white images to color photography, the polarimetric extension brought about the multi-channels POLSAR data as compared to the traditional one-channel SAR data.

With (POL)SAR data becomes cheaper and more available, the recent research emphasis is on understanding and developing applications for the data. %it is of great important to understand the data.
The SAR data, however is stochastic by nature, due to the interference phenomena of EM waves. %, SAR measurements are stochastic by nature.
Under this condition, it is very important to understand the statistical models for SAR data.
It is these models that form the foundation for different SAR data processing techniques,
  for example speckle filtering, target detection, image segmentation, and other clustering, classification techniques.
Central to these machine learning or signal processing algorithms is the need for consistent discrimination measures which are derived from the statistical models.

While the statistical models for homogeneous SAR areas have long been developed and widely used,
  in extending the model toward heterogeneous images, the impact of the heteroskedastic property in these models has virtually been ignored.
Specifically the concept of distance as a measure of dis-similarity is seriously flawed,
  as it is widely known that ratio offers a better discrimination measure for SAR \cite{Rignot_1993_TGRS_896}.
This thesis starts off by proposing the logarithmic transformation which is shown converting the SAR data into an homoskedastic model.
Within this log-transformed domain, a few consistent measures of distance is proposed. 
In a sense the distance concept is re-emerged, as the log-transformation converts the widely used ratio into standard subtractive distance.

In extending our statistical understanding of SAR data towards multi-dimensional POLSAR data, an important issue needs to be addressed.
The trouble that the high dimensional data bring about is that there exists not one, as the intensity in the one-channel SAR, but many observables quantities in multi-channel POLSAR.
While different statistical models have been developed for different POLSAR observables,
  for a statistical model to be useful, scalar discrimination measures need to be derived from.
Thus on the one hand, practical application for POLSAR data processing requires the measure to be scalar, consistent and preferably homoskedastic. % on the one hand.
On the other hand, the observable quantity being modelled needs to be naturally representative for the high dimensional POLSAR data.

In this thesis, the determinant of the POLSAR covariance matrix is proposed as an observable quantity to study.
The representative power of this observable is demonstrated when the multi-dimensional POLSAR data is collapsed into the traditional one-dimensional SAR scenario, the determinant then is transformed into the standard SAR intensity.
Statistical models for this heteroskedastic POLSAR determinant and homoskedastic log-determinant are then derived.
And since POLSAR can be viewed as a multi-dimensional extension of SAR, the thesis describes how the standard statistical models for SAR can be put within the natural coverage of this generic models for POLSAR.
%Subsequently logarithmic transformation is investigated for this generalized SAR model,
%  which lead to the proposal for several linear and additive measures of distance for POLSAR.

%There are a few benefits of using the homoskedastic measures of distance for POLSAR.
In short, to address the dual-problem described above,
  this thesis proposes several scalar, additive and homoskedastic measures of distance for the multi-dimensional, multiplicative and heteroskedastic POLSAR data. 
%The proposed statistical models and the homoskedastic measures of distance for POLSAR bring about a couple of benefits. 
There are a couple of benefits in using these homoskedastic measures of distance for (POL)SAR.
First, compared to the multiplicative dis-similarity measure of ratio, these additive measures of distance fit more naturally with the linear nature of digital imagery.
Second, within the homoskedastic log-transformed domain, where the distance concept as well as the Gauss-Markov theorem is again applicable,
  the use of Mean Squared Error (MSE) as universal objective function deserves to be reviewed. % as a reliable objective function / criteria for (POL)SAR estimators.
%This may open up a number of different research directions for future studies of (POL)SAR.
As example applications of how these statistical models and distance measures,
  a novel clustering algorithm, a new speckle filter and various procedures in using MSE to evaluate speckle filters are also presented as part of the thesis.

\end{abstracts}
%\end{abstractlongs}


% ---------------------------------------------------------------------- 
