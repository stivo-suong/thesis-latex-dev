% Thesis Abstract -----------------------------------------------------


%\begin{abstractslong}    %uncommenting this line, gives a different abstract heading
\begin{abstracts}        %this creates the heading for the abstract page

SAR and POLSAR stochastic data are multiplicative and heteroskedastic in their natural domain.
It is hence desirable to establish additive and homoskedastic models,
  such that the benefits of homoskedastic statistical estimation framework can be demonstrated and realized for practical applications such as speckle filtering.
In addition, the processing of multi-dimensional POLSAR data requires the establishment of discrimination observables
  which are to be scalar and statistically consistent.
Moreover, these same scalar observable quantities also need to be naturally representative for the multidimensional POLSAR data.

In this thesis, the effects of homoskedasticity on SAR and POLSAR speckle filtering within the framework of computational and statistical estimation are extensively studied. 
Concurrently, the statistical behaviour of the determinant of POLSAR covariance matrix is also explored,
  where it is shown to be the representative scalar observable for the multidimensional POLSAR, similar to the role of the intensity in SAR.

As a result of these studies, several scalar statistical models based on the determinant of POLSAR covariance matrix are proposed and validated.
These generic models for POLSAR are also shown to be both multiplicative and heteroskedastic, similar to the models for SAR intensity. 
Subsequently, logarithmic transformation is applied onto both SAR and POLSAR models
  to convert them into additive and homoskedastic models.
From these models, the benefits of homoskedastic statistical estimation framework are then further explored.

There are several beneficial implications of the theory proposed in this thesis. 
Since the scalar and representative models for POLSAR extend the traditional models for SAR's intensity,
  its main benefit is that it enables the adaptation of many existing SAR data processing techniques for POLSAR data.
Similarly, the homoskedastic model carries additional benefits.
For example, for inexperienced researchers (such as this author when he began this research),
  this thesis proposes additive and homoskedastic models for both SAR and POLSAR data,
  which are simpler and more familiar.
They are helpful by keeping these inexperienced researchers from falling into several common traps in (POL)SAR data processing.
%For inexperienced researchers (such as this author when he began this research),
%  who attempt to process either SAR or POLSAR data using existing computation algorithms
%  which most of the time expect an additive and homoskedastic input,
%  this thesis proposes alternative and more familiar additive and homoskedastic models.
For more experienced researchers, besides the unified scalar statistical theory for both SAR and POLSAR, 
  this thesis also proposes several ways that homoskedastic data processing can be used to neutralize certain existing negative impacts of heteroskedasticity on the statistical estimation framework for (POL)SAR.

%\newpage
%In the past decades, the exponential growth in computational power has made the once overly-excessive computationaly-demanding SAR technology now become a feasible and preferred earth observation solution.
%As state-of-the-art technology, the SAR technique has been extended in a few directions, one of which is the polarimetric SAR or POLSAR.
%POLSAR is the natural extension from SAR exploiting the natural polarization property of Electro-Magnetic (EM) waves.
%Similar to the extension from black-and-white images to color photography, the polarimetric extension brought about the multi-channels POLSAR data as compared to the traditional one-channel SAR data.
%
%The extension from SAR to POLSAR brings about the representation issue in that:
% there exists not one, like the intensity in SAR, but many observable quantities in POLSAR.
%While different statistical models have been developed for different POLSAR observables,
% for a statistical model to be useful, scalar discrimination measures need to be derived from.
%Thus on the one hand, practical application for POLSAR data processing requires the measure to be scalar, consistent and preferably homoskedastic. 
%On the other hand, the observable quantity being modelled needs to be naturally representative for the high dimensional POLSAR data.
%
%Furthermore, since POLSAR can be considered as the multi-dimensional extension of the traditional SAR, a common problem exists for both SAR and POLSAR data.
%That is: the sense of subtractive distance is seriously flawed in the original multiplicative and heteroskedastic domain of (POL)SAR data.
%The digital technologies which ranges from image processing and presentation to artificial intelligence and machine learning algorithms that are used to process (POL)SAR data, however, are linear and additive in nature.
%This linear and additive nature is manifested for example: in a large number of algorithms in the field of  computational intelligence which make use of the universal MSE as objective functions, as well as the extensive use of variance and contrast in digital image processing.
%
%Thus a solution is needed to address the dual problem of 1) figuring out scalar and representative statistical models for the multi-dimensional and inter-correlated POLSAR data and 2) arriving at consistent measures of subtractive distance for both SAR and POLSAR data, which can pave the way for further applications of existing digital-based algorithms towards (POL)SAR data.
%
%In this thesis, scalar, representative and homoskedastic measures of distances are derived for the multi-dimensional, inter-correlated and heteroskedastic (POL)SAR data.
%And their benefits are highlighted against well-known published work.
%
%Specifically, compared with other scalar statistical models for POLSAR,
%  the proposed models are representative for POLSAR data in the sense that the observable being modelled is the generalized version of the representative one-channel SAR intensity in the multidimensional case.
%Moreover, several multiplicative dis-similarity measures and additive measures of distance are derived from the proposed statistical models.
%
%And similar to the ratio in the original SAR domain, the subtractive distance in the log-transformed (POL)SAR domain is statistically consistent.
%Compared to the subtractive distance in the original domain, the PDF of the subtractive distances are not dependent on the underlying backscattering coefficient.
%And compared to the ratio in the original domain, the subtractive residual is linear in nature, which fits better with the digital image processing and presentation processes.
%This has created a number of benefits, for example: the MSE is again shown to be capable of evaluating SAR speckle filters or the consistent variance is demonstrated to be instrumental in breaking the vicious circle of SAR speckle filtering.

\end{abstracts}
%\end{abstractlongs}


% ---------------------------------------------------------------------- 
