\chapter{Homoskedastic Models for Heteroskedastic SAR Data} %chapter 3
\label{chap:sar}

This chapter  highlights that SAR
                data are heteroskedastic in the original domain.
Relying solely on statistical and mathematical principles,             
the log-transformed SAR data is then shown to be homoskedastic.
From there, the consistent sense of variance and a few consistent measures of distance are proposed and validated against real-life
                practical SAR data.

\section{Original Heteroskedastic Model of SAR data}
\label{sec:heteroskedastic_sar}

The stochastic nature of SAR data has been illustrated earlier (see Eqns. \ref{eqn:SAR Amplitude PDF} and \ref{eqn:SAR Intensity PDF}).
From a statistical viewpoint, the multiplicative nature of amplitude ($A$) and intensity ($I$)  data can be explained as follows. 
Let us consider two fixed unit distributions which are independent to $\sigma$ given below:

\begin{equation}
\label{eqn:SAR Unit Amplitude PDF}
\caption{eqn:SAR Unit Amplitude PDF}
pdf(A_1) = 2A_1 e^{ - \left( A_1^2 \right) }
\end{equation}
\begin{equation}
  \label{eqn:SAR Unit Intensity PDF}
  \caption{eqn:SAR Unit Intensity PDF}
pdf(I_1) = e^{ \left( -I_1 \right) }
\end{equation}

Using the variable change theorem,
    it can be shown that
    the SAR amplitude and intensity are equal to %are simply
    a scaled version of these unit variables,
    specifically $A= \sigma A_1 $ and $I= \sigma^2 I_1 $. 
These relationships evidently display a mutiplicative nature. 
In fact, the condition has long been noted,
  but from different perspectives,
  in various proposed SAR models (e.g. \cite{Jakeman_1980_JPhysAMathGen}).

Let us now consider the population expected mean and variance of the four statistical distributions for these observables, i.e. $A, A_1, I, I_1$.
They are given in Table \ref{tbl:orginal_sar_avg_var}. 
If spatial homogeneity is defined as an imaging area having the same back-scattering coefficient $\sigma$,
  then over a homogeneous scene, the measured values can then be considered as samples coming from a single stochastic process.
However, over heterogeneous areas where $\sigma$ varies significantly,
  it is then evidently clear that: the amplitude  as well as the intensity of SAR data suffer from hetoroskedastic phenomena. 
%  which is defined as the dependence of conditional expected variance of SAR data on the conditional expectation of mean.
%This means that  the conditional expected variance of SAR data is dependent on the conditional expectation of the mean.
In statistical context, heteroskedasticity and homoskedasticity are terms defined to denote the fact that different sub-populations of a given set of samples can have different or similar variability respectively.
In the context of SAR (and POLSAR) imagery, they respectively denote the fact that different areas in a given image can have different different or similar sample variance, depending on the underlying signal $\sigma$.
In the context of speckle filtering which aims to estimate the unknown parameter $\sigma$,
  Table \ref{tbl:orginal_sar_avg_var} also clearly indicates an ambiguous circle:
in the original domain, estimating variance is equal to estimating the mean, but this is as hard to estimation as the variance.

\begin{table}[!h]
\normalsize
\centering
\caption{ Mean and Variance of Original SAR data }
\label{tbl:orginal_sar_avg_var}
\begin{tabular}{|l|l|}
\hline
Mean & Variance \\
\hline
$avg(A_1) = \frac{ \sqrt{\pi}}{2}$ & $var(A_1) = \frac{(4-\pi)}{4}$ \\
$avg(I_1) = 1$ & $var(I_1) = 1$ \\
$avg(A) = \frac{\sqrt{\pi}}{2} \cdot \sigma $ & $var(A) = \frac{(4-\pi)}{4} \cdot \sigma^2 $ \\
$avg(I) = \sigma^2 $ & $ var(I) = \sigma^4$ \\
\hline 
\end{tabular}
\end{table}

The formulas above are extremely well known in this research field. 
In fact while pioneering the estimation of the equivalent number of look (ENL) index, 
 it was noted by Lee et.al. that the ratio of expected standard deviation to mean is a constant in both cases (i.e. $snr(A)=\sqrt{\frac{4}{\pi}-1}$ and $snr(I)=1$) \cite{Lee_1992_IMA_298}.
%Here we just offer a different perspective,
%  one that avoids the negative effects of heteroskedasticity in SAR data processing.
%Here, we just offer a different interpretation, which sets the stage for the discussion of logarithmic transformation. 

In this thesis, a different perspective will be proposed which avoids the negative effects of heteroskedasticity in SAR data processing.
 
\section{The Effects of Heteroskedasticity in SAR Processing}%IVM: shortened to make it look nicer
\label{sec:heteroskedastic_sar_processing}

%\subsection{Statistical Heteroskedasticity of SAR data}

%The nature of SAR speckle is stochastic over homogeneous areas and heteroskedastic heterogeneously.
As shown in the previous section,
  over heterogeneous area, the stochastic SAR speckle is heteroskedastic in nature.
Statistical models are used to describe how the underlying back-scattering coefficient ($\sigma$) affects the distribution of the observable SAR data. 
While the statistical model within individual resolution cells is well established,
  it is only applicable to homogeneous areas.
Practical images, however, are heterogeneous.
Furthermore, it is precisely this spatial variation that is often of high interest.
%And while the statistical model within individual resolution cells is well established,
%  the applicability of these models, however, are restricted to homogeneous areas. 
%Practical images, nevertheless, are heterogeneous. 
%Crucially it is this spatial variation that is of high interest. 
%This fact gives rise to the question that seemed obvious:
  %how to call an analysis area heterogeneous and subsequently
%  what to do in the case of heterogeneity.

As such, various statistical models for heterogeneous areas have been proposed (see \cite{Touzi_2002_TGRS} for a detailed review). 
However, while most of the models highlight the multiplicative nature of sub-pixel or homogeneous original SAR data, in extending the model to heterogeneous images, virtually none have noted that spatial variation also gives rise to heteroskedasticity phenomena. 
Heteroskedasticity, as explained later, is defined as the dependence of conditional expected variance of original SAR data on the conditional expectation of its mean or equivalently its underlying back-scattering coefficient $(\sigma)$. 

Speckle filters are effectively estimators attempting to estimate the unknown coefficient $\sigma$ from the observable SAR signal $(A)$. 
Thus, SAR speckle-filtering can be and has been positioned within the context of estimation theory \cite{Touzi_2002_TGRS}, where
the statistical framework is used to estimate unknown statistical parameters from the observable data. 
The stages of this framework consist of statistical modelling, estimator development and evaluating the performance of the statistical estimators. 
%In this section, the negative effects of heteroskedasticity on speckle filtering is analysed.
It is also known that heteroskedasticity gives rise to serious negative impacts on various stages of speckle-filtering. 

In modelling, heteroskedasticity has direct consequences in the central question of homogeneity or heterogeneity. 
In optical images, the contrast or variance among neighbouring pixels have been used to measure homogeneity. 
Unfortunately such techniques do not appear to be effective under the heteroskedastic condition of original SAR data. 
In SAR images, both of these measures are dependent on the underlying signal $(\sigma)$. 
This make the problem of estimating variance equal to the problem of estimating the mean and is the same as the main problem: estimating $\sigma$. 
The fact give rise to the ambiguity circle in SAR speckle filtering.

A case in point can be found in a recently published improved sigma filter \cite{Lee_TGRS_2009}. 
The technique determines outlying points as being too far away from the standard deviation. 
However, as is done in \cite{Lee_TGRS_2009}, to estimate standard deviation, an estimator of the mean is required and used. 
It is interesting to note that the MMSE estimator used to estimate the mean in \cite{Lee_TGRS_2009}, itself alone, is a rather successful speckle-filter \cite{Lee_PAMI_1980}.

Heteroskedasticity also poses numerous challenges in designing an efficient estimator. Heteroskedasticity directly violates Gauss-Markov theorem's homoskedastic assumption. 
Thus it renders the efficiency of any naive Ordinary Least Square estimator in serious doubt. 
If the variance is known \textit{a priori}, it has been proven \cite{Aitken_1934_ProcsRoyalSocEdinburg}, that a weighted mean estimator is the best linear unbiased estimator. 
Interestingly, as noted by Lopes \cite{Lopes_TGRS_1990}, most commonly known successful adaptive filters \cite{Lee_PAMI_1980, Kuan_1985_PAMI, Frost_PAMI_1982} do make use of weighted mean estimations. 
The problem, however, is that in SAR speckle filtering, variance is not known \textit{a priori}. 
And even though variance can be estimated from observable values, as the ambiguity circle goes, estimating the variance is as hard as estimating the underlying coefficient $\sigma$ itself.

Heteroskedasticity also affects the performance evaluation stage of the estimation framework.
Statistical estimators are routinely evaluated using their bias and variance performance
  which, when combined, normally result in MSE evaluation.
Unfortunately, under heteroskedastic conditions, Least Squared Error estimation may not be optimal.   
In other words, Minimum Squared Error can no longer serves as a reliable evaluation metric.
A number of different evaluation metrics have been proposed for SAR speckle filters,
  however, none have appeared capable of taking on the MSE's previous role.
%IVM: Hai, the above "Minimum Squared Error" should be Mean Squared Error???

Last but certainly not least, is the bad impact of heteroskedasticity on SAR image interpretation. 
Most of the task to be carried out in interpreting SAR images almost invariably involve target detection, target segmentation and/or target classification. 
All of these tasks require good discriminant functions. 
Fundamental to all of those is the need for a consistent measure of distance. 
Unfortunately, by definition, heteroskedasticity provides inconsistent measures of distance. 
This inconsistent measure of distance coupled with the failure of the ordinary least square regression methods have caused a large class of artificial neural networks as well as a number of other computational intelligence methods to under-perform for SAR data.


\section{The Homoskedastic Effect of Logarithmic Transformation}
\label{sec:homoskedastic_sar}

In this section, a base-2 logarithmic transformation is proposed for SAR data. 
The choice of base-2 is due to implementation reasons,
  since base-2 logarithms can be computed faster by computer than either natural or decimal logarithms,
  and yet the ability to transform heterskedactic speckle into a homoskedastic relationship is still maintained.
Under base-2 log-transformation, the original variables become:

\begin{equation}
  \label{eqn:SAR Unit Log Amplitude Observable}
  \caption{eqn:SAR Unit Log Amplitude Observable}
L_{1}^{A} = \log_2(A_1) = L_{1}^{I} / 2 
\end{equation}
\begin{equation}
  \label{eqn:SAR Log Amplitude Observable}
  \caption{eqn:SAR Log Amplitude Observable}
L_A = \log_2(A) 	= L_{1}^{A} + \log_2\sigma 
\end{equation}
\begin{equation}
  \label{eqn:SAR Unit Log Intensity Observable}
  \caption{eqn:SAR Unit Log Intensity Observable}
L_{1}^{I} = \log_2(I_1) = 2 L_{1}^{A} 
\end{equation}
\begin{equation}
  \label{eqn:SAR Log Intensity Observable}
  \caption{eqn:SAR Log Intensity Observable}
L_I = \log_2(I) 	= L_{1}^{I} + 2 \log_2\sigma
\end{equation}

The equations above highlight the relationships among random variables in the log-transformed domain. 
Two interesting points are evident from these formulae. 
Firstly, in the log-transformed domain, working on either amplitude or intensity will tend to give identical results. 
Secondly, the effects of converting the multiplicative nature to an additive nature through logarithmic transformation
is clearly manifested.
Now, bearing in mind the relationship among the random variables, %it is then trivial to 
%show that the probability distribution of these log-transformed variables are related as follows:
application of variable change theorem leads to: 

\begin{equation}
  \label{eqn:SAR Unit Log Amplitude PDF}
  \caption{eqn:SAR Unit Log Amplitude PDF}
pdf(L_{1}^{A}) = 2 \cdot 2^{\left( 2 L_{1}^{A} \right)}e^{-2^{\left( 2 L_{1}^{A} \right) }} 
\end{equation}
\begin{equation}
  \label{eqn:SAR Log Amplitude PDF}
  \caption{eqn:SAR Log Amplitude PDF}
pdf(L_A) = 2 \cdot 2^{\left( 2 L_A - 2 \log_2 \sigma \right)}e^{-2^{\left( 2 L_A - 2 \log_2 \sigma \right)}} 
\end{equation}
\begin{equation}
pdf(L_{1}^{I}) = 2^{L_1^I} e^{-2^{L_1^I}}  
  \caption{eqn:SAR Unit Log Intensity PDF}
  \label{eqn:SAR Unit Log Intensity PDF}
\end{equation}
\begin{equation}
pdf(L_I) = 2^{\left( L_I - 2 \log_2 \sigma \right)} e^{-2^{\left( L_I - 2 \log_2 \sigma \right)} }  
  \caption{eqn:SAR Log Intensity PDF}
  \label{eqn:SAR Log Intensity PDF}
\end{equation}

Noting that these distributions belong to the Fisher-Tippet family, the population expected mean and variances 
are obtained and presented in Table \ref{tbl:sar_log_domain_avg_var}, with $\gamma \approx 0.577$ being the Euler-Mascheroni constant. 
Most importantly, it can be seen that the means are biased and the variances are no longer related to $\sigma$. 

\begin{table}[h]
\normalsize
\centering
\caption{ Mean and Variance of Log Transformed values: here the variance is no longer dependent on $\sigma$ }
\label{tbl:sar_log_domain_avg_var}
\begin{tabular}{|l|l|}
\hline
Mean & Variance \\
\hline
$avg(L_{1^A}) = \frac{ - \gamma }{2} \cdot \frac{1}{\ln2}$ & $var(L_{1^A}) = \frac{ \pi ^2}{24} \cdot \frac{1}{(\ln2)^2}$ \\
$avg(L_{1^I}) = - \gamma \cdot \frac{1}{\ln2} $ & $var(L_{1^I}) = \frac{ \pi ^2}{6} \cdot \frac{1}{(\ln2)^2} $ \\
$avg(L_A) = \frac{ - \gamma }{2} \cdot \frac{1}{\ln2} + \log_2{\sigma}$ & $var(L_A) = \frac{ \pi ^2}{24} \cdot \frac{1}{(\ln2)^2}$ \\
$avg(L_I) = - \gamma \cdot \frac{1}{\ln2} + 2 \log_2{\sigma}  $ & $ var(L_I) = \frac{ \pi ^2}{6} \cdot \frac{1}{(\ln2)^2}$ \\
\hline
\end{tabular}
\end{table}

Table \ref{tbl:sar_log_domain_avg_var} confirms the condition of homoskedasticity,  
defined as the independence of the conditional expected variance on the conditional expectation of the mean. 
This result is consistent with findings in \cite{Arsenault_JOptSocAm_1976}. 
The main difference would be the use of base-2 logarithm which is preferred here for more efficient computation. 

The discussion so far is further summarized in Table \ref{tbl:sar_variables_properties}.
It shows that while the original data, especially intensity values, should be preferred for 
multi-look processing, the log-transformed domain with its homoskedastic distribution offers a consistent sense of variance. 

\begin{table*}[t]
%\normalsize
%\footnotesize -2
\scriptsize
\centering

\caption{ The properties of observable SAR random variables }
\label{tbl:sar_variables_properties}

\begin{tabular}{|l|l|l|l|}
\hline
 RV & Relationships  & Variance (skedasticity) & Mean (biasness) \\
\hline
$A$ 
	& $A=\sigma A_1 $ 
	& Heteroskedastic $var(A) = \frac{(4-\pi)}{4} \cdot \sigma^2 $ 
	& Unbiased $avg(A) = \frac{\sqrt{\pi}}{2} \cdot \sigma $ \\
$I$ 
	& $I=A^2=\sigma^2 I_1 $ 
	& Heteroskedastic $ var(I) = \sigma^4$ 
	& Unbiased $avg(I) = \sigma^2 $\\
$L_A$ 
	& $L_A=\ln(A)=L_{1^A} + \log_2{\sigma}$ 
	& Homoskedastic $var(L_A) = \frac{ \pi ^2}{24} \cdot \frac{1}{(\ln2)^2}$ 
	& Biased $avg(L_A) = \frac{ - \gamma }{2} \cdot \frac{1}{\ln2} + \log_2{\sigma}$ \\
$L_I$ 
	& $L_I=\ln(I)=L_{1^I} + 2 \log_2{\sigma}$  
	& Homoskedastic $var(L_I) = \frac{ \pi ^2}{6} \cdot \frac{1}{(\ln2)^2}$ 
	& Biased $avg(L_I) = - \gamma \cdot \frac{1}{\ln2} + 2 \log_2{\sigma}  $ \\
\hline
\end{tabular}

\end{table*}

To experimentally validate this theoretical discussion, real world data is extracted from a known  homogenous area (from within a RADARSAT2 image). The PDF histogram from this data is then plotted in 
Fig. \ref{fig:modelled_response} along with the modelled PDF response. 
The excellent agreement observable between the real world and modelled data tends to validate the log-transformed model. 
%In fact, this modelling has been used successful in explaining speckle phenomena, also verified 
%through scientific experiments \cite{Ulaby_TGRS_1988}.

\begin{figure}[!h]
\centering
%\centerline{
\begin{tabular}{c}
	\subfloat[amplitude]{
		 \epsfxsize=2.5in
		 \epsfysize=2.5in
		 \epsffile{images/amplitude_histogram.eps} 	
		 \label{amplitude}
	} 
	\hfill	
	\subfloat[intensity]{
		 \epsfxsize=2.5in
		 \epsfysize=2.5in
		 \epsffile{images/intensity_histogram.eps} 	
		 \label{intensity}
	} \\
	\subfloat[log amplitude]{
		 \epsfxsize=2.5in
		 \epsfysize=2.5in
		 \epsffile{images/log_amplitude_histogram.eps} 	
		 \label{amplitude}
	} 
	\hfill	
	\subfloat[log intensity]{
		 \epsfxsize=2.5in
		 \epsfysize=2.5in
		 \epsffile{images/log_intensity_histogram.eps} 	
		 \label{intensity}
	} 
\end{tabular}
%}
\caption{Observed histogram of a homogenous area and modelled PDF response}
\label{fig:modelled_response}
\end{figure}

\section{Consistent Measures of Distance in the Log-Transformed Domain}
\label{sec:consistent_measures_distance_sar}

The consistent feature is explored and illustrated in this section from two different perspectives.

\subsection{The Distance and Log-Distance Models}

First, the back-scattering coefficient $\sigma$ is assumed to be \textit{known a priori}.
With that, let us consider the dispersion random 
variable which is defined as the distance between an observable sample and its expected value:

\begin{equation}
  \label{eqn:SAR Amplitude Distance Observable}
  \caption{eqn:SAR Amplitude Distance Observable}
D_A = A - avg(A)   
\end{equation}
\begin{equation}
  \label{eqn:SAR Intensity Distance Observable}
  \caption{eqn:SAR Intensity Distance Observable}
D_I = I - avg(I)  
\end{equation}
\begin{equation}
  \label{eqn:SAR Log Amplitude Distance Observable}
  \caption{eqn:SAR Log Amplitude Distance Observable}
D_{L^A} = L_A - avg(L_A)
\end{equation}
\begin{equation}
  \label{eqn:SAR Log Intensity Distance Observable}
  \caption{eqn:SAR Log Intensity Distance Observable}
D_{L^I} = L_I - avg(L_I) 
\end{equation}

Noting the results from the previous section, the PDF for these variables can be derived as follows:

\begin{equation}
  \label{eqn:SAR Amplitude Distance PDF}
  \caption{eqn:SAR Amplitude Distance PDF}
pdf(D_A) = 2 \cdot \frac{\left( D_A + \sigma \sqrt{\pi}/2 \right)}{\sigma^2}e^{ \left[ - \frac{\left( D_A + \sigma \sqrt{\pi}/2 \right)^2}{\sigma^2}   \right] }
\end{equation}
\begin{equation}
  \label{eqn:SAR Intensity Distance PDF}
  \caption{eqn:SAR Intensity Distance PDF}
pdf(D_I) = \frac{1}{\sigma^2}e^{\left[ -\left( D_I + \sigma^2 \right) / \sigma^2 \right] } \\
\end{equation}
\begin{equation}
  \label{eqn:SAR Log Amplitude Distance PDF}
  \caption{eqn:SAR Log Amplitude Distance PDF}
pdf(D_{L^A}) = 2 \cdot 2^{\left( 2 D_{L^A} + 2 \frac{\gamma}{2 \ln2} \right)} e^{-2^{\left( 2 D_{L^A} + 2 \frac{\gamma}{2 \ln2} \right)}}
\end{equation}
\begin{equation}
  \label{eqn:SAR Log Intensity Distance PDF}
  \caption{eqn:SAR Log Intensity Distance PDF}
pdf(D_{L^I}) = 2^{\left( D_{L^I} + \frac{\gamma}{\ln2} \right)} e^{-2^{\left( D_{L^I} + \frac{\gamma}{\ln2} \right)}}
\end{equation}

It is clear that these distributions are dependent on $\sigma$ in the original domain (the upper two equations), are now independent of it when processed in the log-domain (the lower two equations).

\subsection{The Dispersion and Contrast Models}

Let us now investigate SAR data from the second perspective,
  where two adjacent resolution cells are known to have \textit{an identical but unknown} back-scattering coefficient $\sigma$.
Now, the contrast random variable is defined, also in the log-transformed domain, as the distance between two measured samples.

\begin{equation}
  \label{eqn:SAR Amplitude Contrast Observable}
  \caption{eqn:SAR Amplitude Contrast Observable}
C_A = A_1^\sigma - A_2^\sigma 
\end{equation}
\begin{equation}
  \label{eqn:SAR Intensity Contrast Observable}
  \caption{eqn:SAR Intensity Contrast Observable}
C_I = I_1^\sigma - I_2^\sigma 
\end{equation}
\begin{equation}
  \label{eqn:SAR Log Amplitude Contrast Observable}
  \caption{eqn:SAR Log Amplitude Contrast Observable}
C_{L^A} = L_1^{A_\sigma} - L_2^{A_\sigma} = log_2 { \left( {A_1}/{A_2} \right) }
\end{equation}
\begin{equation}
  \label{eqn:SAR Log Intensity Contrast Observable}
  \caption{eqn:SAR Log Intensity Contrast Observable}
C_{L^I} = L_1^{I_\sigma} - L_2^{I_\sigma} = log_2 { \left( {I_1}/{I_2} \right) }
\end{equation}

Noting that $C_x = D_1^x - D_2^x$, it should come as no surprise that the measure of contrast is also consistent in the 
log-domain but inconsistent in the original domain. The PDF of the log-transformed variables can be expressed as:

\begin{equation}
  \label{eqn:SAR Log Amplitude Contrast PDF}
  \caption{eqn:SAR Log Amplitude Contrast PDF}
pdf(C_{L^A}) = 2 \frac{2^{\left(2 C_{L^A} \right)}}{\left[ 1+2^{\left( 2 C_{L^A} \right)} \right]^2} \ln2
\end{equation}
\begin{equation}
  \label{eqn:SAR Log Intensity Contrast PDF}
  \caption{eqn:SAR Log Intensity Contrast PDF}
pdf(C_{L^I}) = \frac{2^{\left( C_{L^I} \right)}}{\left[ 1+2^{\left( C_{L^I} \right)} \right]^2} \ln2 
\end{equation}

Evidently these distributions are also consistent, i.e. they are independent of $\sigma$.
To illustrate this property, Fig. \ref{fig:residual_as_noise} is generated using the data extracted from a homogeneous area in a RADARSAT-2 image.
As can be seen, it shows excellent agreement between the analytical PDF and the observable histogram of dispersion and contrast computed from an homogeneous area in the image. 

\begin{figure}[h]
\centering
\begin{tabular}{c}
	\subfloat[dispersion]{
		 \epsfxsize=2.5in
		 \epsfysize=2.5in
		 \epsffile{images/log_intensity_dispersion_histogram.eps} 	
		 \label{amplitude}
	} 
	\hfill
	\subfloat[contrast]{
		 \epsfxsize=2.5in
		 \epsfysize=2.5in
		 \epsffile{images/log_intensity_contrast_histogram.eps} 	
		 \label{intensity}
	}
\end{tabular}
\caption{Observed and modelled PDF of dispersion 
and contrast in homogeneous log-transformed intensity images.}
\label{fig:residual_as_noise}
\end{figure}

Given that real SAR images are heterogeneous and the back-scattering coefficient is unknown, it is evident that 
the measure of observable contrast in original SAR data differs, i.e. is inconsistent, across different homogeneous 
areas. On the other hand, the observable contrast in the log-transformed domain is consistently the same across 
different homogeneous areas. As such, one possible benefit is that should the sigma filter \cite{Lee_TGRS_2009} be 
designed against the $pdf(C_{L^I})$, then the scale estimator would probably no longer be required. 

\subsection{Sampling distribution of Variance in Log-Transformed domain}

From the previous section result, two sample variance distribution ($V_x = C_x^2$) can be given analytically as

\begin{equation}
  \label{eqn: SAR Two Samples Log Intensity Variance PDF}
  \caption{eqn: SAR Two Samples Log Intensity Variance PDF}
pdf(V_{L^I}) = 
	\frac{\ln2}{2} \frac{1}{\sqrt{V_{L^I}}} \left( \frac{2^{\sqrt{V_{L^I}}}}{\left( 1+2^{\sqrt{V_{L^I}}} \right)^2} + \frac{2^{-\sqrt{V_{L^I}}}}{\left( 1+2^{-\sqrt{V_{L^I}}} \right)^2} \right)
\end{equation}
\begin{equation}
  \label{eqn: SAR Two Samples Log Amplitude Variance PDF}
  \caption{eqn: SAR Two Samples Log Amplitude Variance PDF}
pdf(V_{L^A}) =
	\frac{\ln2}{4} \frac{1}{\sqrt{V_{L^A}}} \left( \frac{2^{\sqrt{4 V_{L^A}}}}{\left( 1+2^{\sqrt{4 V_{L^A}}} \right)^2} + \frac{2^{-\sqrt{4 V_{L^A}}}}{\left( 1+2^{-2\sqrt{V_{L^A}}} \right)^2} \right)
\end{equation}

It could be seen that, in the log-transformed domain, not only the expected value (mean) of the observable variance is constant, its distribution is also independent of $\sigma$. 
The above analysis is applicable for two sample variances. 
For larger number of samples $n > 2$, Monte-Carlo simulation is used to visualise the distribution of sample variances. 
The simulation is described as follows:

\begin{enumerate}
\item A large number of input samples are generated ($n \cdot 10^6$ in the figure \ref{fig:variance}) from the PDF given in Eqn. \ref{eqn:SAR Log Intensity PDF}.
\item The input samples are then divided into $10^6$ independent sets, each set consisting of $n$ samples.
\item For each input set, the sample variance is calculated.
\item For all input sets, a histogram of the calculated values is then plotted.
\item The consistent variance property is observable when the same plots are obtained, regardless of the value $\sigma$ used in simulations.
\end{enumerate}

\begin{figure}[h!]
\centering
\begin{tabular}{c}
	\subfloat[simulation match analysis]{
		 \epsfxsize=2.5in
		 \epsfysize=2.5in
		 \epsffile{images/log_intensity_variance_hist_model_cdf_scene1.eps} 	
		 \label{amplitude}
	} 
	\hfill
	\subfloat[simulation for different number of samples]{
		 \epsfxsize=2.5in
		 \epsfysize=2.5in
		 \epsffile{images/log_intensity_variance_no_of_samples_cdf.eps} 	
		 \label{intensity}
	}
\end{tabular}
\caption{ Sample Variance: Observable and Modelled CDF Response }
\label{fig:variance}
\end{figure}

Fig. \ref{fig:variance} not only provides evidence for the validity of the simulation process,
  it also visualizes the shapes of the CDF for sample variance distribution of different look numbers. 

\subsection{Sample variance as a statistical measure of homogenity}

The discussion in the previous sub-section is based on the assumption of single stochastic processes, and thus homogeneity.  
In a given analysis window within real practical SAR images, such a perfect condition is unknown. 
The principle of SAR speckle filtering however is to group pixels into homogeneous areas, allowing stochastic components to be removed. 
In order to do that, a methodology is required to assert a partitioned area as being homogeneous. 
Fortunately the sampling distribution of log-variance provides a consistent method for testing this homo/hetero-geneity condition.

Under the null hypothesis that the given analysis area is homogeneous,
  the theoretical distribution of sample variance has been given earlier. 
The PDF, and equivalently the CDF, of this distribution can be visualized from Monte-Carlo simulations, as shown in Fig. \ref{fig:variance}. 
Thus, given a computed value of the sample variance in the log-transformed domain,
  the notion that a given analysis area is heterogeneous or homogeneous can be put to test. 
Specifically given a pre-determined threshold-value and number of samples,
  if the observable sample variance is higher than a cut-off value, the null hypothesis can be rejected. 

The steps for finding these cut-off values are given below:

\begin{enumerate}
\item For each number of samples, a CDF plot is derived using the Monte-Carlo simulation described earlier.
  \item Interpolating from the CDF plot, a corresponding cut-off value is derived for each threshold-level.
  \item The combined results form a table, similar in structure to Table \ref{tab:var_cut_off_values}.
  \item The first 3 steps are repeated a few times,
     and the summary statistics (i.e. mean and variance) of the cut-off values for each combination of threshold-value and look-number is reported.  
\end{enumerate}

Table \ref{tab:var_cut_off_values} gives the result of one such experiment for an arbitrarily chosen set of threshold values.
The table is useful in cases when an area is to be tested for homogeneity,
  for example in the kMLE speckle filter \cite{Le_2010_ACRS}

\begin{table}[H]
\centering
\caption{Cut off values}
\label{tab:var_cut_off_values}

\small
%awk -F"&" '{print $4}' tmp.0.txt | awk -F"(" '{print $1}'
\begin{tabular}{c|c|c|c|c|c}
No  & 70\%            &75\%             &80\%             &85\%             &90\%\\
\hline
2   &2.9024 (0.0252)  &3.7088 (0.0261)  &4.7960 (0.0327)  &6.3488 (0.0242)  &8.7974 (0.0370)\\
3   &3.4698 (0.0141)  &4.1308 (0.0177)  &4.9585 (0.0245)  &6.1176 (0.0136)  &7.9207 (0.0316)\\
4   &3.6505 (0.0225)  &4.2096 (0.0386)  &4.9392 (0.0347)  &5.9195 (0.0341)  &7.4209 (0.0162)\\
5   &3.7421 (0.0134)  &4.2513 (0.0165)  &4.8967 (0.0297)  &5.7729 (0.0170)  &7.0725 (0.0250)\\
6   &3.7850 (0.0149)  &4.2588 (0.0242)  &4.8611 (0.0160)  &5.6449 (0.0162)  &6.8062 (0.0282)\\
7   &3.8274 (0.0119)  &4.2686 (0.0102)  &4.8179 (0.0117)  &5.5447 (0.0114)  &6.6049 (0.0158)\\
8   &3.8468 (0.0110)  &4.2591 (0.0124)  &4.7747 (0.0176)  &5.4519 (0.0087)  &6.4370 (0.0085)\\
9   &3.8610 (0.0081)  &4.2503 (0.0125)  &4.7430 (0.0120)  &5.3688 (0.0085)  &6.2829 (0.0112)\\
10  &3.8663 (0.0089)  &4.2423 (0.0126)  &4.6967 (0.0102)  &5.3015 (0.0103)  &6.1574 (0.0116)\\
11  &3.8707 (0.0089)  &4.2284 (0.0109)  &4.6693 (0.0078)  &5.2387 (0.0111)  &6.0473 (0.0122)\\
12  &3.8763 (0.0063)  &4.2228 (0.0072)  &4.6370 (0.0139)  &5.1805 (0.0050)  &5.9502 (0.0044)\\
13  &3.8808 (0.0048)  &4.2089 (0.0083)  &4.6096 (0.0055)  &5.1315 (0.0092)  &5.8654 (0.0070)\\
14  &3.8761 (0.0048)  &4.1920 (0.0112)  &4.5816 (0.0069)  &5.0845 (0.0058)  &5.7870 (0.0106)\\
15  &3.8806 (0.0048)  &4.1901 (0.0063)  &4.5592 (0.0088)  &5.0372 (0.0076)  &5.7158 (0.0119)\\
16  &3.8774 (0.0032)  &4.1778 (0.0060)  &4.5397 (0.0049)  &4.9985 (0.0087)  &5.6483 (0.0083)\\
17  &3.8760 (0.0043)  &4.1682 (0.0035)  &4.5149 (0.0060)  &4.9652 (0.0051)  &5.5919 (0.0078)\\
18  &3.8708 (0.0083)  &4.1584 (0.0026)  &4.4951 (0.0030)  &4.9273 (0.0043)  &5.5319 (0.0083)\\
19  &3.8694 (0.0055)  &4.1464 (0.0037)  &4.4755 (0.0039)  &4.8961 (0.0027)  &5.4838 (0.0031)\\
20  &3.8687 (0.0051)  &4.1364 (0.0041)  &4.4555 (0.0072)  &4.8673 (0.0042)  &5.4371 (0.0076)\\
21  &3.8672 (0.0064)  &4.1268 (0.0049)  &4.4401 (0.0052)  &4.8360 (0.0045)  &5.3927 (0.0027)\\
22  &3.8637 (0.0046)  &4.1215 (0.0037)  &4.4253 (0.0050)  &4.8136 (0.0043)  &5.3499 (0.0060)\\
23  &3.8610 (0.0032)  &4.1110 (0.0039)  &4.4093 (0.0057)  &4.7870 (0.0059)  &5.3118 (0.0056)\\
24  &3.8571 (0.0033)  &4.1015 (0.0051)  &4.3942 (0.0060)  &4.7650 (0.0026)  &5.2732 (0.0054)\\
25  &3.8541 (0.0054)  &4.0931 (0.0052)  &4.3803 (0.0032)  &4.7416 (0.0024)  &5.2453 (0.0045)
\end{tabular}

\end{table}

In summary, a few measurements of distance, being dispersion and contrast, are shown to agree with the consistent sense of variance and homoskedasticity.
The dispersion measurement can be used to test 
if a pixel belongs to a ``known'' class of physical scatterer.
The consistent sense of contrast can be used to test if any pair of measured data points belong to  the same homogeneous class.
An example application of this is described in our fMLE speckle filter.
The consistent sense 
of contrast also gives rise to consistent measures of sample variances, which for example, can be used to test 
if a group of pixels can form a homogeneous area.
%All of these applications are presented in chapter \ref{chap:applications}
Chapter \ref{chap:applications} presents several of these applications which further validates the models proposed herein.
